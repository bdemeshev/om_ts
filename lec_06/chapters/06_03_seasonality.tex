% !TEX root = ../om_ts_06.tex

\begin{frame} % название фрагмента

\videotitle{Сезонная ARIMA}

\end{frame}



\begin{frame}{Сезонная ARIMA: план}
  \begin{itemize}[<+->]
    \item ARMA должна быть экономной! 
    \item Сезонные полиномы.
    \item Нужно ли переходить к разностям?
  \end{itemize}

\end{frame}


\begin{frame}
  \frametitle{Сезонность и $ARIMA$}

  С помощью $ARMA$ и $ARIMA$ моделей можно моделировать сезонность!

  \pause
  Только \alert{дорого}!\pause
  \[
   MA(12):     y_t = c + u_t + a_1 u_{t-1} + a_2 u_{t-2} + \ldots + a_{12} u_{t-12}.
  \]
  \[
   ARIMA(12, 1, 0):     \Delta y_t = c + u_t  + b_1 \Delta y_{t-1} + \ldots + b_{12} \Delta y_{t-12}.
  \]

\end{frame}



\begin{frame}
  \frametitle{ARMA должна быть экономной!}

  Сосредоточимся на коэффициентах \alert{сильнее отличных} от нуля!
  \pause
\begin{block}{Определение}
Если стационарную $ARMA$ модель для $y_t$ можно записать с меньшим числом параметров в виде
\[
P_{non}(L)P_{seas}(L^{12}) y_t = c + Q_{non}(L) Q_{seas}(L^{12}) u_t,
\]
где степени у лаговых полиномов равны $\deg P_{non} =p$, $\deg P_{seas} =P$, $\deg Q_{non} =q$, $\deg Q_{seas} =Q$, 
то она также называется $SARMA(p, q)(P, Q)[12]$.
\end{block}
\end{frame}


\begin{frame}
  \frametitle{Примеры}

  \begin{itemize}[<+->]
    \item $SARMA(1,0)(0,2)[12]$
    \[
    (1 - b_1 L) y_t = c + (1 + d_1 L^{12} + d_2 L^{24}) u_t;  
    \]
    \item $SARMA(0,2)(1,0)[12]$
    \[
    (1 - f_1 L^{12}) y_t = c + (1 + a_1 L + a_2 L^2) u_t;  
    \]
    \item $SARMA(1,2)(2,1)[12]$
    \[
    (1 - f_1 L^{12} - f_2 L^{24}) (1 - b_1 L^1) y_t = c + (1 + a_1 L + a_2 L^2) (1 + d_1 L^{12}) u_t;  
    \]
  \end{itemize}

  

\end{frame}





\begin{frame}
  \frametitle{SARIMA}

  По аналогии с разностью $\Delta y_t = y_t - y_{t-1}$ можно рассмотреть сезонную разность $\Delta_{12} y_t = y_t - y_{t-12}$. 

  \pause 
  \begin{block}{Определение}
    Если ряд $z_t = \Delta^d \Delta^D_{12} y_t$ описывается стационарной моделью $SARMA(p, q)(P, Q)[12]$,
    то говорят, что $y_t$ описывается моделью $SARIMA(p, d, q)(P, D, Q)[12]$.
  \end{block}
  \pause
  $d$ — количество взятий обычной разности $\Delta = 1 - L$;

  $D$ — количество взятий сезонной разности $\Delta_{12} = 1- L^{12}$;
  \pause
  $y_t \sim SARIMA(0, 0, 2)(1, 1, 2)[12]$ означает, что 
  
  $\Delta_{12} y_t \sim SARMA(0, 2)(1, 2)[12]$
\end{frame}

\begin{frame}
  \frametitle{Как выбрать?}
  
  $SARIMA(p, 0, q)(P, 0, Q)$ или $SARIMA(p, 0, q)(P, 1, Q)[12]$?

  \begin{itemize}
    \onslide<2->{\item Посмотреть на \alert{график}!}
    
    \onslide<3->{Слишком выраженная сезонность — повод перейти к $\Delta_{12}y_t$.}

    \onslide<4->{\item Оценить все эти модели и выбрать наилучшую по \alert{кросс-валидации}.}
    
    \onslide<5->{Затратно по времени!}

    \onslide<6->{\item \alert{Применять $AIC$ нельзя}!}

    \onslide<7->{Условная и безусловная функции правдоподобия содержат разное число слагаемых.}
    
    \onslide<8->{\item Есть \alert{тесты на единичный корень}!}
    
    \onslide<9->{И эмпирические правила\ldots}
  \end{itemize}

\end{frame}


\begin{frame}
  \frametitle{STL разложение и сила сезонности}

  \onslide<1->{Шаг 1. Находим $STL$ разложение ряда $(y_t)$.

  \[
     y_t = trend_t + seas_t + remainder_t
  \]}
  
  \onslide<2->{Шаг 2. Рассчитываем силу сезонности.

  \[
    F_{seas} = \max\left\{1 - \frac{\sVar(remainder)}{\sVar(seas + remainder)}, 0 \right\}.
  \]}
  \onslide<3->{Шаг 3. Если сила сезонности выше порога, то переходим к $\Delta_{12} y_t = y_t - y_{t-12}$.}

\end{frame}


\begin{frame}
  \frametitle{Сезонная ARIMA: итоги}
  \begin{itemize}[<+->]
  \item Сезонная ARIMA \alert{экономит} параметры.
  \item Сила сезонности из \alert{STL} разложения используется для решения о необходимости сезонной разности $\Delta_12 y_t$.
  \end{itemize}
\end{frame}



