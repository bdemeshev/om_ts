% arara: xelatex
%% arara: xelatex


% https://koalatea.io/r-knn-regression/
% http://freerangestats.info/blog/2017/04/09/propensity-v-regression
% https://economics.stackexchange.com/questions/45335/what-is-the-difference-between-ate-and-att
% https://kosukeimai.github.io/MatchIt/articles/matching-methods.html


\documentclass[14pt,xcolor=dvipsnames]{beamer}


% !TEX root = om_metrics_14.tex

%\usepackage{epsdice} % dice 1-6 for probability :)

% \usepackage[absolute,overlay]{textpos}

% \usefonttheme[onlymath]{serif}

\usefonttheme{professionalfonts}
% by default beamer changes math fonts for better visibility for projection
% this professionalfonst theme removes this behavior


\usepackage[orientation=portrait,size=custom,width=25.4,height=19.05]{beamerposter}




%25,4 см 19,05 см размеры слайда в powerpoint

\usetheme{metropolis}
\metroset{
  %progressbar=none,
  numbering=none,
  subsectionpage=progressbar,
  block=fill
}

%\usecolortheme{seahorse}

\usepackage{xunicode} % хак для акцентов!
% https://tex.stackexchange.com/questions/28003/

\usepackage{fontspec}
\usepackage{polyglossia}
\setmainlanguage{russian}


% \usepackage{fontawesome5} % removed [fixed]
\setmainfont[Ligatures=TeX]{Myriad Pro}
% \setsansfont{Myriad Pro}




% why do we need \newfontfamily:
% http://tex.stackexchange.com/questions/91507/
\newfontfamily{\cyrillicfonttt}{Myriad Pro}
\newfontfamily{\cyrillicfont}{Myriad Pro}
%\newfontfamily{\cyrillicfontbs}{Myriad Pro}
\newfontfamily{\cyrillicfontsf}{Myriad Pro}


% https://tex.stackexchange.com/questions/175860/why-does-unicode-math-break-the-kerning-of-accents-in-combination-with-amssymb
% "You shouldn't be using amssymb together with unicode-math"
\usepackage{amsmath}
\usepackage{amsthm} % amssymb 


% https://tex.stackexchange.com/questions/483722/
% \usepackage[MnSymbol]{mathspec}  % Includes amsmath.
% \usepackage{mathspec}  % Includes amsmath.
% \setmathsfont(Digits,Latin,Greek,Symbols)[Numbers={Lining,Proportional}]{Latin Modern Math}
% mathspec must be loaded earlier than amsmath



%\usepackage{bm}

% \usepackage{fdsymbol} % \nperp

% \usepackage{unicode-math} % \symbf
% \setmathfont{Latin Modern Math}



\usepackage{centernot}

\usepackage{graphicx}

\usepackage{wrapfig}
% \usepackage{animate} % animations :)
% \usepackage{tikz}
%\usetikzlibrary{shapes.geometric,patterns,positioning,matrix,calc,arrows,shapes,fit,decorations,decorations.pathmorphing}
% \usepackage{pifont}
\usepackage{comment}
\usepackage[font=small,labelfont=bf]{caption}
\captionsetup[figure]{labelformat=empty}
% \includecomment{techno}



%Расположение

\setbeamersize{text margin left=15 mm,text margin right=5mm} 
\setlength{\leftmargini}{38 pt}

%\usepackage{showframe}
%\usepackage{enumitem}
% \setlist{leftmargin=5.5mm}


%Цвета от дирекции

\definecolor{dirblack}{RGB}{58, 58, 58}
\definecolor{dirwhite}{RGB}{245, 245, 245}
\definecolor{dirred}{RGB}{149, 55, 53}
\definecolor{dirblue}{RGB}{0, 90, 171}
\definecolor{dirorange}{RGB}{235, 143, 76}
\definecolor{dirlightblue}{RGB}{75, 172, 198}
\definecolor{dirgreen}{RGB}{155, 187, 89}
\definecolor{dircomment}{RGB}{128, 100, 162}

\setbeamercolor{title separator}{bg=dirlightblue!50, fg=dirblue}

%Цвета блоков

% Голубой блок!
\setbeamercolor{block title}{bg=dirblue!30,fg=dirblack}
\setbeamercolor{block title example}{bg=dirlightblue!50,fg=dirblack}
\setbeamercolor{block body example}{bg=dirlightblue!20,fg=dirblack}

\AtBeginEnvironment{exampleblock}{\setbeamercolor{itemize item}{fg=dirblack}}
%\setbeamertemplate{blocks}[rounded][shadow]

% Набор команд для удобства верстки

% Набор команд для структуризации

%\newcommand{\quest}{\faQuestionCircleO}
%\faPencilSquareO \faPuzzlePiece \faQuestionCircleO  \faIcon*[regular]{file} {\textcolor{dirblue}
%\newcommand{\quest}{\textcolor{dirblue}{\boxed{\textbf{?}}}
%\newcommand{\task}{\faIcon{tasks}}
%\newcommand{\exmpl}{\faPuzzlePiece}
%\newcommand{\dfn}{\faIcon{pen-square}}
%\newcommand{\quest}{\textcolor{dirblue}{\faQuestionCircle[regular]}}
%\newcommand{\acc}[1]{\textcolor{dirred}{#1}}
%\newcommand{\accm}[1]{\textcolor{dirred}{#1}}
%\newcommand{\acct}[1]{\textcolor{dirblue}{#1}}
%\newcommand{\acctm}[1]{\textcolor{dirblue}{#1}}
%\newcommand{\accex}[1]{\textcolor{dirblack}{\bf #1}}
%\newcommand{\accexm}[1]{\textcolor{dirblack}{ \mathbf{#1}}}
%\newcommand{\acclp}[1]{\textcolor{dirorange}{\it #1}}
\newcommand{\todo}[1]{\textcolor{dircomment}{\bf #1}}
%\newcommand{\graylink}[1]{{\fontsize{11}{12}\selectfont \textcolor{gray}{#1}}}
%\newcommand{\figcaption}[1]{{\fontsize{18}{20}\selectfont #1}}


\newcommand{\videotitle}[1]{
    {\fontsize{33}{30}\selectfont \textcolor{dirblue}{\textbf{#1}} }

    %\todo{название видеофрагмента}
}

\newcommand{\lecturetitle}[1]{
  {\fontsize{33}{30}\selectfont \textcolor{dirblue}{\textbf{#1}} }

    %\todo{название лекции}
}





%\newcommand{\spcbig}{\vspace{-10 pt}}
%\newcommand{\spcsmall}{\vspace{-5 pt}}

%\usepackage{listings}
%\lstset{
%xleftmargin=0 pt,
%  basicstyle=\small, 
%  language=Python,
  %tabsize = 2,
%  backgroundcolor=\color{mc!20!white}
%}



%\newcommand{\mypart}[1]{\begin{frame}[standout]{\huge #1}\end{frame}}

\setbeamercolor{background canvas}{bg=}

% frame title setup
\setbeamercolor{frametitle}{bg=,fg=dirblue}
\setbeamertemplate{frametitle}[default][left]

\addtobeamertemplate{frametitle}{\hspace*{0.1 cm}}{\vspace*{0.25cm}}


%Шрифты
\setbeamerfont{frametitle}{family=\rmfamily,series=\bfseries,size={\fontsize{33}{30}}}
\setbeamerfont{framesubtitle}{family=\rmfamily,series=\bfseries,size={\fontsize{26}{20}}}


% удобнее знать номер слайда, чтобы вносить правки!  

\setbeamercolor{footline}{fg=dircomment}
\setbeamerfont{footline}{series=\bfseries, size={\fontsize{12}{14}}}
%\setbeamertemplate{footline}[page number]


\defbeamertemplate{footline}{custom footline}
{%
  \hspace*{\fill}%
  \usebeamercolor[fg]{page number in head/foot}%
  \usebeamerfont{page number in head/foot}%
  page: \insertpagenumber\,/\,\insertpresentationendpage%
  \hspace{20pt}%
  slide: \insertframenumber\,/\,\inserttotalframenumber%
  %\hspace*{\fill}
  \vskip2pt%
}
%\setbeamertemplate{footline}[custom footline]

\usepackage{physics}
\usepackage[makeroom]{cancel}



% tikz block

\usepackage{pgfplots}
\pgfplotsset{compat=newest}

\usepackage{tikz}
\usetikzlibrary{calc}
\usetikzlibrary{quotes,angles}
\usetikzlibrary{arrows}
\usetikzlibrary{arrows.meta}
\usetikzlibrary{positioning,intersections,decorations.markings}
\usetikzlibrary{patterns}

\usepackage{tkz-euclide} 
%\tikzset{>=latex}

\tikzset{cross/.style={cross out, draw=black, minimum size=2*(#1-\pgflinewidth), inner sep=0pt, outer sep=0pt},
%default radius will be 1pt. 
cross/.default={5pt}}

\colorlet{veca}{red}
\colorlet{vecb}{blue}
\colorlet{vecc}{olive}


\newcommand{\grid}{\draw[color=gray,step=1.0,dotted] (-2.1,-2.1) grid (9.6,6.1)}

% end tikz block

\newcommand{\R}{\mathbb{R}}
\newcommand{\Rot}{\mathrm{R}}
\newcommand{\HH}{\mathrm{H}}
\newcommand{\Id}{\mathrm{I}}
\newcommand{\RR}{\mathbb{R}}
\newcommand{\ZZ}{\mathbb{Z}}
\newcommand{\la}{\lambda}
\let\P\relax
\newcommand{\P}{\mathbb{P}}
\newcommand{\E}{\mathbb{E}}

\newcommand{\cN}{\mathcal{N}}
\newcommand{\dN}{\mathcal{N}}

\newcommand{\qL}{q_{\text{left}}}
\newcommand{\qR}{q_{\text{right}}}



\newcommand{\ba}{\mathbf{a}}
\newcommand{\be}{\mathbf{e}}
\newcommand{\bb}{\mathbf{b}}
\newcommand{\bc}{\mathbf{c}}
\newcommand{\bd}{\mathbf{d}}
\newcommand{\bx}{\mathbf{x}}
\newcommand{\bff}{\mathbf{f}} % \bf is already def
\newcommand{\bv}{\mathbf{v}}
\newcommand{\bzero}{\mathbf{0}}



\DeclareMathOperator{\Var}{Var}
\DeclareMathOperator{\sVar}{sVar}
\DeclareMathOperator{\Cov}{Cov}
\DeclareMathOperator{\sCov}{sCov}
\DeclareMathOperator{\sCorr}{sCorr}
\DeclareMathOperator{\pCorr}{pCorr}
\DeclareMathOperator{\Corr}{Corr}
\DeclareMathOperator{\Med}{Med}
\let\L\relax
\DeclareMathOperator{\L}{L}


\DeclareMathOperator{\plim}{plim}
\DeclareMathOperator{\sign}{sign}


\newcommand{\graylink}[1]{{\fontsize{11}{12}\selectfont \textcolor{gray}{#1}}}
\newcommand{\figcaption}[1]{{\fontsize{18}{20}\selectfont #1}}





\begin{document}


\begin{frame} % название лекции


\lecturetitle{ARIMA и сезонная ARIMA}

\end{frame}


% !TEX root = ../om_ts_06.tex

\begin{frame} % название фрагмента

\videotitle{Буковка I}

\end{frame}



\begin{frame}{Буковка I: план}
  \begin{itemize}[<+->]
    \item Стационарность $ARMA$. 
    \item Определение $ARIMA$.
    \item Нужно ли переходить к разностям?
  \end{itemize}

\end{frame}


\begin{frame}
  \frametitle{$ARMA$ процесс}

  \begin{block}{Определение}
    $ARMA(p, q)$ процессом с уравнением 
    \[
      y_t = c + \beta_1 y_{t-1} + \ldots + \beta_p y_{t-p} + u_t + \alpha_1 u_{t-1} + \ldots + \alpha_q u_{t-q}, 
    \]
    где $(u_t)$ — белый шум, $\beta_p \neq 0$ и $\alpha_q \neq 0$, уравнение несократимо, называется 
    решение этого уравнения вида $MA(\infty)$ относительно $(u_t)$.
  \end{block}

  \pause  
  \begin{block}{Определение с лагами}
    $ARMA(p,q)$ процессом с уравнением 
    \[
      P(L)y_t = c + Q(L)u_t, 
    \]
    где $(u_t)$ — белый шум, $P(L)$ имеет степень $p$, $Q(L)$ имеет степень $q$, и $P(0)=Q(0)=1$, 
    $P(Q)$ и $Q(L)$ несократимы, называется решение этого уравнения вида $MA(\infty)$ относительно $(u_t)$.  
  \end{block}
\end{frame}

\begin{frame}
  \frametitle{Нюансы}

  \begin{itemize}[<+->]
    \item Процесс $y_t \sim ARMA(p, q)$ стационарен \alert{по определению}:
    
    $\E(y_t) = \mu_y$, $\Var(y_t) = \gamma_0$, $\Cov(y_t, y_{t-k}) = \gamma_k$.

    \item В \alert{канонической записи} $ARMA(p, q)$ процесса $P(L) y_t = c+ Q(L) u_t$ у полинома $P(L)$
    все корни $\abs{\ell} > 1$. 

    Возможны неканонические варианты.   

    \item При оценке $ARMA(p, q)$ процесса методом максимального правдоподобия эти ограничения наложены \alert{а-приори}. 
    
    Есть упрощённые варианты правдоподобия.
  \end{itemize}

\end{frame}


\begin{frame}
  \frametitle{Что делать с нестационарными процессами?}

  \begin{block}{Определение}
    Случайный процесс $(y_t)$ называется $ARIMA(p, 1, q)$ процессом относительно белого шума $(u_t)$, 
    если $(y_t)$ нестационарен, но $\Delta y_t$ — стационарный $ARMA(p, q)$ процесс относительно белого шума $(u_t)$.  
  \end{block}

  \pause

  \begin{block}{Определение}
    Случайный процесс $(y_t)$ называется $ARIMA(p, 2, q)$ процессом относительно белого шума $(u_t)$, 
    если $(y_t)$ нестационарен, но $\Delta^2 y_t$ — стационарный $ARMA(p, q)$ процесс относительно белого шума $(u_t)$.  
  \end{block}

  \pause
  $\Delta y_t = y_t - y_{t-1}$, $\Delta^2 y_t = \Delta y_t - \Delta y_{t-1}$

  \pause 
  ARIMA — \alert{A}uto\alert{R}egressive \alert{I}ntegrated \alert{M}oving \alert{A}verage
  
\end{frame}

\begin{frame}
  \frametitle{Как выбрать?}
  
  $y_t \sim ARMA(p, q)$ или $\Delta y_t \sim ARMA(p, q)$ или $\Delta^2 y_t \sim ARMA(p, q)$
\pause
  \begin{itemize}[<+->]
    \item Посмотреть на график!
    
    \pause График стационарного процесса колеблется в полосе постоянной ширины вокруг своего ожидания.

    \item Оценить все эти модели и выбрать наилучшую по кросс-валидации.
    
    \pause Затратно по времени!

    \item Применять $AIC$ нельзя!
    
    \pause $\ln L(y_1, \ldots, y_n \mid \theta)$ и $\ln L(y_2, \ldots, y_n \mid \theta, y_1)$ и $\ln L(y_3, \ldots, y_n \mid \theta, y_1, y_2)$

    несравнимы!
    \item Есть тесты на единичный корень!
    
    \pause ADF, KPSS, PP, \ldots
  \end{itemize}

\end{frame}

\begin{frame}
  \frametitle{Выбираем «на глазок»}

  четыре небольших графика на одном слайде

  явно стационарный (x)

  случайное блуждание (y)

  явно с трендом (z)

  что-то спорное (w)
  
\end{frame}



\begin{frame}{Буковка I: итоги}

  \begin{itemize}[<+->]
    \item $ARMA$ модель подходит только для \alert{стационарных} рядов. 
    \item Иногда стационарен $\Delta y_t$ или $\Delta^2 y_t$. 
    \item Выбираем между $ARMA$ и $ARIMA$.
  \end{itemize}
\end{frame}



% !TEX root = ../om_ts_06.tex

\begin{frame} % название фрагмента

\videotitle{ADF тест}

\end{frame}



\begin{frame}{ADF тест: план}
  \begin{itemize}[<+->]
    \item ADF тест. 
    \item Три вариации теста.
  \end{itemize}

\end{frame}


\begin{frame}
  \frametitle{Зачем нужен ADF тест?}

  Хотим ответить на вопросы:
  \pause
  \begin{itemize}[<+->]
    \item Использовать $ARMA$ модель для $(y_t)$ или для $(\Delta y_t)$?
    \item Как включать константу в модель?
  \end{itemize}

  \pause
  Название «тест на единичные корни»
  \[
  \Delta = 1 - L = P(L) 
  \]
  Уравнение $1 - \ell = 0$ имеет корень $\ell =1$.

\end{frame}

\begin{frame}
  \frametitle{ADF тест}
  
  \begin{block}{Расшифровка}
    Augmented Dickey Fuller test, расширенный тест Дики-Фуллера  
  \end{block}

  \pause 
  Три вариации теста: без константы, с константой, c трендом.
  
\end{frame}


\begin{frame}
  \frametitle{ADF с константой}

  Предположения:

  \[
  \Delta y_t = c + \beta y_{t-1} + d_1 \Delta y_{t-1} + \ldots + d_k \Delta y_{t-p} + u_t,  
  \]

  \pause

  $H_0$: $\beta = 0$;
  
  $\Delta y_t$ — стационарный $AR(p)$ процесс;

  $y_t = y_0 + ct + \sum_{i=1}^t (\Delta y_t - \E(\Delta y_t))$;

  \pause

  $H_a$: $\beta < 0$;

  $y_t$ — стационарный $AR(p + 1)$ процесс;

\end{frame}

\begin{frame}
  \frametitle{ADF с константой: $H_0$ и $H_a$}


  тут два графика,


  слева $H_0$

  $\Delta y_t$ — стационарный $AR(p)$ процесс;

  $y_t = y_0 + ct + \sum_{i=1}^t (\Delta y_t - \E(\Delta y_t))$;


  справа $H_a$

  $y_t$ — стационарный $AR(p + 1)$ процесс c ненулевым ожиданием

\end{frame}

\begin{frame}
  \frametitle{ADF с константой: алгоритм}

  Шаг 1. Оцениваем \alert{регрессию}
  \[
    \widehat{\Delta y_t} = \hat c + \hat \beta y_{t-1} + \hat d_1 \Delta y_{t-1} + \ldots + \hat d_k \Delta y_{t-p}.  
  \]

  \pause
  Шаг 2. Считаем по \alert{классической формуле} $t$-статистики
  \[
  ADF = \frac{\hat \beta -  0}{se(\hat \beta)}.  
  \]

  \pause
  При верной $H_0$ распределение $ADF$-статистики стремится к \alert{особому распределению} $DF^c$!

  \pause 
  Шаг 3. Делаем вывод:
  
  Если $ADF < DF^c$, то $H_0$ отвергается. 

\end{frame}


\begin{frame}
  \frametitle{ADF без константы}


  Предположения:

  \[
  \Delta y_t = \beta y_{t-1} + d_1 \Delta y_{t-1} + \ldots + d_k \Delta y_{t-p} + u_t,  
  \]

  \pause

  $H_0$: $\beta = 0$;
  
  $\Delta y_t$ — стационарный $AR(p)$ процесс c $\E(\Delta y_t) = 0$;

  $y_t = y_0 + \sum_{i=1}^t \Delta y_t $;

  \pause

  $H_a$: $\beta < 0$;

  $y_t$ — стационарный $AR(p + 1)$ процесс с $\E(y_t) = 0$;

  \pause 

  В алгоритме будет \alert{регрессия без константы} и другое распределение $DF^0$.

\end{frame}


\begin{frame}
  \frametitle{ADF с трендом}


  Предположения:

  \[
  \Delta y_t = c + g t + \beta y_{t-1} + d_1 \Delta y_{t-1} + \ldots + d_k \Delta y_{t-p} + u_t,  
  \]

  \pause

  $H_0$: $\beta = 0$;
  
  $\Delta y_t = k_1 + k_2 t + x_t$;
  
  $x_t$ — стационарный $AR(p)$ процесс c $\E(x_t) = 0$;

  $y_t = y_0 + m_1 t + m_2 t^2 + \sum_{i=1}^t x_t$;

  \pause

  $H_a$: $\beta < 0$;

  $y_t = m_1 + m_2 t + x_t$;
  
  $x_t$ — стационарный $AR(p + 1)$ процесс с $\E(x_t) = 0$;

  \pause 

  В алгоритме будет регрессия \alert{с константой и трендом} и другое распределение $DF^{ct}$.

\end{frame}

\begin{frame}{$ADF$ тест: итоги}

  \begin{itemize}[<+->]
    \item Применим для принятия решения о переходе к $\Delta y_t$.
    \item Есть три варианта теста с разными предпосылками.
  \end{itemize}
\end{frame}




% !TEX root = ../om_ts_06.tex

\begin{frame} % название фрагмента

  \videotitle{KPSS тест}
  
  \end{frame}
  
  
  
  \begin{frame}{KPSS тест: план}
    \begin{itemize}[<+->]
      \item Долгосрочная дисперсия.
      \item Предпосылки теста.
      \item Две вариации теста.
    \end{itemize}
  
  \end{frame}
  
  
  \begin{frame}
    \frametitle{Зачем нужен KPSS тест?}
  
    Хотим ответить на вопросы:
    \pause
    \begin{itemize}[<+->]
      \item Использовать $ARMA$ модель для $(y_t)$ или для $(\Delta y_t)$?
      \item Как включать константу в модель?
    \end{itemize}
    
  \end{frame}
  
  \begin{frame}
    \frametitle{KPSS тест}
    
    \begin{block}{Расшифровка}
      Kwiatkowski–Phillips–Schmidt–Shin test
      
      Тест Квятковского-Филлипса-Шмидта-Шина
    \end{block}
  
    \pause 
    Две вариации теста: с константой, c трендом.
    
  \end{frame}


  \begin{frame}
    \frametitle{Долгосрочная дисперсия}

    \begin{block}{Определение}
      Для стационарного процесса $(y_t)$ величина $\lambda^2$ называется \alert{долгосрочной дисперсией}, если
      \[
        \Var(\bar y) = \frac{\lambda^2}{T} + o(1/T)
      \]
      или 
      \[
        \lim_{T \to \infty} T \Var(\bar y) = \lambda^2, 
      \]
      где $\bar y = (y_1 + \ldots + y_T) / T$.
    \end{block}

    \pause 

    \begin{block}{Мотивация}
      Для независимых наблюдений с одинаковой дисперсией 
      \[
        \Var(\bar y) = \frac{\sigma^2}{T},\text{ где }\sigma^2 = \Var(y_i).
      \]
  \end{block}  
  
  \end{frame}
  
  
  \begin{frame}
    \frametitle{KPSS с константой}
    \[
      y_t = c + rw_t + x_t,
    \]
  
    \pause
  
    \alert{$H_0$: $rw_t = 0$};
    
    $(x_t)$ — стационарный процесс с $\E(x_t) = 0$;
    
    \pause
  
    \alert{$H_a$: $rw_t = rw_{t-1} + u_t$};

    $rw_0 = 0$;
  
    $(x_t)$ — стационарный процесс с $\E(x_t) = 0$;

    $(u_t)$ — белый шум, независимый с $(x_t)$.
  
  \end{frame}
  
  \begin{frame}
    \frametitle{KPSS с константой: $H_0$ и $H_a$}
  
  
    тут два графика,
  
  
    слева $H_0$
    
    справа $H_a$
  
  
  \end{frame}
  
  \begin{frame}
    \frametitle{KPSS с константой: алгоритм}
  
    Шаг 1. Оцениваем \alert{регрессию на константу} 
    \[
      \widehat{y_t} = \hat c.  
    \]
  
    \pause
    Шаг 2. Считаем $KPSS$ статистику
    \[
    KPSS = \frac{\sum_{t=1}^T S_t^2}{T^2 \hat \lambda^2},
    \]
    где $S_t$ — накопленная сумма остатков, $S_t = \hat u_1 + \ldots + \hat u_T$,

    а $\hat\lambda^2$ — состоятельная оценка долгосрочной дисперсии. 
  
    \pause
    При верной $H_0$ распределение $KPSS$-статистики стремится к \alert{особому распределению} $KPSS^c$!
  
    \pause 
    Шаг 3. Делаем вывод:
    
    Если $KPSS > KPSS^c$, то $H_0$ отвергается. 
  
  \end{frame}
  
  
  \begin{frame}
    \frametitle{KPSS с трендом}
    \[
      y_t = c + bt + rw_t + x_t,
    \]
  
    \pause
  
    \alert{$H_0$: $rw_t = 0$};
    
    $(x_t)$ — стационарный процесс с $\E(x_t) = 0$;
    
    \pause
  
    \alert{$H_a$: $rw_t = rw_{t-1} + u_t$};

    $rw_0 = 0$;
  
    $(x_t)$ — стационарный процесс с $\E(x_t) = 0$;

    $(u_t)$ — белый шум, независимый с $(x_t)$.
  
    \pause 
  
    В алгоритме будет регрессия \alert{с константой и трендом} и другое распределение $KPSS^{ct}$.
  
  \end{frame}
  
  
  \begin{frame}
    \frametitle{$KPSS$ с трендом: $H_0$ и $H_a$}
  
  
    тут два графика,
  
  
    слева $H_0$
  
  
    справа $H_a$
  
  
  \end{frame}
  

  \begin{frame}
    \frametitle{Устоявшаяся терминология:}
    \[
      A. y_t = a + bt + x_t;
    \]

    $(y_t)$ — \alert{стационарный вокруг тренда} (trend stationary).

    $(x_t)$ — стационарный процесс с $\E(x_t) = 0$.

    \pause Рецепт: оценим регрессию $a + bt$ с $ARMA$ ошибками для $(y_t)$.
    \pause 
    \[
      B. y_t = a + \sum_{i=1}^t x_i \text{ или } y_t = a + bt + \sum_{i=1}^t x_i
    \]

    $(x_t)$ — стационарный процесс с $\E(x_t) = 0$.

    $(y_t)$ — \alert{стационарный в разностях} (difference stationary).

    \pause Рецепт: оценим $ARMA$ для $(\Delta y_t)$.

    \pause Оба $(y_t)$ нестационарны!
    
  \end{frame}


  
  \begin{frame}{$KPSS$ тест: итоги}
  
    \begin{itemize}[<+->]
      \item Применим для принятия решения о переходе к $\Delta y_t$.
      \item Есть два варианта теста с разными предпосылками.
    \end{itemize}
  \end{frame}
  
  
  
  


% !TEX root = ../om_ts_06.tex

\begin{frame} % название фрагмента

\videotitle{Сезонная ARIMA}

\end{frame}



\begin{frame}{Сезонная ARIMA: план}
  \begin{itemize}[<+->]
    \item ARMA должна быть экономной! 
    \item Сезонные полиномы.
    \item Нужно ли переходить к разностям?
  \end{itemize}

\end{frame}


\begin{frame}
  \frametitle{Сезонность и $ARIMA$}

  С помощью $ARMA$ и $ARIMA$ моделей можно моделировать сезонность!

  \pause
  Только \alert{дорого}!\pause
  \[
   MA(12):     y_t = c + u_t + a_1 u_{t-1} + a_2 u_{t-2} + \ldots + a_{12} u_{t-12}.
  \]
  \[
   ARIMA(12, 1, 0):     \Delta y_t = c + u_t  + b_1 \Delta y_{t-1} + \ldots + b_{12} \Delta y_{t-12}.
  \]

\end{frame}



\begin{frame}
  \frametitle{ARMA должна быть экономной!}

  Сосредоточимся на коэффициентах \alert{сильнее отличных} от нуля!
  \pause
\begin{block}{Определение}
Если стационарную $ARMA$ модель для $y_t$ можно записать с меньшим числом параметров в виде
\[
P_{non}(L)P_{seas}(L^{12}) y_t = c + Q_{non}(L) Q_{seas}(L^{12}) u_t,
\]
где степени у лаговых полиномов равны $\deg P_{non} =p$, $\deg P_{seas} =P$, $\deg Q_{non} =q$, $\deg Q_{seas} =Q$, 
то она также называется $SARMA(p, q)(P, Q)[12]$.
\end{block}
\end{frame}


\begin{frame}
  \frametitle{Примеры}

  \begin{itemize}[<+->]
    \item $SARMA(1,0)(0,2)[12]$
    \[
    (1 - b_1 L) y_t = c + (1 + d_1 L^{12} + d_2 L^{24}) u_t;  
    \]
    \item $SARMA(0,2)(1,0)[12]$
    \[
    (1 - f_1 L^{12}) y_t = c + (1 + a_1 L + a_2 L^2) u_t;  
    \]
    \item $SARMA(1,2)(2,1)[12]$
    \[
    (1 - f_1 L^{12} - f_2 L^{24}) (1 - b_1 L^1) y_t = c + (1 + a_1 L + a_2 L^2) (1 + d_1 L^{12}) u_t;  
    \]
  \end{itemize}

  

\end{frame}





\begin{frame}
  \frametitle{SARIMA}

  По аналогии с разностью $\Delta y_t = y_t - y_{t-1}$ можно рассмотреть сезонную разность $\Delta_{12} y_t = y_t - y_{t-12}$. 

  \pause 
  \begin{block}{Определение}
    Если ряд $z_t = \Delta^d \Delta^D_{12} y_t$ описывается стационарной моделью $SARMA(p, q)(P, Q)[12]$,
    то говорят, что $y_t$ описывается моделью $SARIMA(p, d, q)(P, D, Q)[12]$.
  \end{block}
  \pause
  $d$ — количество взятий обычной разности $\Delta = 1 - L$;

  $D$ — количество взятий сезонной разности $\Delta_{12} = 1- L^{12}$;
  \pause
  $y_t \sim SARIMA(0, 0, 2)(1, 1, 2)[12]$ означает, что 
  
  $\Delta_{12} y_t \sim SARMA(0, 2)(1, 2)[12]$
\end{frame}

\begin{frame}
  \frametitle{Как выбрать?}
  
  $SARIMA(p, 0, q)(P, 0, Q)$ или $SARIMA(p, 0, q)(P, 1, Q)[12]$?

  \begin{itemize}
    \onslide<2->{\item Посмотреть на \alert{график}!}
    
    \onslide<3->{Слишком выраженная сезонность — повод перейти к $\Delta_{12}y_t$.}

    \onslide<4->{\item Оценить все эти модели и выбрать наилучшую по \alert{кросс-валидации}.}
    
    \onslide<5->{Затратно по времени!}

    \onslide<6->{\item \alert{Применять $AIC$ нельзя}!}

    \onslide<7->{Условная и безусловная функции правдоподобия содержат разное число слагаемых.}
    
    \onslide<8->{\item Есть \alert{тесты на единичный корень}!}
    
    \onslide<9->{И эмпирические правила\ldots}
  \end{itemize}

\end{frame}


\begin{frame}
  \frametitle{STL разложение и сила сезонности}

  \onslide<1->{Шаг 1. Находим $STL$ разложение ряда $(y_t)$.

  \[
     y_t = trend_t + seas_t + remainder_t
  \]}
  
  \onslide<2->{Шаг 2. Рассчитываем силу сезонности.

  \[
    F_{seas} = \max\left\{1 - \frac{\sVar(remainder)}{\sVar(seas + remainder)}, 0 \right\}.
  \]}
  \onslide<3->{Шаг 3. Если сила сезонности выше порога, то переходим к $\Delta_{12} y_t = y_t - y_{t-12}$.}

\end{frame}


\begin{frame}
  \frametitle{Сезонная ARIMA: итоги}
  \begin{itemize}[<+->]
  \item Сезонная ARIMA \alert{экономит} параметры.
  \item Сила сезонности из \alert{STL} разложения используется для решения о необходимости сезонной разности $\Delta_12 y_t$.
  \end{itemize}
\end{frame}





% !TEX root = ../om_ts_05.tex

\begin{frame} % название фрагмента

\videotitle{Происхождение мифа}

\end{frame}



\begin{frame}{Происхождение мифа: план}
  \begin{itemize}[<+->]
    \item Нюансы $MA(\infty)$. 
    \item Шумы бывают разные!
    \item Выводы о структуре решений. 
  \end{itemize}

\end{frame}

\begin{frame}
    \frametitle{Распространённый миф!}
    \[
      y_t =  2y_{t-1} + u_t, \pause  \lambda_1 = 2: \text{ есть стационарное решение! }  
    \]
    Стационарное решение в студию!
    \[
    y_t =  -0.5 u_{t+1} - 0.5^2 u_{t+2} - 0.5^3 u_{t+3} - 0.5^4 u_{t+4} +\ldots 
    \]
    
  
\end{frame}

\begin{frame}
    \frametitle{Это не MA($\mathbf{\infty}$)?}

    \[
        y_t =  -0.5 u_{t+1} - 0.5^2 u_{t+2} - 0.5^3 u_{t+3} - 0.5^4 u_{t+4} +\ldots 
    \]

    \pause 
    \begin{block}{Определение}
        Процесс $(y_t)$, который \alert{можно} представить в виде
        \[
        y_t = \mu + u_t + \alpha_1 u_{t-1} + \alpha_2 u_{t-2} + \ldots,
        \]
        где $(u_t)$ — белый шум, бесконечное количество $\alpha_i \neq 0$ и 
        $\sum_{i=1}^{\infty} \alpha_i^2 < \infty$, 
        называется $MA(\infty)$ процессом. 
    \end{block}
    
\end{frame}

\begin{frame}
    \frametitle{Это MA($\mathbf{\infty}$)!}

    \[
        y_t =  -0.5 u_{t+1} - 0.5^2 u_{t+2} - 0.5^3 u_{t+3} - 0.5^4 u_{t+4} +\ldots 
    \]

    Данный \alert{можно} представить в виде
    \pause 
    \[
      y_t = \nu_t + 0.5 \nu_{t-1} + 0.5^2 \nu_{t-2} +0.5^3 \nu_{t-3} + \ldots, 
    \]
    где $(\nu_t)$ — белый шум.
    \pause 
    \[
        \nu_t = (1 - 0.5L) y_t = (1- 0.5L) (-0.5 u_{t+1} - 0.5^2 u_{t+2} - 0.5^3 u_{t+3} +\ldots )   
    \]

\end{frame}





\begin{frame}
    \frametitle{Происхождение мифа}
  
    \begin{block}{Правильная теорема}
      Если $ARMA$ уравнение $P(L) y_t = c + Q(L) u_t$ несократимо, то оно 
       имеет ровно одно стационарное решение \alert{вида $MA(\infty)$ относительно $(u_t)$}, 
        если и только если $\abs{\ell_i} > 1$ для всех корней лагового полинома $P(L)$.
    \end{block}
  
    \pause
    \begin{block}{Вариация}
      Если $ARMA$ уравнение $P(L) y_t = c + Q(L) u_t$ несократимо, то оно 
       имеет ровно одно стационарное решение \alert{вида $MA(\infty)$ относительно $(u_t)$}, 
        если и только если $\abs{\lambda_i} < 1$ для всех корней характеристического полинома $\phi_{AR}(\lambda)$.
    \end{block}
  \end{frame}
  
  
  
  \begin{frame}
    \frametitle{Разница в шумах!}
    
    Несократимое $ARMA$ уравнение $P(L) y_t = c + Q(L)u_t$.
  \pause
    \begin{itemize}[<+->]
      \item Если у $\phi_{AR}(\lambda)$ есть корень с $\abs{\lambda_i} = 1$,
      \pause
      то стационарных решений нет. 
      \pause
      \item Если у $\phi_{AR}(\lambda)$ все корни с $\abs{\lambda_i} < 1$,
      \pause
      то стационарное решение единственно и имеет вид $MA(\infty)$ относительно $(u_t)$:
      \[
        y_t = \mu + u_t + \alpha_1 u_{t-1} + \alpha_2 u_{t-2} +\ldots 
      \]
      \pause
      \item Если у $\phi_{AR}(\lambda)$ все корни с $\abs{\lambda_i} \neq 1$,
      но есть корень с $\abs{\lambda_i} > 1$,
      \pause
      то стационарное решение единственно и \alert{имеет вид $MA(\infty)$}:
      \[
        y_t = \mu +  \alert{\nu_t + \alpha_1 \nu_{t-1} + \alpha_2 \nu_{t-2} +\ldots}
      \]
  
    \end{itemize}
  
    
  
  \end{frame}
  



\begin{frame}{Происхождение мифа: итоги}

  \begin{itemize}[<+->]
    \item Слово «\alert{можно}» в $MA(\infty)$ важно. 
    \item Разница между $\abs{\lambda_i} < 1$ и $\abs{\lambda_i} > 1$.
  \end{itemize}
\end{frame}







\end{document}
