% arara: xelatex
%% arara: xelatex


% https://koalatea.io/r-knn-regression/
% http://freerangestats.info/blog/2017/04/09/propensity-v-regression
% https://economics.stackexchange.com/questions/45335/what-is-the-difference-between-ate-and-att
% https://kosukeimai.github.io/MatchIt/articles/matching-methods.html


\documentclass[14pt,xcolor=dvipsnames]{beamer}


% !TEX root = om_metrics_14.tex

%\usepackage{epsdice} % dice 1-6 for probability :)

% \usepackage[absolute,overlay]{textpos}

% \usefonttheme[onlymath]{serif}

\usefonttheme{professionalfonts}
% by default beamer changes math fonts for better visibility for projection
% this professionalfonst theme removes this behavior


\usepackage[orientation=portrait,size=custom,width=25.4,height=19.05]{beamerposter}




%25,4 см 19,05 см размеры слайда в powerpoint

\usetheme{metropolis}
\metroset{
  %progressbar=none,
  numbering=none,
  subsectionpage=progressbar,
  block=fill
}

%\usecolortheme{seahorse}

\usepackage{xunicode} % хак для акцентов!
% https://tex.stackexchange.com/questions/28003/

\usepackage{fontspec}
\usepackage{polyglossia}
\setmainlanguage{russian}


% \usepackage{fontawesome5} % removed [fixed]
\setmainfont[Ligatures=TeX]{Myriad Pro}
% \setsansfont{Myriad Pro}




% why do we need \newfontfamily:
% http://tex.stackexchange.com/questions/91507/
\newfontfamily{\cyrillicfonttt}{Myriad Pro}
\newfontfamily{\cyrillicfont}{Myriad Pro}
%\newfontfamily{\cyrillicfontbs}{Myriad Pro}
\newfontfamily{\cyrillicfontsf}{Myriad Pro}


% https://tex.stackexchange.com/questions/175860/why-does-unicode-math-break-the-kerning-of-accents-in-combination-with-amssymb
% "You shouldn't be using amssymb together with unicode-math"
\usepackage{amsmath}
\usepackage{amsthm} % amssymb 


% https://tex.stackexchange.com/questions/483722/
% \usepackage[MnSymbol]{mathspec}  % Includes amsmath.
% \usepackage{mathspec}  % Includes amsmath.
% \setmathsfont(Digits,Latin,Greek,Symbols)[Numbers={Lining,Proportional}]{Latin Modern Math}
% mathspec must be loaded earlier than amsmath



%\usepackage{bm}

% \usepackage{fdsymbol} % \nperp

% \usepackage{unicode-math} % \symbf
% \setmathfont{Latin Modern Math}



\usepackage{centernot}

\usepackage{graphicx}

\usepackage{wrapfig}
% \usepackage{animate} % animations :)
% \usepackage{tikz}
%\usetikzlibrary{shapes.geometric,patterns,positioning,matrix,calc,arrows,shapes,fit,decorations,decorations.pathmorphing}
% \usepackage{pifont}
\usepackage{comment}
\usepackage[font=small,labelfont=bf]{caption}
\captionsetup[figure]{labelformat=empty}
% \includecomment{techno}



%Расположение

\setbeamersize{text margin left=15 mm,text margin right=5mm} 
\setlength{\leftmargini}{38 pt}

%\usepackage{showframe}
%\usepackage{enumitem}
% \setlist{leftmargin=5.5mm}


%Цвета от дирекции

\definecolor{dirblack}{RGB}{58, 58, 58}
\definecolor{dirwhite}{RGB}{245, 245, 245}
\definecolor{dirred}{RGB}{149, 55, 53}
\definecolor{dirblue}{RGB}{0, 90, 171}
\definecolor{dirorange}{RGB}{235, 143, 76}
\definecolor{dirlightblue}{RGB}{75, 172, 198}
\definecolor{dirgreen}{RGB}{155, 187, 89}
\definecolor{dircomment}{RGB}{128, 100, 162}

\setbeamercolor{title separator}{bg=dirlightblue!50, fg=dirblue}

%Цвета блоков

% Голубой блок!
\setbeamercolor{block title}{bg=dirblue!30,fg=dirblack}
\setbeamercolor{block title example}{bg=dirlightblue!50,fg=dirblack}
\setbeamercolor{block body example}{bg=dirlightblue!20,fg=dirblack}

\AtBeginEnvironment{exampleblock}{\setbeamercolor{itemize item}{fg=dirblack}}
%\setbeamertemplate{blocks}[rounded][shadow]

% Набор команд для удобства верстки

% Набор команд для структуризации

%\newcommand{\quest}{\faQuestionCircleO}
%\faPencilSquareO \faPuzzlePiece \faQuestionCircleO  \faIcon*[regular]{file} {\textcolor{dirblue}
%\newcommand{\quest}{\textcolor{dirblue}{\boxed{\textbf{?}}}
%\newcommand{\task}{\faIcon{tasks}}
%\newcommand{\exmpl}{\faPuzzlePiece}
%\newcommand{\dfn}{\faIcon{pen-square}}
%\newcommand{\quest}{\textcolor{dirblue}{\faQuestionCircle[regular]}}
%\newcommand{\acc}[1]{\textcolor{dirred}{#1}}
%\newcommand{\accm}[1]{\textcolor{dirred}{#1}}
%\newcommand{\acct}[1]{\textcolor{dirblue}{#1}}
%\newcommand{\acctm}[1]{\textcolor{dirblue}{#1}}
%\newcommand{\accex}[1]{\textcolor{dirblack}{\bf #1}}
%\newcommand{\accexm}[1]{\textcolor{dirblack}{ \mathbf{#1}}}
%\newcommand{\acclp}[1]{\textcolor{dirorange}{\it #1}}
\newcommand{\todo}[1]{\textcolor{dircomment}{\bf #1}}
%\newcommand{\graylink}[1]{{\fontsize{11}{12}\selectfont \textcolor{gray}{#1}}}
%\newcommand{\figcaption}[1]{{\fontsize{18}{20}\selectfont #1}}


\newcommand{\videotitle}[1]{
    {\fontsize{33}{30}\selectfont \textcolor{dirblue}{\textbf{#1}} }

    %\todo{название видеофрагмента}
}

\newcommand{\lecturetitle}[1]{
  {\fontsize{33}{30}\selectfont \textcolor{dirblue}{\textbf{#1}} }

    %\todo{название лекции}
}





%\newcommand{\spcbig}{\vspace{-10 pt}}
%\newcommand{\spcsmall}{\vspace{-5 pt}}

%\usepackage{listings}
%\lstset{
%xleftmargin=0 pt,
%  basicstyle=\small, 
%  language=Python,
  %tabsize = 2,
%  backgroundcolor=\color{mc!20!white}
%}



%\newcommand{\mypart}[1]{\begin{frame}[standout]{\huge #1}\end{frame}}

\setbeamercolor{background canvas}{bg=}

% frame title setup
\setbeamercolor{frametitle}{bg=,fg=dirblue}
\setbeamertemplate{frametitle}[default][left]

\addtobeamertemplate{frametitle}{\hspace*{0.1 cm}}{\vspace*{0.25cm}}


%Шрифты
\setbeamerfont{frametitle}{family=\rmfamily,series=\bfseries,size={\fontsize{33}{30}}}
\setbeamerfont{framesubtitle}{family=\rmfamily,series=\bfseries,size={\fontsize{26}{20}}}


% удобнее знать номер слайда, чтобы вносить правки!  

\setbeamercolor{footline}{fg=dircomment}
\setbeamerfont{footline}{series=\bfseries, size={\fontsize{12}{14}}}
%\setbeamertemplate{footline}[page number]


\defbeamertemplate{footline}{custom footline}
{%
  \hspace*{\fill}%
  \usebeamercolor[fg]{page number in head/foot}%
  \usebeamerfont{page number in head/foot}%
  page: \insertpagenumber\,/\,\insertpresentationendpage%
  \hspace{20pt}%
  slide: \insertframenumber\,/\,\inserttotalframenumber%
  %\hspace*{\fill}
  \vskip2pt%
}
%\setbeamertemplate{footline}[custom footline]

\usepackage{physics}
\usepackage[makeroom]{cancel}



% tikz block

\usepackage{pgfplots}
\pgfplotsset{compat=newest}

\usepackage{tikz}
\usetikzlibrary{calc}
\usetikzlibrary{quotes,angles}
\usetikzlibrary{arrows}
\usetikzlibrary{arrows.meta}
\usetikzlibrary{positioning,intersections,decorations.markings}
\usetikzlibrary{patterns}

\usepackage{tkz-euclide} 
%\tikzset{>=latex}

\tikzset{cross/.style={cross out, draw=black, minimum size=2*(#1-\pgflinewidth), inner sep=0pt, outer sep=0pt},
%default radius will be 1pt. 
cross/.default={5pt}}

\colorlet{veca}{red}
\colorlet{vecb}{blue}
\colorlet{vecc}{olive}


\newcommand{\grid}{\draw[color=gray,step=1.0,dotted] (-2.1,-2.1) grid (9.6,6.1)}

% end tikz block

\newcommand{\R}{\mathbb{R}}
\newcommand{\Rot}{\mathrm{R}}
\newcommand{\HH}{\mathrm{H}}
\newcommand{\Id}{\mathrm{I}}
\newcommand{\RR}{\mathbb{R}}
\newcommand{\ZZ}{\mathbb{Z}}
\newcommand{\la}{\lambda}
\let\P\relax
\newcommand{\P}{\mathbb{P}}
\newcommand{\E}{\mathbb{E}}

\newcommand{\cN}{\mathcal{N}}
\newcommand{\dN}{\mathcal{N}}

\newcommand{\qL}{q_{\text{left}}}
\newcommand{\qR}{q_{\text{right}}}



\newcommand{\ba}{\mathbf{a}}
\newcommand{\be}{\mathbf{e}}
\newcommand{\bb}{\mathbf{b}}
\newcommand{\bc}{\mathbf{c}}
\newcommand{\bd}{\mathbf{d}}
\newcommand{\bx}{\mathbf{x}}
\newcommand{\bff}{\mathbf{f}} % \bf is already def
\newcommand{\bv}{\mathbf{v}}
\newcommand{\bzero}{\mathbf{0}}



\DeclareMathOperator{\Var}{Var}
\DeclareMathOperator{\sVar}{sVar}
\DeclareMathOperator{\Cov}{Cov}
\DeclareMathOperator{\sCov}{sCov}
\DeclareMathOperator{\sCorr}{sCorr}
\DeclareMathOperator{\pCorr}{pCorr}
\DeclareMathOperator{\Corr}{Corr}
\DeclareMathOperator{\Med}{Med}
\let\L\relax
\DeclareMathOperator{\L}{L}


\DeclareMathOperator{\plim}{plim}
\DeclareMathOperator{\sign}{sign}


\newcommand{\graylink}[1]{{\fontsize{11}{12}\selectfont \textcolor{gray}{#1}}}
\newcommand{\figcaption}[1]{{\fontsize{18}{20}\selectfont #1}}





\begin{document}


\begin{frame} % название лекции


\lecturetitle{ARIMA и сезонная ARIMA}

\end{frame}


% !TEX root = ../om_ts_06.tex

\begin{frame} % название фрагмента

\videotitle{Буковка I}

\end{frame}



\begin{frame}{Буковка I: план}
  \begin{itemize}[<+->]
    \item Стационарность $ARMA$. 
    \item Определение $ARIMA$.
    \item Нужно ли переходить к разностям?
  \end{itemize}

\end{frame}


\begin{frame}
  \frametitle{$ARMA$ процесс}

  \begin{block}{Определение}
    $ARMA(p, q)$ процессом с уравнением 
    \[
      y_t = c + \beta_1 y_{t-1} + \ldots + \beta_p y_{t-p} + u_t + \alpha_1 u_{t-1} + \ldots + \alpha_q u_{t-q}, 
    \]
    где $(u_t)$ — белый шум, $\beta_p \neq 0$ и $\alpha_q \neq 0$, уравнение несократимо, называется 
    решение этого уравнения вида $MA(\infty)$ относительно $(u_t)$.
  \end{block}

  \pause  
  \begin{block}{Определение с лагами}
    $ARMA(p,q)$ процессом с уравнением 
    \[
      P(L)y_t = c + Q(L)u_t, 
    \]
    где $(u_t)$ — белый шум, $P(L)$ имеет степень $p$, $Q(L)$ имеет степень $q$, и $P(0)=Q(0)=1$, 
    $P(Q)$ и $Q(L)$ несократимы, называется решение этого уравнения вида $MA(\infty)$ относительно $(u_t)$.  
  \end{block}
\end{frame}

\begin{frame}
  \frametitle{Нюансы}

  \begin{itemize}[<+->]
    \item Процесс $y_t \sim ARMA(p, q)$ стационарен \alert{по определению}:
    
    $\E(y_t) = \mu_y$, $\Var(y_t) = \gamma_0$, $\Cov(y_t, y_{t-k}) = \gamma_k$.

    \item В \alert{канонической записи} $ARMA(p, q)$ процесса $P(L) y_t = c+ Q(L) u_t$ у полинома $P(L)$
    все корни $\abs{\ell} > 1$. 

    Возможны неканонические варианты.   

    \item При оценке $ARMA(p, q)$ процесса методом максимального правдоподобия эти ограничения наложены \alert{а-приори}. 
    
    Есть упрощённые варианты правдоподобия.
  \end{itemize}

\end{frame}


\begin{frame}
  \frametitle{Что делать с нестационарными процессами?}

  \begin{block}{Определение}
    Случайный процесс $(y_t)$ называется $ARIMA(p, 1, q)$ процессом относительно белого шума $(u_t)$, 
    если $(y_t)$ нестационарен, но $\Delta y_t$ — стационарный $ARMA(p, q)$ процесс относительно белого шума $(u_t)$.  
  \end{block}

  \pause

  \begin{block}{Определение}
    Случайный процесс $(y_t)$ называется $ARIMA(p, 2, q)$ процессом относительно белого шума $(u_t)$, 
    если $(y_t)$ нестационарен, но $\Delta^2 y_t$ — стационарный $ARMA(p, q)$ процесс относительно белого шума $(u_t)$.  
  \end{block}

  \pause
  $\Delta y_t = y_t - y_{t-1}$, $\Delta^2 y_t = \Delta y_t - \Delta y_{t-1}$

  \pause 
  ARIMA — \alert{A}uto\alert{R}egressive \alert{I}ntegrated \alert{M}oving \alert{A}verage
  
\end{frame}

\begin{frame}
  \frametitle{Как выбрать?}
  
  $y_t \sim ARMA(p, q)$ или $\Delta y_t \sim ARMA(p, q)$ или $\Delta^2 y_t \sim ARMA(p, q)$
\pause
  \begin{itemize}[<+->]
    \item Посмотреть на график!
    
    \pause График стационарного процесса колеблется в полосе постоянной ширины вокруг своего ожидания.

    \item Оценить все эти модели и выбрать наилучшую по кросс-валидации.
    
    \pause Затратно по времени!

    \item Применять $AIC$ нельзя!
    
    \pause $\ln L(y_1, \ldots, y_n \mid \theta)$ и $\ln L(y_2, \ldots, y_n \mid \theta, y_1)$ и $\ln L(y_3, \ldots, y_n \mid \theta, y_1, y_2)$

    несравнимы!
    \item Есть тесты на единичный корень!
    
    \pause ADF, KPSS, PP, \ldots
  \end{itemize}

\end{frame}

\begin{frame}
  \frametitle{Выбираем «на глазок»}

  четыре небольших графика на одном слайде

  явно стационарный (x)

  случайное блуждание (y)

  явно с трендом (z)

  что-то спорное (w)
  
\end{frame}



\begin{frame}{Буковка I: итоги}

  \begin{itemize}[<+->]
    \item $ARMA$ модель подходит только для \alert{стационарных} рядов. 
    \item Иногда стационарен $\Delta y_t$ или $\Delta^2 y_t$. 
    \item Выбираем между $ARMA$ и $ARIMA$.
  \end{itemize}
\end{frame}



% !TEX root = ../om_ts_06.tex

\begin{frame} % название фрагмента

\videotitle{ADF тест}

\end{frame}



\begin{frame}{ADF тест: план}
  \begin{itemize}[<+->]
    \item ADF тест. 
    \item Три вариации теста.
  \end{itemize}

\end{frame}


\begin{frame}
  \frametitle{Зачем нужен ADF тест?}

  Хотим ответить на вопросы:
  \pause
  \begin{itemize}[<+->]
    \item Использовать $ARMA$ модель для $(y_t)$ или для $(\Delta y_t)$?
    \item Как включать константу в модель?
  \end{itemize}

  \pause
  Название «тест на единичные корни»
  \[
  \Delta = 1 - L = P(L) 
  \]
  Уравнение $1 - \ell = 0$ имеет корень $\ell =1$.

\end{frame}

\begin{frame}
  \frametitle{ADF тест}
  
  \begin{block}{Расшифровка}
    Augmented Dickey Fuller test, расширенный тест Дики-Фуллера  
  \end{block}

  \pause 
  Три вариации теста: без константы, с константой, c трендом.
  
\end{frame}


\begin{frame}
  \frametitle{ADF с константой}

  Предположения:

  \[
  \Delta y_t = c + \beta y_{t-1} + d_1 \Delta y_{t-1} + \ldots + d_k \Delta y_{t-p} + u_t,  
  \]

  \pause

  $H_0$: $\beta = 0$;
  
  $\Delta y_t$ — стационарный $AR(p)$ процесс;

  $y_t = y_0 + ct + \sum_{i=1}^t (\Delta y_t - \E(\Delta y_t))$;

  \pause

  $H_a$: $\beta < 0$;

  $y_t$ — стационарный $AR(p + 1)$ процесс;

\end{frame}

\begin{frame}
  \frametitle{ADF с константой: $H_0$ и $H_a$}


  тут два графика,


  слева $H_0$

  $\Delta y_t$ — стационарный $AR(p)$ процесс;

  $y_t = y_0 + ct + \sum_{i=1}^t (\Delta y_t - \E(\Delta y_t))$;


  справа $H_a$

  $y_t$ — стационарный $AR(p + 1)$ процесс c ненулевым ожиданием

\end{frame}

\begin{frame}
  \frametitle{ADF с константой: алгоритм}

  Шаг 1. Оцениваем \alert{регрессию}
  \[
    \widehat{\Delta y_t} = \hat c + \hat \beta y_{t-1} + \hat d_1 \Delta y_{t-1} + \ldots + \hat d_k \Delta y_{t-p}.  
  \]

  \pause
  Шаг 2. Считаем по \alert{классической формуле} $t$-статистики
  \[
  ADF = \frac{\hat \beta -  0}{se(\hat \beta)}.  
  \]

  \pause
  При верной $H_0$ распределение $ADF$-статистики стремится к \alert{особому распределению} $DF^c$!

  \pause 
  Шаг 3. Делаем вывод:
  
  Если $ADF < DF^c$, то $H_0$ отвергается. 

\end{frame}


\begin{frame}
  \frametitle{ADF без константы}


  Предположения:

  \[
  \Delta y_t = \beta y_{t-1} + d_1 \Delta y_{t-1} + \ldots + d_k \Delta y_{t-p} + u_t,  
  \]

  \pause

  $H_0$: $\beta = 0$;
  
  $\Delta y_t$ — стационарный $AR(p)$ процесс c $\E(\Delta y_t) = 0$;

  $y_t = y_0 + \sum_{i=1}^t \Delta y_t $;

  \pause

  $H_a$: $\beta < 0$;

  $y_t$ — стационарный $AR(p + 1)$ процесс с $\E(y_t) = 0$;

  \pause 

  В алгоритме будет \alert{регрессия без константы} и другое распределение $DF^0$.

\end{frame}


\begin{frame}
  \frametitle{ADF с трендом}


  Предположения:

  \[
  \Delta y_t = c + g t + \beta y_{t-1} + d_1 \Delta y_{t-1} + \ldots + d_k \Delta y_{t-p} + u_t,  
  \]

  \pause

  $H_0$: $\beta = 0$;
  
  $\Delta y_t = k_1 + k_2 t + x_t$;
  
  $x_t$ — стационарный $AR(p)$ процесс c $\E(x_t) = 0$;

  $y_t = y_0 + m_1 t + m_2 t^2 + \sum_{i=1}^t x_t$;

  \pause

  $H_a$: $\beta < 0$;

  $y_t = m_1 + m_2 t + x_t$;
  
  $x_t$ — стационарный $AR(p + 1)$ процесс с $\E(x_t) = 0$;

  \pause 

  В алгоритме будет регрессия \alert{с константой и трендом} и другое распределение $DF^{ct}$.

\end{frame}

\begin{frame}{$ADF$ тест: итоги}

  \begin{itemize}[<+->]
    \item Применим для принятия решения о переходе к $\Delta y_t$.
    \item Есть три варианта теста с разными предпосылками.
  \end{itemize}
\end{frame}




% !TEX root = ../om_ts_06.tex

\begin{frame} % название фрагмента

  \videotitle{KPSS тест}
  
  \end{frame}
  
  
  
  \begin{frame}{KPSS тест: план}
    \begin{itemize}[<+->]
      \item Долгосрочная дисперсия.
      \item Предпосылки теста.
      \item Две вариации теста.
    \end{itemize}
  
  \end{frame}
  
  
  \begin{frame}
    \frametitle{Зачем нужен KPSS тест?}
  
    Хотим ответить на вопросы:
    \pause
    \begin{itemize}[<+->]
      \item Использовать $ARMA$ модель для $(y_t)$ или для $(\Delta y_t)$?
      \item Как включать константу в модель?
    \end{itemize}
    
  \end{frame}
  
  \begin{frame}
    \frametitle{KPSS тест}
    
    \begin{block}{Расшифровка}
      Kwiatkowski–Phillips–Schmidt–Shin test
      
      Тест Квятковского-Филлипса-Шмидта-Шина
    \end{block}
  
    \pause 
    Две вариации теста: с константой, c трендом.
    
  \end{frame}


  \begin{frame}
    \frametitle{Долгосрочная дисперсия}

    \begin{block}{Определение}
      Для стационарного процесса $(y_t)$ величина $\lambda^2$ называется \alert{долгосрочной дисперсией}, если
      \[
        \Var(\bar y) = \frac{\lambda^2}{T} + o(1/T)
      \]
      или 
      \[
        \lim_{T \to \infty} T \Var(\bar y) = \lambda^2, 
      \]
      где $\bar y = (y_1 + \ldots + y_T) / T$.
    \end{block}

    \pause 

    \begin{block}{Мотивация}
      Для независимых наблюдений с одинаковой дисперсией 
      \[
        \Var(\bar y) = \frac{\sigma^2}{T},\text{ где }\sigma^2 = \Var(y_i).
      \]
  \end{block}  
  
  \end{frame}
  
  
  \begin{frame}
    \frametitle{KPSS с константой}
    \[
      y_t = c + rw_t + x_t,
    \]
  
    \pause
  
    \alert{$H_0$: $rw_t = 0$};
    
    $(x_t)$ — стационарный процесс с $\E(x_t) = 0$;
    
    \pause
  
    \alert{$H_a$: $rw_t = rw_{t-1} + u_t$};

    $rw_0 = 0$;
  
    $(x_t)$ — стационарный процесс с $\E(x_t) = 0$;

    $(u_t)$ — белый шум, независимый с $(x_t)$.
  
  \end{frame}
  
  \begin{frame}
    \frametitle{KPSS с константой: $H_0$ и $H_a$}
  
  
    тут два графика,
  
  
    слева $H_0$
    
    справа $H_a$
  
  
  \end{frame}
  
  \begin{frame}
    \frametitle{KPSS с константой: алгоритм}
  
    Шаг 1. Оцениваем \alert{регрессию на константу} 
    \[
      \widehat{y_t} = \hat c.  
    \]
  
    \pause
    Шаг 2. Считаем $KPSS$ статистику
    \[
    KPSS = \frac{\sum_{t=1}^T S_t^2}{T^2 \hat \lambda^2},
    \]
    где $S_t$ — накопленная сумма остатков, $S_t = \hat u_1 + \ldots + \hat u_T$,

    а $\hat\lambda^2$ — состоятельная оценка долгосрочной дисперсии. 
  
    \pause
    При верной $H_0$ распределение $KPSS$-статистики стремится к \alert{особому распределению} $KPSS^c$!
  
    \pause 
    Шаг 3. Делаем вывод:
    
    Если $KPSS > KPSS^c$, то $H_0$ отвергается. 
  
  \end{frame}
  
  
  \begin{frame}
    \frametitle{KPSS с трендом}
    \[
      y_t = c + bt + rw_t + x_t,
    \]
  
    \pause
  
    \alert{$H_0$: $rw_t = 0$};
    
    $(x_t)$ — стационарный процесс с $\E(x_t) = 0$;
    
    \pause
  
    \alert{$H_a$: $rw_t = rw_{t-1} + u_t$};

    $rw_0 = 0$;
  
    $(x_t)$ — стационарный процесс с $\E(x_t) = 0$;

    $(u_t)$ — белый шум, независимый с $(x_t)$.
  
    \pause 
  
    В алгоритме будет регрессия \alert{с константой и трендом} и другое распределение $KPSS^{ct}$.
  
  \end{frame}
  
  
  \begin{frame}
    \frametitle{$KPSS$ с трендом: $H_0$ и $H_a$}
  
  
    тут два графика,
  
  
    слева $H_0$
  
  
    справа $H_a$
  
  
  \end{frame}
  

  \begin{frame}
    \frametitle{Устоявшаяся терминология:}
    \[
      A. y_t = a + bt + x_t;
    \]

    $(y_t)$ — \alert{стационарный вокруг тренда} (trend stationary).

    $(x_t)$ — стационарный процесс с $\E(x_t) = 0$.

    \pause Рецепт: оценим регрессию $a + bt$ с $ARMA$ ошибками для $(y_t)$.
    \pause 
    \[
      B. y_t = a + \sum_{i=1}^t x_i \text{ или } y_t = a + bt + \sum_{i=1}^t x_i
    \]

    $(x_t)$ — стационарный процесс с $\E(x_t) = 0$.

    $(y_t)$ — \alert{стационарный в разностях} (difference stationary).

    \pause Рецепт: оценим $ARMA$ для $(\Delta y_t)$.

    \pause Оба $(y_t)$ нестационарны!
    
  \end{frame}


  
  \begin{frame}{$KPSS$ тест: итоги}
  
    \begin{itemize}[<+->]
      \item Применим для принятия решения о переходе к $\Delta y_t$.
      \item Есть два варианта теста с разными предпосылками.
    \end{itemize}
  \end{frame}
  
  
  
  


%% !TEX root = ../om_ts_04.tex

\begin{frame} % название фрагмента

\videotitle{$MA(\infty)$}

\end{frame}



\begin{frame}{$MA(\infty)$: план}
  \begin{itemize}[<+->]
    \item Определение.
    \item Существование бесконечных сумм. 
    \item Стационарность.
  \end{itemize}
\end{frame}

\begin{frame}
  \frametitle{$MA(\infty)$}

  \begin{block}{Определение}
    Процесс $(y_t)$, который \alert{можно} представить в виде
    \[
    y_t = \mu + u_t + \alpha_1 u_{t-1} + \alpha_2 u_{t-2} + \ldots
    \]
    где $(u_t)$ — белый шум, бесконечное количество $\alpha_i \neq 0$ и 
    $\sum_{i=1}^{\infty} \alpha_i^2 < \infty$, 
    называется $MA(\infty)$ процессом. 
  \end{block}

  \pause 
  $MA(\infty)$:
  \[
  y_t = 5 + u_t + 0.5 u_{t-1} + 0.5^2 u_{t-2} + 0.5^3 u_{t-3} + \ldots
  \]

  \pause
  А \alert{так нельзя}:
  \[
   y_t = 5 + u_t + \frac{1}{\sqrt{2}}u_{t-1} + \frac{1}{\sqrt{3}} u_{t-2} + \frac{1}{\sqrt{4}} u_{t-3} + \ldots
  \]

\end{frame}

\begin{frame}
  \frametitle{Сходимости}
  \begin{block}{Теорема}
    Если 
    $\sum_{i=0}^{\infty} \alpha_i^2 < \infty$ и $(u_t)$ — стационарный процесс с нулевым ожиданием, 
    то последовательность частичных сумм $y^q_t$
    \[
      y^q_t = \sum_{i=0}^q \alpha_i u_{t-i}
    \]
  сходится при $q \to \infty$  \alert{в среднеквадртичном}, \alert{по вероятности} и \alert{по распределению}.
  \end{block}

  Нюанс: сходимость взвешенной суммы гарантирована для стационарного $(u_t)$.  
  \pause
  \begin{block}{Бонус}
    \ldots и получающийся процесс $(y_t)$ стационарен.
  \end{block}
\end{frame}


\begin{frame}
  \frametitle{Виды сходимости: $q \to \infty$}

  \begin{block}{$y^q_t \to y_t$ в среднеквадратичном}
    \[
      \E((y_t - y_t^q)^2) \to 0.
    \]
  \end{block}
\pause
\begin{block}{$y^q_t \to y_t$ по вероятности}
  \[
    \P(\abs{y_t - y_t^q} > \varepsilon) \to 0 \text{ для любого числа } \varepsilon > 0.
  \]
\end{block}

\pause
\begin{block}{$y^q_t \to y_t$ по распределению}
  \[
    \P(y_t^q \leq c)  \to \P(y_t \leq c)
  \]
  в точках непрерывности $F(c) = \P(y_t \leq c)$.
\end{block}

\end{frame}



\begin{frame}
  \frametitle{Теорема Вольда}

  \begin{block}{Теорема}
    Если $(y_t)$ — стационарный процесс, то он представим в виде:
    \[
    y_t = \sum_{i=0}^{\infty} \alpha_i u_{t-i} + r_t,   
    \]
    где 
    \begin{itemize}
      \item $(u_t)$ — белый шум,
      \item $\sum \alpha_i^2 < \infty$,
      \item $r_t$ — линейно \alert{предсказуемый} случайный процесс,
      \item $\Cov(u_t, r_t) = 0$.
    \end{itemize}
  \end{block}

  \pause
  Ахтунг: \alert{deterministic} часто ошибочно переводят как 
  последовательность констант. 

\end{frame}

\begin{frame}
  \frametitle{Предсказуемый процесс}


\begin{block}{Правильное определение}
  Процесс $(r_t)$ называется \alert{линейно предсказуемым}, если
  \begin{itemize}
    \item $(r_t)$ стационарен,
    \item $r_t = \beta_0 + \beta_1 r_{t-1} + \beta_2 r_{t-2} + \ldots + \beta_p r_{t-p}$.
  \end{itemize}

  \end{block}

\end{frame}


\begin{frame}
  \frametitle{$MA(\infty)$: плюсы}

  \begin{itemize}[<+->]
    \item \alert{Стационарный} процесс.
    \item Богатая структура корреляций $\rho_k$.
    \item \alert{Практически} любой стационарный процесс. 
  \end{itemize}

\end{frame}


\begin{frame}
    \frametitle{$MA(\infty)$: проблемы}

    \begin{itemize}[<+->]
        \item Оценить невозможно: \alert{бесконечное} число параметров $\alpha_i$. 
        
        \pause
        Введём \alert{ограничения} на $\alpha_i$!

        \item Да ещё и запись \alert{не единственна}.
        
        \pause
        Договоримся о \alert{канонической} записи!
    \end{itemize}

\end{frame}


\begin{frame}{$MA(\infty)$: итоги}

  \begin{itemize}[<+->]
    \item \alert{Быстро} стремящиеся к нулю коэффициенты. 
    \item \alert{Стационарный процесс}. 
    \item \alert{Пока} не ясно как оценивать.
  \end{itemize}
\end{frame}



%% !TEX root = ../om_ts_02.tex

\begin{frame} % название фрагмента

\videotitle{Сравнение моделей}

\end{frame}



\begin{frame}{Сравнение моделей: план}
  \begin{itemize}[<+->]
    \item MAE и ещё куча страшных слов. 
    \item Кросс-валидация.
    \item Критерий Акаике.
  \end{itemize}

\end{frame}


\begin{frame}
  \frametitle{Помните о цели!}

  Если цель построения модели — прогнозы на один шаг вперёд, 
  то разумно сравнивать модели по прогнозной силе на один шаг вперёд. 

  \pause
  Если цель — обнаружить момент разладки,
  то разумно искать модель дающую минимальную ошибку, когда нет разладки, 
  и максимальную ошибку, когда разладка есть. 

\end{frame}

\begin{frame}
    \frametitle{Обозначения для краткости}

    Для прогноза важно, \alert{когда} его строят, и на \alert{сколько шагов вперёд}:
    \[
    \hat y_{t+h \mid t}.    
    \]

    \pause 
    Иногда для \alert{краткости}:
    \[
    \hat y_{t+h}    
    \]
    \pause 
    Проблемка:
    \[
    \hat y_{(t+1) + 2} \neq \hat y_{(t+2) + 1}.    
    \]    
    
\end{frame}


\begin{frame}
    \frametitle{Показатели антикачества}

    \alert{Ошибка прогноза}: $e_{t+h} = y_{t+h} - \hat y_{t+h}$.

    \pause
    \alert{Средняя абсолютная ошибка} (Mean Absolute Error):
    \[
    MAE = \frac{\abs {e_{T+1}} + \abs{e_{T+1}}+ \ldots + \abs{e_{T+H}} }{H}.
    \]
    \pause
    \alert{Средняя квадратичная ошибка} (Root Mean Squared Error):
    \[
        RMSE = \sqrt{ \frac{e^2_{T+1} + e^2_{T+1}+ \ldots + e^2_{T+H} }{H} }.
    \]
    
\end{frame}


\begin{frame}
    \frametitle{Масштабируем}

    Переводим ошибку $e_{t+h} = y_{t+h} - \hat y_{t+h}$  \alert{в проценты} $p_t= e_t/y_t \cdot 100$ или
    $p^s_t = e_t/(0.5 y_t + 0.5\hat y_t) \cdot 100$.

    \pause
    \alert{Средняя абсолютная процентная ошибка} (Mean Absolute Persentage Error):
    \[
    MAPE = \frac{\abs {p_{T+1}} + \abs{p_{T+1}}+ \ldots + \abs{p_{T+H}} }{H}.
    \]
    \pause 
    \alert{Симметричная средняя абсолютная процентная ошибка} (Symmetric Mean Absolute Persentage Error):
    \[
    sMAPE = \frac{\abs {p^s_{T+1}} + \abs{p^s_{T+1}}+ \ldots + \abs{p^s_{T+H}} }{H}.
    \]
    
\end{frame}

\begin{frame}
    \frametitle{Автоматически сравниваем с наивной}

    \alert{Наивный прогноз}: $\hat y^{naive}_t = y_{t-1}$ или $\hat y^{naive}_t = y_{t-12}$.
    \pause
    Отмасштабируем ошибку нашего прогноза $e_t$ к $MAE^{naive}$:
    \[
    q_t = \frac{e_t}{MAE^{naive}}.
    \]

    \pause
    \alert{Средняя абсолютная отмасштабированная ошибка} (Mean Absolute Scaled Error):
    \[
    MASE  = \frac{\abs {q_{T+1}} + \abs{q_{T+1}}+ \ldots + \abs{q_{T+H}} }{H}.
    \]

    \pause 
    Сравнение $q$ с единицей сравнивает нашу модель с наивной. 


\end{frame}


\begin{frame}
    \frametitle{Обучающая и тестовая выборка}

    Стратегия: 
    \begin{enumerate}[<+->]
        \item Делим всю выборку на \alert{обучающую} (в начале) и \alert{тестовую} (в конце).
        \item \alert{Оцениваем} несколько моделей по обучающей выборке.
        \item \alert{Прогнозируем} каждое наблюдение тестовой выборки с помощью каждой модели.
        \item Рассчитываем \alert{ошибки} прогнозов моделей.  
        \item \alert{Сравниваем} модели по $MAE$ и выбираем лучшую.
    \end{enumerate}

    \pause
    Недостаток: \alert{у прогнозов разный горизонт}.

\end{frame}


\begin{frame}
    \frametitle{Деление на обучающую и тестовую}

    картинка с растущими стрелочками-параболками

\end{frame}


\begin{frame}
    \frametitle{Кросс-валидация скользящим окном}

    Стратегия:
    \begin{enumerate}[<+->]
        \item Выбираем стартовый размер \alert{обучающей} выборки (в начале).
        \item \alert{Оцениваем} несколько моделей по обучающей выборке.
        \item \alert{Прогнозируем} на один шаг вперёд с помощью каждой модели. 
        \item Рассчитываем \alert{ошибки} прогнозов моделей.  
        \item \alert{Сдвигаем} обучающую выборку на одно наблюдение вправо. 
        \item Повторяем шаги 2-5.
        \item \alert{Сравниваем} модели по $MAE$ и выбираем лучшую.
    \end{enumerate}

\end{frame}


\begin{frame}
    \frametitle{Кросс-валидация скользящим окном}

    картинка с растущими стрелочками-параболками

\end{frame}



\begin{frame}
    \frametitle{Кросс-валидация растущим окном}

    Стратегия:
    \begin{enumerate}
        \item Выбираем стартовый размер обучающей выборки (в начале).
        \item Оцениваем несколько моделей по обучающей выборке.
        \item Прогнозируем на один шаг вперёд с помощью каждой модели. 
        \item Рассчитываем ошибки прогнозов моделей.  
        \item \alert{Увеличиваем} обучающую выборку на одно наблюдение. 
        \item Повторяем шаги 2-5.
        \item Сравниваем модели по $MAE$ и выбираем лучшую.
    \end{enumerate}

\end{frame}


\begin{frame}
    \frametitle{Кросс-валидация растущим окном}

    картинка с растущими стрелочками-параболками

\end{frame}


\begin{frame}
    \frametitle{Кросс-валидация: обсуждение}

    Кросс-валидация \alert{скользящим} окном: наблюдений много и мы подозреваем, что 
    зависимость изменяется.
    \pause
    Кросс-валидация \alert{растущим} окном: наблюдений мало или мы уверены 
    в том, что зависимость сохраняется.
    \pause
    Кросс-валидация — может быть долгой!

\end{frame}

\begin{frame}
    \frametitle{Сделаем кросс-валидацию по-быстрому!}

    Примерная замена кросс-валидации на один шаг вперёд по $RMSE$.

    \alert{Критерий Акаике} (Akaike Information Criterion):
    \pause
    \[
      AIC = -2 \ln L + 2k,
    \]
    гда $\ln L$ — логарифм максимума правдоподобия на обучающей выборке, $k$ — общее число параметров модели. 
    
\end{frame}

\begin{frame}
    \frametitle{Нюансы $AIC$}

    \begin{itemize}[<+->]
        \item $AIC$ имеет \alert{теоретические основания}:
        \[
            \frac{AIC_A - AIC_B}{2} \approx KL(\text{Truth} || \text{Model A}) - KL(\text{Truth} || \text{Model B}).
        \]
        \item Может использоваться \alert{для невложенных моделей}. 
        \item Для гауссовских моделей $y_t$ критерий аппроксимирует \alert{сравнение по $RMSE$}.
        \item Сравниваемые модели должны моделировать \alert{те же} наблюдения. 
        \item Разный софт может исключать из правдоподобия \alert{разные константы}. 
    \end{itemize}

    

\end{frame}


\begin{frame}{Сравнение моделей: итоги}

  \begin{itemize}[<+->]
    \item MAE, RMSE, MAPE, MASE.
    \item Кросс-валидация: скользящее и растущее окно.
    \item AIC — быстрый примерный аналог кросс-валидации. 
  \end{itemize}
\end{frame}






\end{document}
