% !TEX root = ../om_ts_05.tex

\begin{frame} % название фрагмента

\videotitle{Происхождение мифа}

\end{frame}



\begin{frame}{Происхождение мифа: план}
  \begin{itemize}[<+->]
    \item Нюансы $MA(\infty)$. 
    \item Шумы бывают разные!
    \item Выводы о структуре решений. 
  \end{itemize}

\end{frame}

\begin{frame}
    \frametitle{Распространённый миф!}
    \[
      y_t =  2y_{t-1} + u_t, \pause  \lambda_1 = 2: \text{ есть стационарное решение! }  
    \]
    Стационарное решение в студию!
    \[
    y_t =  -0.5 u_{t+1} - 0.5^2 u_{t+2} - 0.5^3 u_{t+3} - 0.5^4 u_{t+4} +\ldots 
    \]
    
  
\end{frame}

\begin{frame}
    \frametitle{Это не MA($\mathbf{\infty}$)?}

    \[
        y_t =  -0.5 u_{t+1} - 0.5^2 u_{t+2} - 0.5^3 u_{t+3} - 0.5^4 u_{t+4} +\ldots 
    \]

    \pause 
    \begin{block}{Определение}
        Процесс $(y_t)$, который \alert{можно} представить в виде
        \[
        y_t = \mu + u_t + \alpha_1 u_{t-1} + \alpha_2 u_{t-2} + \ldots,
        \]
        где $(u_t)$ — белый шум, бесконечное количество $\alpha_i \neq 0$ и 
        $\sum_{i=1}^{\infty} \alpha_i^2 < \infty$, 
        называется $MA(\infty)$ процессом. 
    \end{block}
    
\end{frame}

\begin{frame}
    \frametitle{Это MA($\mathbf{\infty}$)!}

    \[
        y_t =  -0.5 u_{t+1} - 0.5^2 u_{t+2} - 0.5^3 u_{t+3} - 0.5^4 u_{t+4} +\ldots 
    \]

    Данный \alert{можно} представить в виде
    \pause 
    \[
      y_t = \nu_t + 0.5 \nu_{t-1} + 0.5^2 \nu_{t-2} +0.5^3 \nu_{t-3} + \ldots, 
    \]
    где $(\nu_t)$ — белый шум.
    \pause 
    \[
        \nu_t = (1 - 0.5L) y_t = (1- 0.5L) (-0.5 u_{t+1} - 0.5^2 u_{t+2} - 0.5^3 u_{t+3} +\ldots )   
    \]

\end{frame}





\begin{frame}
    \frametitle{Происхождение мифа}
  
    \begin{block}{Правильная теорема}
      Если $ARMA$ уравнение $P(L) y_t = c + Q(L) u_t$ несократимо, то оно 
       имеет ровно одно стационарное решение \alert{вида $MA(\infty)$ относительно $(u_t)$}, 
        если и только если $\abs{\ell_i} > 1$ для всех корней лагового полинома $P(L)$.
    \end{block}
  
    \pause
    \begin{block}{Вариация}
      Если $ARMA$ уравнение $P(L) y_t = c + Q(L) u_t$ несократимо, то оно 
       имеет ровно одно стационарное решение \alert{вида $MA(\infty)$ относительно $(u_t)$}, 
        если и только если $\abs{\lambda_i} < 1$ для всех корней характеристического полинома $\phi_{AR}(\lambda)$.
    \end{block}
  \end{frame}
  
  
  
  \begin{frame}
    \frametitle{Разница в шумах!}
    
    Несократимое $ARMA$ уравнение $P(L) y_t = c + Q(L)u_t$.
  \pause
    \begin{itemize}[<+->]
      \item Если у $\phi_{AR}(\lambda)$ есть корень с $\abs{\lambda_i} = 1$,
      \pause
      то стационарных решений нет. 
      \pause
      \item Если у $\phi_{AR}(\lambda)$ все корни с $\abs{\lambda_i} < 1$,
      \pause
      то стационарное решение единственно и имеет вид $MA(\infty)$ относительно $(u_t)$:
      \[
        y_t = \mu + u_t + \alpha_1 u_{t-1} + \alpha_2 u_{t-2} +\ldots 
      \]
      \pause
      \item Если у $\phi_{AR}(\lambda)$ все корни с $\abs{\lambda_i} \neq 1$,
      но есть корень с $\abs{\lambda_i} > 1$,
      \pause
      то стационарное решение единственно и \alert{имеет вид $MA(\infty)$}:
      \[
        y_t = \mu +  \alert{\nu_t + \alpha_1 \nu_{t-1} + \alpha_2 \nu_{t-2} +\ldots}
      \]
  
    \end{itemize}
  
    
  
  \end{frame}
  



\begin{frame}{Происхождение мифа: итоги}

  \begin{itemize}[<+->]
    \item Слово «\alert{можно}» в $MA(\infty)$ важно. 
    \item Разница между $\abs{\lambda_i} < 1$ и $\abs{\lambda_i} > 1$.
  \end{itemize}
\end{frame}



