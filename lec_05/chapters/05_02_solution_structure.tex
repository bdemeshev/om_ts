% !TEX root = ../om_ts_05.tex

\begin{frame} % название фрагмента

\videotitle{Структура решений ARMA уравнения}

\end{frame}



\begin{frame}{Структура решений: план}
  \begin{itemize}[<+->]
    \item Начальные условия. 
    \item Когда есть стационарные решения? 
    \item Разрушаем мифы!
  \end{itemize}

\end{frame}


\begin{frame}
  \frametitle{Начальные условия}
  Несократимое $ARMA$ уравнение:
  \[
  y_t = 0.5 y_{t-1} + u_t, \text{ где } (u_t) \text{  — белый шум.}  
  \]

  \pause
  Пробуем разные начальные условия:
  \pause
  \begin{itemize}
    \item $\alert{y_0=0}$: 
  \end{itemize}
  \pause 
  $y_1 = u_1$, $y_2 = u_2 + 0.5 u_1$, $y_3 = u_3 + 0.5 u_2 + 0.25 u_1$, \ldots
  \pause \pause
  \begin{itemize}
    \item $\alert{y_0=2u_1}$:
  \end{itemize}
  \pause
    $y_1 = 2u_1$, $y_2 = u_2 + u_1$, $y_3 = u_3 + 0.5 u_2 + 0.5 u_1$, \ldots  
  \pause

  Начальные условия определяют и прошлые $y_t$!

\end{frame}

\begin{frame}
  \frametitle{}

  \begin{block}{Теоремка один}
    Любое $ARMA$ уравнение, где есть хотя бы один лаг $y_t$, имеет бесконечное количество решений. 
  \end{block}
  
  \pause
  \begin{block}{Теоремка два}
    Для того, чтобы получить единственное решение $ARMA$ уравнения вида $P(L)y_t = c + Q(L) u_t$, 
    достаточно задать начальные условия в количестве равном степени $P(L)$.
  \end{block}
  \pause
  \[
  y_t = 0.6 y_{t-1} + 0.08y_{t-2} + u_t \pause \text{ и } \alert{y_0 = u_0, y_1 = u_0 + 4}
  \]

\end{frame}

\begin{frame}
  \frametitle{А сколько стационарных решений?}

  \begin{block}{Правильная теорема}
    Если $ARMA$ уравнение $P(L) y_t = c + Q(L) u_t$ несократимо, то оно 
    \begin{itemize}
      \item имеет ровно одно стационарное решение, если у лагового полинома $P(\ell)$ у всех корней $\abs{\ell_i} \neq 1$;
      \item не имеет стационарных решений, если у лагового полинома $P(\ell)$ есть корень с $\abs{\ell_i} = 1$.
    \end{itemize}    
  \end{block}

  \pause
  \begin{itemize}
    \item $y_t = 0.5 y_{t-1} + u_t, P(L) = 1 - 0.5L, \ell_1 = 2$: \pause одно стационарное решение;
  \end{itemize}
  \pause  
  \begin{itemize}
    \item $y_t = y_{t-1} + u_t, P(L) = 1 - L, \ell_1 = 1$:  нет стационарных решений. 
  \end{itemize}
    
\end{frame}



\begin{frame}
  \frametitle{Вариация теоремы}

  \begin{block}{Правильная теорема}
    Если $ARMA$ уравнение несократимо, то оно 
    \begin{itemize}
      \item имеет ровно одно стационарное решение, если у характеристического полинома $\phi_{AR}(\lambda)$ у всех корней $\abs{\lambda_i} \neq 1$;
      \item не имеет стационарных решений, еслиу характеристического полинома $\phi_{AR}(\lambda)$ есть корень с $\abs{\lambda_i} = 1$.
    \end{itemize}    
  \end{block}



  \pause
  \begin{itemize}
    \item $y_t = 0.5 y_{t-1} + u_t, \phi_{AR}(\lambda) = \lambda - 0.5, \lambda_1 = 0.5$: \pause одно стационарное решение;
  \end{itemize}
  \pause  
  \begin{itemize}
    \item $y_t = y_{t-1} + u_t, \phi_{AR}(\lambda) = \lambda - 1, \lambda_1 = 1$: нет стационарных решений. 
  \end{itemize}

\end{frame}




\begin{frame}
  \frametitle{Распространённый миф!}
  \[
    y_t =  2y_{t-1} + u_t, \pause  \lambda_1 = 2: \text{ нет стационарных решений? }  
  \]
  
  \pause

  Стационарное решение в студию!
  \[
  y_t =  -0.5 u_{t+1} - 0.5^2 u_{t+2} - 0.5^3 u_{t+3} - 0.5^4 u_{t+4} +\ldots 
  \]
  

\end{frame}



\begin{frame}{Структура решений: итоги}

  \begin{itemize}[<+->]
    \item \alert{Начальные условия} дают единственность решения.
    \item У несократимого уравнения стационарное решение либо \alert{единственно}, либо не существует.
    \item Разрушаем \alert{миф} про $\abs{\lambda_i} > 1$.
  \end{itemize}
\end{frame}



