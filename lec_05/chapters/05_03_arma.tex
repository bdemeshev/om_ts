% !TEX root = ../om_ts_05.tex

\begin{frame} % название фрагмента

\videotitle{ARMA процесс}

\end{frame}



\begin{frame}{ARMA процесс: план}
  \begin{itemize}[<+->]
    \item Определение $AR$ и $ARMA$ процесса. 
    \item Обратимость для единственности записи. 
    \item Свойства.
  \end{itemize}

\end{frame}

\begin{frame}
  \frametitle{AR процесс}

  \begin{block}{Определение}
    $AR(p)$ процессом с уравнением 
    \[
      y_t = c + \beta_1 y_{t-1} + \ldots + \beta_p y_{t-p} + u_t, 
    \]
    где $(u_t)$ — белый шум и $\beta_p \neq 0$ называется 
    решение этого уравнения вида $MA(\infty)$ относительно $(u_t)$.
  \end{block}

  \pause  
  \begin{block}{Определение с лагами}
    $AR(p)$ процессом с уравнением 
    \[
      P(L)y_t = c + u_t, 
    \]
    где $(u_t)$ — белый шум, $P(L)$ имеет степень $p$ и $P(0)=1$, называется 
    решение этого уравнения вида $MA(\infty)$ относительно $(u_t)$.  
  \end{block}

\end{frame}

\begin{frame}
  \frametitle{Определения у разных авторов}

  \pause 
  В нашем определении $AR(p)$ процесс обязательно \alert{стационарен}. 
  
  \pause
  Некоторые авторы \alert{не включают} в определение $AR(p)$ процесса требование стационарности. 

\end{frame}

\begin{frame}
  \frametitle{ARMA процесс}

  \begin{block}{Определение}
    $ARMA(p, q)$ процессом с уравнением 
    \[
      y_t = c + \beta_1 y_{t-1} + \ldots + \beta_p y_{t-p} + u_t + \alpha_1 u_{t-1} + \ldots + \alpha_q u_{t-q}, 
    \]
    где $(u_t)$ — белый шум, $\beta_p \neq 0$ и $\alpha_q \neq 0$, называется 
    решение этого уравнения вида $MA(\infty)$ относительно $(u_t)$.
  \end{block}

  \pause  
  \begin{block}{Определение с лагами}
    $ARMA(p,q)$ процессом с уравнением 
    \[
      P(L)y_t = c + Q(L)u_t, 
    \]
    где $(u_t)$ — белый шум, $P(L)$ имеет степень $p$, $Q(L)$ имеет степень $q$, и $P(0)=Q(0)=1$, называется 
    решение этого уравнения вида $MA(\infty)$ относительно $(u_t)$.  
  \end{block}
\end{frame}


\begin{frame}
  \frametitle{ARMA: нюансы}

  \pause
  \begin{itemize}
    \item $AR(1)$ процесс с уравнением $y_t = 0.5 y_{t-1} + u_t$:
  \end{itemize}
  \pause
  \[
  y_t = u_t + 0.5 u_{t-1} + 0.5^2 u_{t-2} + \ldots   \pause
  \]
  \begin{itemize}
    \item $AR(1)$ процесса с уравнением $y_t = y_{t-1} + u_t$ не существует:
  \end{itemize}
  \pause
  $\phi_{AR}(\lambda) = \lambda - 1$, $\lambda_1 = 1$, нет стационарных решений. 

  \pause
  \begin{itemize}
    \item $AR(1)$ процесса с уравнением $y_t = 2 y_{t-1} + u_t$ не существует:
  \end{itemize}
  \pause
  $\phi_{AR}(\lambda) = \lambda - 2$, $\lambda_1 = 2$, есть стационарное решение 
  вида $MA(\infty)$, но не относительно $(u_t)$.
  
\end{frame}

\begin{frame}
  \frametitle{А что с неединственностью записи?}

  Один и тот же $ARMA(p, q)$ процесс $(y_t)$ может описываться разными уравнениями!
  \pause

  \begin{block}{Спасительная обратимость}
      Если ряд $(y_t)$ является $ARMA(p, q)$ процессом с уравнением $P(L) y_t = c + Q(L) u_t$, 
      то данное уравнение будет единственным, если на $MA$ часть выполнено условие обратимости. 
  \end{block}

  \pause
  Условие обратимости $ARMA$ уравнения:
  \begin{itemize}
    \item у характеристического многочлена $\phi_{MA}(\lambda)$ все корни $\abs{\lambda_i} <1$;
    \item у лагового многочлена $Q(L)$ все корни $\abs{\ell_i} >1$.
  \end{itemize}


\end{frame}



\begin{frame}
  \frametitle{ARMA процесс: итоги}
  \begin{itemize}[<+->]
    \item Процесс стационарен по \alert{определению};
    \item $AR$ и $MA$ процессы являются \alert{частными случаями} $ARMA$ процесса;
    \item \alert{Малым числом параметров} описываются разнообразные $MA(\infty)$ процессы. 
    \item Можно \alert{приблизить} любую структуру $ACF$ и $PACF$.
    \item Теоретические $ACF$ и $PACF$ \alert{экспоненциально} убывают.
   \item Условие обратимости гарантирует \alert{единственность записи}. 
  \end{itemize}

\end{frame}



