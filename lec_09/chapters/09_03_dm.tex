% !TEX root = ../om_ts_09.tex

\begin{frame} % название фрагмента

\videotitle{Сравнение двух прогнозов}

\end{frame}



\begin{frame}{Сравнение прогнозов: план}
  \begin{itemize}[<+->]
    \item Тест Диболда-Мариано.
    \item \alert{Предпосылки} теста.
    \item Реализация теста. 
  \end{itemize}

\end{frame}

\begin{frame}
  \frametitle{Тест Диболда-Мариано}

  \begin{itemize}[<+->]
    \item Предназначен для сравнения \alert{двух} прогнозов.
    \item Сравнивает прогнозы на \alert{заданный горизонт} прогнозирования $h$.
    \item Не является оптимальным для \alert{сравнения моделей}. 
    \item Не подходит для \alert{попарного} сравнения множества прогнозов.
  \end{itemize}
  
\end{frame}

\begin{frame}
  \frametitle{Предпосылки DM-теста}

  Рассмотрим \alert{разницу потерь} двух прогнозов:
  \[
  d_t = e_{A,t}^2 - e_{B,t}^2, \quad e_{\text{Model},t} = \hat y_{\text{Model},t} - y_t;
  \]
  \pause
  Разница $d_t$ предполагается \alert{стационарной}:\pause
  \[
  \E(d_t) = \mu_d,
  \]
  \pause 
  \[
  \Cov(d_t, d_{t-k}) = \gamma_k,    
  \] \pause
  в частности,
  \[
  \Var(d_t) = \gamma_0.    
  \]
  
\end{frame}

\begin{frame}
    \frametitle{Способ тестирования}
    При верной $H_0: \mu_d = 0$:
    \[
       DM = \frac{\bar d}{se(\bar d)} \to \cN(0;1),
    \]
    где $se^2(\bar d)$ — состоятельная оценка для $\Var(\bar d)$.
    \pause 

    На практике оценивают регрессию на константу
    \[
    \hat d_t = \hat \beta_1,
    \]
    \pause 
    получают $\hat\beta_1 = \bar d$ и используют готовые \alert{робастные стандартные ошибки},
    \[
        DM = \frac{\hat \beta_1}{se_{HAC}(\hat\beta_1)}.
    \]
\end{frame}

\begin{frame}
  \frametitle{Как устроена робастная оценка?}
  
  Сравниваем прогнозы по $P$ точкам, 
  \[
  \Var(\bar d) = \frac{(\Var(d_1) + \Var(d_2) + \ldots + 2\Cov(d_1, d_2) + \ldots)}{P^2}
  \]\pause
  Из стационарности $d_t$:
  \[
    \Var(\bar d) = \frac{P\gamma_0 + 2(P-1) \gamma_1 + 2(P-2)\gamma_2 +\ldots}{P^2} 
  \]\pause 
  Наивная оценка:
  \[
    \widehat\Var(\bar d) = \frac{ P\hat\gamma_0 + 2(P-1) \hat\gamma_1 + 2(P-2)\hat\gamma_2 +\ldots}{P^2}
  \]
\end{frame}



\begin{frame}
    \frametitle{Почему сравнение прогнозов?}

    \alert{Нюанс}: сравнение прогнозов и сравнение моделей — разные задачи. 
    \pause

    Модель может сильно выигрывать \alert{по простоте} и немного проигрывать по прогнозам.
    \pause 

    На малой выборке \alert{потеря информации} о качестве прогнозов на обучающей выборке существенна. 
    \pause 

    На практике часто \alert{говорят «моделей»}.
\end{frame}


\begin{frame}{Сравнение двух прогнозов: итоги}

  \begin{itemize}[<+->]
    \item Тест Диболда-Мариано подходит для сравнения \alert{двух} прогнозов. 
    \item Сравнение прогнозов и сравнение моделей — \alert{немного разные} задачи.  
  \end{itemize}
\end{frame}

