% !TEX root = ../om_ts_09.tex

\begin{frame} % название фрагмента

\videotitle{Сравнение прогнозов}

\end{frame}



\begin{frame}{Сравнение прогнозов: план}
  \begin{itemize}[<+->]
    \item Тест Диболда-Мариано.
    \item RC-тест Уайта.
    \item SPA-тест Хансена. 
  \end{itemize}

\end{frame}

\begin{frame}
  \frametitle{Тест Диболда-Мариано}

  \begin{itemize}[<+->]
    \item Предназначен для сравнения \alert{двух} прогнозов.
    \item Сравнивает прогнозы на \alert{заданный горизонт} прогнозирования $h$.
    \item Не является оптимальным для \alert{сравнения моделей}. 
    \item Не подходит для \alert{попарного} сравнения множества прогнозов.
  \end{itemize}
  
\end{frame}

\begin{frame}
  \frametitle{Предпосылки DM-теста}

  Рассмотрим \alert{разницу потерь} двух прогнозов:
  \[
  d_t = e_{A,t}^2 - e_{B,t}^2, \quad e_{\text{Model},t} = \hat y_{\text{Model},t} - y_t;
  \]
  \pause
  Разница $d_t$ предполагается \alert{стационарной}:\pause
  \[
  \E(d_t) = \mu_d,
  \]
  \pause 
  \[
  \Cov(d_t, d_{t-k}) = \gamma_k,    
  \] \pause
  В частности,
  \[
  \Var(d_t) = \gamma_0.    
  \]
  
\end{frame}

\begin{frame}
    \frametitle{Способ тестирования}
    При верной $H_0: \mu_d = 0$:
    \[
       DM = \frac{\bar d}{se(\bar d)} \to \cN(0;1),
    \]
    где $se^2(\bar d)$ — состоятельная оценка для $\Var(\bar d)$.
    \pause 
    На практике оценивают регрессию на константу
    \[
    \hat d_t = \hat \beta_1
    \]
    \pause 
    Получают $\hat\beta_1 = \bar d$ и используют \alert{робастные стандартные ошибки},
    \[
        DM = \frac{\hat \beta_1}{se_{HAC}(\hat\beta_1)}.
    \]

\end{frame}


\begin{frame}
    \frametitle{RC-тест Уайта}

    RC = Reality Check, проверка реальностью. 
    \pause 

    \begin{itemize}
        \item Предназначен для сравнения \alert{множества} прогнозов с эталонным. \pause
        \item Сравнивает прогнозы на \alert{заданный горизонт} прогнозирования $h$. \pause
        \item Не является оптимальным для \alert{сравнения моделей}. \pause
      \end{itemize}

      $H_0$: ни один из прогнозов не обыгрывает эталонный прогноз. 

      $H_a$: хотя бы один из прогнозов лучше. 
\end{frame}

\begin{frame}
    \frametitle{Обозначения}
    
    \begin{itemize}
        \item $e_{jt} = \hat y_{jt} - y_t$ — ошибки прогноза модели $j$; \pause
        \item $d_{jt} = e_{jt}^2 - e_{\text{bench}, t}^2$ — разница потерь по сравнению с эталонным прогнозом. \pause
        \item $\bar d_j$ — средняя разница потерь модели $j$. \pause
        \item Статистика теста $RC = \max_j \bar d_j$.  
    \end{itemize}    
    \pause 
    Гипотезы:
    $H_0$: $\max_j \E(d_jt) \leq 0$. 

    $H_a$: $\max_j \E(d_jt) > 0$.


\end{frame}

\begin{frame}
    \frametitle{Запускаем бутстрэп}

    \begin{enumerate}
        \item По 
    \end{enumerate}
    

\end{frame}


\begin{frame}
    \frametitle{Вариации}

    Есть \alert{много} вариаций этого подхода. 


    

\end{frame}



\begin{frame}
    \frametitle{Почему сравнение прогнозов?}

    \alert{Тонкий нюанс}: сравнение прогнозов и сравнение моделей — разные задачи. 
    \pause
    Модель может сильно выигрывать \alert{по простоте} и немного проигрывать по прогнозам.
    \pause 
    На малой выборке \alert{потеря информации} о качестве прогнозов на обучающей выборке существенна. 
\end{frame}


\begin{frame}{Сравнение прогнозов: итоги}

  \begin{itemize}[<+->]
    \item Тест Диболда-Мариано подходит для сравнения \alert{двух} прогнозов. 
    \item SPA-тест и RC-тест подходят для сравнения \alert{множества} прогнозов. 
    \item Лучше использовать \alert{SPA-тест} Хансена. 
    \item Иногда названия \alert{SPA} и \alert{RC} путают. 
    \item SPA-тест используют, например, для сравнения \alert{торговых стратегий}. 
  \end{itemize}
\end{frame}

