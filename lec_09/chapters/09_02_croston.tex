% !TEX root = ../om_ts_09.tex

\begin{frame} % название фрагмента

\videotitle{Данные прерывающиеся нулями}

\end{frame}



\begin{frame}{Данные прерывающиеся нулями: план}
  \begin{itemize}[<+->]
    \item Нули в данных.
    \item Алгоритм Кростона. 
  \end{itemize}

\end{frame}

\begin{frame}
  \frametitle{Откуда нули в данных?}

  Счётные данные с \alert{небольшим} ожиданием: 
  \begin{itemize}[<+->]
    \item Ежедневное количество пожаров в небольшое городе.
    \item Еженедельное количество завершенных писателем романов. 
    \item \ldots
  \end{itemize}
  
\end{frame}

\begin{frame}
  \frametitle{Как моделировать?}

  \begin{itemize}[<+->]
    \item Специальные модели для счётных данных. 
    
    Используют распределение \alert{Пуассона}, \alert{отрицательное биномиальное}, \ldots

    \item Простой алгоритм Кростона. 
    
    Подходит для \alert{несезонных} данных, основан на \alert{экспоненциальном сглаживании}. 
  \end{itemize}

  

\end{frame}


\begin{frame}
  \frametitle{Напоминание про ETS(ANN)}

  Уравнения модели:
  \[
  \begin{cases}
  y_t = \ell_{t-1} + u_t  \\
  \ell_t = \ell_{t-1} + \alpha u_t \\
  \end{cases}
  \]
  \pause
  \[
  \ell_t = \alpha y_t + (1 - \alpha) \ell_{t-1}
  \]
  \pause 
  Прогноз на 1 шаг вперёд:
  \[
  \hat y_{t+1} = \alpha y_t + (1-\alpha )\hat y_{t}.
  \]
\end{frame}


\begin{frame}{Алгоритм Кростона}
  
  Шаг 1. Разобъём исходный ряд $(y_t)$
  \[
   3, 0, 2, 0, 0, 4, 0, 0, 0, 3, 0, 1, \ldots
  \] \pause
  на ряд \alert{положительных} значений $(q_t)$
  \[
  3, 2, 4, 3, 1, \ldots  
  \]
  и \alert{длины нулевых промежутков} $(a_t)$:
  \[
  1, 2, 3, 1, \ldots   
  \]
  \pause 

  Шаг 2. Применим простое экспоненциальное сглаживание.
  \[
\begin{cases}
  \hat q_{t+1} = \alpha_q q_t + (1-\alpha_q )\hat q_{t} \\
  \hat a_{t+1} = \alpha_a a_t + (1-\alpha_a )\hat a_{t}
\end{cases}
  \]
  \pause 
  Параметры: $\alpha_a$, $\alpha_q$, $\hat a_{0}$, $\hat q_{0}$.


\end{frame}
  

\begin{frame}
  \frametitle{Прогнозирование}

  Из алгоритма Кростона можно извлечь:
  \begin{itemize}[<+->]
    \item $\hat q_{T+1}$ — прогноз следующего ненулевого числа. 
    \item $\hat a_{T+1}$ — прогноз длины нулевого промежутка. 
    \item $\hat y_{T+1} = \hat q_{T+1} / \hat a_{T+1}$ — прогноз для исходного ряда. 
  \end{itemize}
  
\end{frame}

\begin{frame}
  \frametitle{Сравнение прогнозов}

  Используйте \alert{MASE}:
    \[
      MASE  = \frac{\abs {q_{T+1}} + \abs{q_{T+2}}+ \ldots + \abs{q_{T+H}} }{H},
  \]  
  где 
  \begin{itemize}
    \item $e_t$ — ошибка прогноза;
    \item $q_t = \frac{e_t}{MAE^{naive}}$ — ошибка прогноза,
    отмашстабированная на среднюю абсолютную ошибку наивного прогноза. 
  \end{itemize}
\end{frame}


\begin{frame}{Данные прерывающиеся нулями: итоги}

  \begin{itemize}[<+->]
    \item Как правило, много нулей в \alert{счётных} данных.
    \item Алгоритм Кростона подойдёт для \alert{несезонных} данных. 
    \item Алгоритм Кростона \alert{нестатический}: нет прогнозных интервалов. 
  \end{itemize}
\end{frame}

