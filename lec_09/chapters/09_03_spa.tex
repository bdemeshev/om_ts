% !TEX root = ../om_ts_09.tex

\begin{frame} % название фрагмента

\videotitle{Сравнение множества прогнозов}

\end{frame}



\begin{frame}{Сравнение множества прогнозов: план}
  \begin{itemize}[<+->]
    \item RC-теста Уайта.
    \item Стационарный \alert{бутстрэп}.
    \item \alert{Уточнение} в виде SPA-теста.
  \end{itemize}

\end{frame}


\begin{frame}
    \frametitle{RC-тест Уайта и SPA-тест Хансена}

    RC = Reality Check, проверка реальностью;

    SPA = Superior Predictive Ability, превосходящее качество прогнозов.
    \pause 

    \begin{itemize}
        \item Два похожих теста, предназначенных для сравнения \alert{множества} прогнозов с эталонным. \pause
        \item Сравнивают прогнозы на \alert{заданный горизонт} прогнозирования $h$. \pause
        \item Не являются оптимальным для \alert{сравнения моделей}. \pause
        \item SPA-тест — более \alert{робастная} вариация RC-теста. \pause
        \item Предполагают \alert{оптимальные} параметры всех алгоритмов.
    \end{itemize}

\end{frame}

\begin{frame}
    \frametitle{Обозначения}
    
    \begin{itemize}
        \item $e_{jt} = \hat y_{jt} - y_t$ — ошибки прогноза алгоритма $j$; \pause
        \item Превосходство алгоритмов над эталонным в момент $t$:
        \[
        d_t = \begin{pmatrix}
          e_{\text{bench}, t}^2 - e_{At}^2 \\
          e_{\text{bench}, t}^2 - e_{Bt}^2 \\          
          e_{\text{bench}, t}^2 - e_{Ct}^2 \\
          \ldots
        \end{pmatrix}  
        \] \pause
        \item $\bar d = \sum d_t / P$ —  среднее превосходство алгоритмов, \\
        $P$ — число наблюдений, по которым идёт сравнение.  
    \end{itemize}    
    \pause 
    Гипотезы RC-теста:
    
    $H_0$: $\E(d_t) = 0$. 

    $H_a$: $\max(\E(d_t)) > 0$.
\end{frame}




\begin{frame}
    \frametitle{Реализация RC-теста}

    \begin{enumerate}[<+->]
        \item Находим значение наилучшего среднего превосходства $RC = \max (\bar d)$.  
        \item Генерируем \alert{бутстрэп-копию} траектории $d_t$: 
        \[
        d_1, d_2, \ldots, d_P  \quad \to \quad     d_1^*, d_2^*, \ldots, d_P^*.
        \]
        \item По бустрэп-копии находим для каждого алгоритма $j$
        \[
            \Delta_j = \bar d^*_j - \bar d_j.
        \]
        \item Находим $RC^* = \max(\Delta)$.
        \item Повторяем (2-4) 10000 раз, получаем $RC^*_1$, \ldots, $RC^*_{10000}$.
        \item Считаем P-значение как долю $RC^*$, оказавшихся больше фактического $RC$.
    \end{enumerate}
\end{frame}

\begin{frame}
    \frametitle{Бутстрэп-подделка}

    Генерируем \alert{бутстрэп-копию} траектории $d_t$: 
        \[
        d_1, d_2, \ldots, d_P  \quad \to \quad     d_1^*, d_2^*, \ldots, d_P^*.
        \] \pause
    \begin{itemize}
        \item Бутстрэп-подделка имеет длину исходного ряда.\pause 
        \item Состоит из случайных, возможно накладывающихся, \alert{фрагментов} ряда. \pause 
        \item Длина каждого фрагмента имеет \alert{геометрическое} распределение. 
    \end{itemize}
\end{frame}




\begin{frame}
    \frametitle{Уточнения SPA-теста}


    \alert{Стьюдентизация} (нормировка) среднего превосходства каждого алгоритма. 
        \[
        \bar d = \begin{pmatrix}
            \bar d_A \\
            \bar d_B \\
            \bar d_C \\
            \ldots 
        \end{pmatrix}     \to 
        d_{stud} = \begin{pmatrix}
            \bar d_A / se(\bar d_A) \\
            \bar d_B / se(\bar d_B) \\
            \bar d_C / se(\bar d_C) \\
            \ldots 
        \end{pmatrix}        
        \] \pause
        \[
        SPA = \max(d_{stud})    
        \]
\end{frame}


\begin{frame}
    \frametitle{Уточнения SPA-теста}

    \alert{Предварительное деление} алгоритмов на «хорошие» и «плохие». \pause

    По бустрэп-копии находим для каждого алгоритма $j$
    \[
        \Delta_j = \begin{cases}
            \bar d^*_j / se(\bar d_j) \text{ для плохого алгоритма } j\\
            (\bar d^*_j - \bar d_j)/ se(\bar d_j) \text{ для хорошего алгоритма }j.
        \end{cases}
    \]
\end{frame}



\begin{frame}{Сравнение множества прогнозов: итоги}

  \begin{itemize}[<+->]
    \item SPA-тест и RC-тест подходят для сравнения \alert{множества} прогнозов. 
    \item \alert{SPA-тест} Хансена имеет большую мощность. 
    \item Иногда названия \alert{SPA} и \alert{RC} путают. 
    \item SPA-тест используют, например, для сравнения \alert{торговых стратегий}. 
    \item Сравнение прогнозов и сравнение моделей — \alert{разные} задачи. 
  \end{itemize}
\end{frame}

