% !TEX root = ../om_ts_06.tex

\begin{frame} % название фрагмента

\videotitle{Алгоритм Хандакара-Хиндмана}

\end{frame}

\begin{frame}{Алгоритм Хандакара-Хиндмана: план}
  \begin{itemize}[<+->]
    \item Три шага алгоритма. 
    \item Нюансы и рекомендации.
  \end{itemize}

\end{frame}


\begin{frame}
  \frametitle{Как всё это собрать в кучу?}

  \onslide<2->{Шаг 1 (для сезонных рядов). Сколько раз надо брать $\Delta_{12}$?}

  \onslide<3->{Шаг 2. Сколько раз надо брать $\Delta$?}

  \onslide<4->{Шаг 3. Какую стационарную $SARMA$ модель оценивать после взятия разностей?}

\end{frame}

\begin{frame}
  \frametitle{Шаг 1. Сколько раз надо брать $\Delta_{12}$?}

  Ни разу, раз или два раза.
  \pause
  \begin{itemize}[<+->]
    \item Находим $STL$ разложение ряда. 
  
    \item Если сила сезонности меньше пороговой, то работаем с исходным рядом $y_t$. 

    \item Если сила больше пороговой, то переходим к сезонной разности и 
    после нового $STL$ разложения сравниваем силу сезонности с пороговой ещё раз.

    \item Если сила сезонности снова больше пороговой, то работаем с $\Delta^2_{12} y_t$,
    иначе работаем с $\Delta_{12}y_t$.
  \end{itemize}
  \pause 
  Есть альтернатива в виде теста Канова-Хансена (Canova-Hansen).

\end{frame}

\begin{frame}
  \frametitle{Шаг 2. Сколько раз надо брать $\Delta$?}

  Ни разу, раз или два раза.
  \pause
  \begin{itemize}[<+->]
    \item Применяем $KPSS$ тест с константой к исходному ряду. 
  
    \item Если $H_0$ не отвергается, то работаем с рядом $y_t$. 

    \item Если $H_0$ отвергается, то проводим $KPSS$ тест для разности $\Delta y_t$.

    \item Если у повторного $KPSS$ теста $H_0$ отвергается, то работаем с $\Delta^2 y_t$
    иначе работаем с $\Delta y_t$.
  \end{itemize}
  \pause 
  Есть альтернатива в виде $ADF$ теста.

\end{frame}

\begin{frame}
  \frametitle{Шаг 3. Выбор $SARMA$ модели для преобразованного ряда}

  \begin{itemize}[<+->]
    \item Оцениваем большое количество экономных $SARMA$ моделей.
    
    $\Delta^d \Delta^D_{12} y_t \sim SARMA(p, q)(P, Q) [12]$, $p + q \leq 5$, $P + Q \leq 5$
    \item Выбираем наилучшую модель по штрафному критерию Акаике:
    
    $AIC = 2 K - 2 \ln L$, где $K$ — общее число параметров,  
    
    $\ln L$ — логарифм правдоподобия.
  \end{itemize}

  \pause
  Есть альтернатива в виде перебора с помощью кросс-валидации. 
  

\end{frame}


\begin{frame}
  \frametitle{Методология Бокса-Дженкинсона}

  \begin{itemize}[<+->]
    \item Идентификация подходящей модели. 
    
    Графический анализ, тесты. Выбор количества сезонных и обычных разностей.
    \item Оценивание подходящей модели.
    
    Оценивание параметров $SARMA$ модели для преобразованного ряда. 

    \item Статистическая проверка модели. 
    
    Визуализация остатков. Тесты на остатки модели.
  \end{itemize}
  
  \pause  
  Алгоритм Хандакара-Хиндмана — практическая реализация методологии.

\end{frame}


\begin{frame}
  \frametitle{Нюансы алгоритма}

  \begin{itemize}

  \onslide<1->{\item Очень \alert{много опций}\ldots}

  \onslide<2->{Возможны отличия реализаций в софте.}

  \onslide<3->{\item Обратите внимание на включение \alert{константы}:}

  \onslide<4->{$P(L) y_t = c + Q(L) u_t$ или $P(L) (y_t - \mu) = Q(L) u_t$}

  \onslide<5->{\item Требует \alert{много времени}.}

  \onslide<6->{Не стоит использовать кросс-валидацию.}

  \onslide<7->{\item Суммирует опыт \alert{десятилетий}.}

  \onslide<8->{Не забудьте им воспользоваться!}
  \end{itemize}
\end{frame}




\begin{frame}{Алгоритм Хандакара-Хиндмана: итоги}
  \begin{itemize}[<+->]
    \item Шаг 1. Решение о переходе к сезонным разностям.
    \item Шаг 2. Решение о переходе к разностям.
    \item Шаг 3. Оценка множества $SARMA$ моделей с выбором по $AIC$.
    \item Обязательно попробуйте алгоритм!
  \end{itemize}

\end{frame}
