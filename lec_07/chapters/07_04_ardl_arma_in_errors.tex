% !TEX root = ../om_ts_06.tex

\begin{frame} % название фрагмента

\videotitle{Предикторы и $ARIMA$}

\end{frame}

\begin{frame}{Предикторы и $ARIMA$: план}
  \begin{itemize}[<+->]
    \item Регрессия с \alert{$ARMA$ ошибками}.
    \item \alert{$ARMAX$ модель}.
    \item \alert{$ARDL$ модель}.
  \end{itemize}
\end{frame}

\begin{frame}
  \frametitle{Регрессия с $ARMA$ ошибками}
  
  \begin{block}{Уравнение}
    \[
      y_t = \beta_1 + \beta_2 a_t + \beta_3 b_t + \varepsilon_t,
    \]
  где $a_t$ и $b_t$ — \alert{предикторы}.
  \end{block}
  \pause
  \begin{itemize}
  \item Ряды $(y_t)$, $(a_t)$, $(b_t)$, $(\varepsilon_t)$ \alert{стационарны}.
  \item $\varepsilon_t \sim ARMA(p, q)$ относительно белого шума $(u_t)$.
\item $\E(\varepsilon_t \mid a_{t-1}, b_{t-1}, a_{t-2}, b_{t-2}, \ldots) = 0$.
\item \alert{Четвертые моменты} предикторов конечны.  
\end{itemize}
  
\end{frame}


\begin{frame}
  \frametitle{$ARMAX$ модель}

  \begin{block}{Уравнение}
    \[
      y_t = c + \gamma_1 a_t + \gamma_2 b_t + \beta_1 y_{t-1} + \ldots + \beta_p y_{t-p} + u_t + \alpha_1 u_{t-1} + \ldots + \alpha_q u_{t-q},
    \]
  где $a_t$ и $b_t$ — \alert{предикторы}, а $(u_t)$ — \alert{белый шум}.
  \end{block}
\pause
  \begin{itemize}[<+->]
  \item Ряды $(y_t)$, $(a_t)$, $(b_t)$ \alert{стационарны}.
\item $\E(u_t \mid a_{t-1}, b_{t-1}, y_{t-1}, a_{t-2}, b_{t-2}, y_{t-2}, \ldots) = 0$.
\item \alert{Четвертые моменты предикторов} конечны.  
\end{itemize}
  \pause
  $ARMAX$ модель не полностью эквивалентна регрессии с $ARMA$ ошибками, но качество прогнозов у моделей \alert{примерно одинаковое}.  

\end{frame}


\begin{frame}
  \frametitle{Свойства $ARMAX$ модели и регрессии с $ARMA$ ошибками}

  \begin{itemize}[<+->]
    \item Если выполнены предпосылки, то оценки метода максимального правдоподобия \alert{состоятельны}.
    \item Для нестационарных переменных $(y_t)$ и предикторов $(a_t)$ и $(b_t)$ можно перейти к первым \alert{разностям}. 
    \item Оценки \alert{останутся} состоятельны, если добавить тренд, дамми на сезонность и тригонометрические предикторы.
    \item \alert{Не любой} предиктор даёт возможность получить состоятельную оценку коэффициента. 
    \item \alert{Иногда} можно получить хорошие прогнозы, даже если предпосылки нарушены.
  \end{itemize}
  
\end{frame}



\begin{frame}
  \frametitle{$ARDL$ модель}
  
  \alert{ARDL} — \alert{A}uto\alert{R}egressive \alert{D}istributed \alert{L}ag model

  Авторегрессионная модель с распределёнными лагами.

  \begin{block}{Уравнение $ARDL(p, q)$ модели}
    \[
      y_t = c + \beta_1 y_{t-1} + \ldots + \beta_p y_{t-p} + x_t + \alpha_1 x_{t-1} + \ldots + \alpha_q x_{t-q} + u_t
    \]
  \end{block}
\pause
  \begin{itemize}[<+->]
    \item Ошибки $(u_t)$ являются \alert{белым шумом}. 
    \item Процесс $(x_t)$ \alert{или} процесс $(\Delta x_t)$ является стационарным. 
    \item Процесс $(y_t)$ является \alert{нестационарным}, но $(\Delta y_t)$ является стационарным. 
    \item $\E(u_t \mid y_{t-1}, x_{t-1}, y_{t-2}, x_{t-2}, \ldots) = 0$.
  \end{itemize}

\end{frame}

\begin{frame}
  \frametitle{Свойства $ARDL$ модели}
  \begin{itemize}[<+->]
    \item Вместо лагов шума $(u_t)$ \alert{лаги предиктора} $(x_t)$.
    \item Подходит для \alert{нестационарного} $(y_t)$.
    \item Используют для нахождения \alert{долгосрочных взаимоотношений} между рядами.
    \item Если предпосылки выполнены, то \alert{оценки МНК состоятельны}, хотя и смещены. 
    \item Можно добавить \alert{несколько предикторов} с разным числом лагов. 
  \end{itemize}

\end{frame}


\begin{frame}{Предикторы и $ARIMA$: итоги}
  \begin{itemize}[<+->]
    \item Для \alert{стационарных данных} можно использовать регрессию с $ARMA$ ошибками или $ARMAX$ модель. 
    \item Регрессию с $ARMA$ ошибками можно строить \alert{в разностях}.
    \item Для \alert{нестационарных рядов} иногда можно использовать $ARDL$ модель. 
  \end{itemize}

\end{frame}
