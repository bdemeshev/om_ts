% !TEX root = ../om_ts_07.tex

\begin{frame} % название фрагмента

\videotitle{У нас есть ещё время!}

% https://www.youtube.com/watch?v=7CCBsshm-cQ

\end{frame}



\begin{frame}{У нас ещё есть время: план}
  \begin{itemize}[<+->]
    \item Предикторы \alert{тренда}.
    \item \alert{Сезонные} и \alert{праздничные} дамми.
    \item \alert{Косинусы} и \alert{синусы}.
  \end{itemize}

\end{frame}

\begin{frame}
  \frametitle{Используем время!}

  Для примера возьмём $t$ и $\sqrt{t}$. 
  \pause

  \alert{Обучающая} выборка:
  \[
  \begin{pmatrix}
    y_1 \\
    y_2 \\
    y_3 \\
    \vdots \\
    y_T 
  \end{pmatrix}  \quad 
  \begin{pmatrix}
    1 & \sqrt{1} \\
    2 & \sqrt{2} \\
    3 & \sqrt{3} \\
    \vdots & \vdots \\
    T & \sqrt{T} \\ 
  \end{pmatrix}
  \]
  \pause
  Выборка для \alert{прогнозирования}:
  \[
  \begin{pmatrix}
    ? 
  \end{pmatrix}  \quad 
  \begin{pmatrix}
    T+1 & \sqrt{T+1} \\ 
  \end{pmatrix}
  \]
  
\end{frame}


\begin{frame}
  \frametitle{Включать ли монотонные преобразования времени?}

  \begin{itemize}[<+->]
    \item Всегда \alert{можно попробовать} включить!
    \item Алгоритмам основанным на построении \alert{деревьев} (случайные лес, градиентный бустинг)
    дополнительные монотонные преобразования времени \alert{бесполезны}. 
    \item Помните о возможном преобразовании \alert{исходной переменной} (логарифм, преобразование Бокса-Кокса).
  \end{itemize}
\end{frame}

\begin{frame}
  \frametitle{Сезонные и праздничные дамми}

  Если сезонов \alert{немного}, то разумно включить дамми на каждый сезон.

  \pause
  \alert{Обучающая} выборка для квартальных данных:
  \[
  \begin{pmatrix}
    y_1 \\
    y_2 \\
    y_3 \\
    y_4 \\
    y_5 \\
    y_6 \\
    \vdots \\
    y_T 
  \end{pmatrix}  \quad 
  \begin{pmatrix}
    1 & 0 & 0 & 0 \\
    0 & 1 & 0 & 0 \\
    0 & 0 & 1 & 0 \\
    0 & 0 & 0 & 1 \\
    1 & 0 & 0 & 0 \\
    0 & 1 & 0 & 0 \\
    \vdots & \vdots &  \vdots & \vdots \\
    0 & 0 & 1 & 0 \\
  \end{pmatrix}
  \]
\end{frame}


\begin{frame}
  \frametitle{Ловушка дамми-переменных}

  
  В \alert{регрессии} помните о \alert{ловушке} дамми-переменных! \pause 
  
  \begin{itemize}[<+->]
    \item Либо дамми на каждый сезон и модель без константы.
    \item Либо дамми на все сезоны кроме одного и модель с константой. 
  \end{itemize}
  
  \pause 
  Алгоритмы основанные на построении \alert{деревьев} (случайные лес, градиентный бустинг)
 \alert{устойчивы} к ловушке дамми. 
\end{frame}


\begin{frame}
  \frametitle{Зачем нужны синусы и косинусы?}

  Стратегия добавления всех дамми переменных \alert{плохо} работает, если их нужно \alert{много}. \pause 
  
  Вряд ли стоит добавлять 365 дамми-переменных для \alert{дневных} данных. \pause

  Обойтись \alert{малым числом} предикторов помогут синус и косинус!

  \pause 
  Два факта:
  \begin{itemize}[<+->]
    \item Период у $\sin t$ и $\cos t$ равен $2\pi$;
    \item При умножении аргумента на $a$ период \alert{сокращается} в $a$ раз.
  \end{itemize}

\end{frame}


\begin{frame}
  \frametitle{Разложение Фурье}

  \begin{block}{Теорема}
    Любая непрерывная и дифференциируемая функция $f$ с периодом $2\pi$ может быть представлена в виде 
        \[
        f(t) = c + \sum_{k=1}^{\infty} a_k \cos(kt) + b_k \sin (kt).   
        \]
  \end{block}

  \pause 
  Практический рецепт для дневных данных:
  \begin{itemize}[<+->]
    \item Добавьте предикторы $\cos\left(\frac{2\pi}{365} \cdot t\right)$ и $\sin\left(\frac{2\pi}{365}  \cdot t\right)$;
    \item Добавьте предикторы $\cos\left(\frac{2\pi}{365}  \cdot 2t\right)$ и $\sin\left(\frac{2\pi}{365} \cdot  2t\right)$;
    \item Добавьте предикторы $\cos\left(\frac{2\pi}{365}  \cdot 3t\right)$ и $\sin\left(\frac{2\pi}{365}  \cdot 3t\right)$;
    \item \ldots 
  \end{itemize}


  

\end{frame}

\begin{frame}{У нас есть ещё время: итоги}

  \begin{itemize}[<+->]
    \item Используйте \alert{время} в качестве предиктора.
    \item Сезонность в предикторах можно отразить с помощью \alert{дамми-переменных} или 
    с помощью \alert{косинуса} и \alert{синуса}.
  \end{itemize}
\end{frame}



