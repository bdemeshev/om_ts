% arara: xelatex
%% arara: xelatex


% https://koalatea.io/r-knn-regression/
% http://freerangestats.info/blog/2017/04/09/propensity-v-regression
% https://economics.stackexchange.com/questions/45335/what-is-the-difference-between-ate-and-att
% https://kosukeimai.github.io/MatchIt/articles/matching-methods.html


\documentclass[14pt,xcolor=dvipsnames]{beamer}


% !TEX root = om_metrics_14.tex

%\usepackage{epsdice} % dice 1-6 for probability :)

% \usepackage[absolute,overlay]{textpos}

% \usefonttheme[onlymath]{serif}

\usefonttheme{professionalfonts}
% by default beamer changes math fonts for better visibility for projection
% this professionalfonst theme removes this behavior


\usepackage[orientation=portrait,size=custom,width=25.4,height=19.05]{beamerposter}




%25,4 см 19,05 см размеры слайда в powerpoint

\usetheme{metropolis}
\metroset{
  %progressbar=none,
  numbering=none,
  subsectionpage=progressbar,
  block=fill
}

%\usecolortheme{seahorse}

\usepackage{xunicode} % хак для акцентов!
% https://tex.stackexchange.com/questions/28003/

\usepackage{fontspec}
\usepackage{polyglossia}
\setmainlanguage{russian}


% \usepackage{fontawesome5} % removed [fixed]
\setmainfont[Ligatures=TeX]{Myriad Pro}
% \setsansfont{Myriad Pro}




% why do we need \newfontfamily:
% http://tex.stackexchange.com/questions/91507/
\newfontfamily{\cyrillicfonttt}{Myriad Pro}
\newfontfamily{\cyrillicfont}{Myriad Pro}
%\newfontfamily{\cyrillicfontbs}{Myriad Pro}
\newfontfamily{\cyrillicfontsf}{Myriad Pro}


% https://tex.stackexchange.com/questions/175860/why-does-unicode-math-break-the-kerning-of-accents-in-combination-with-amssymb
% "You shouldn't be using amssymb together with unicode-math"
\usepackage{amsmath}
\usepackage{amsthm} % amssymb 


% https://tex.stackexchange.com/questions/483722/
% \usepackage[MnSymbol]{mathspec}  % Includes amsmath.
% \usepackage{mathspec}  % Includes amsmath.
% \setmathsfont(Digits,Latin,Greek,Symbols)[Numbers={Lining,Proportional}]{Latin Modern Math}
% mathspec must be loaded earlier than amsmath



%\usepackage{bm}

% \usepackage{fdsymbol} % \nperp

% \usepackage{unicode-math} % \symbf
% \setmathfont{Latin Modern Math}



\usepackage{centernot}

\usepackage{graphicx}

\usepackage{wrapfig}
% \usepackage{animate} % animations :)
% \usepackage{tikz}
%\usetikzlibrary{shapes.geometric,patterns,positioning,matrix,calc,arrows,shapes,fit,decorations,decorations.pathmorphing}
% \usepackage{pifont}
\usepackage{comment}
\usepackage[font=small,labelfont=bf]{caption}
\captionsetup[figure]{labelformat=empty}
% \includecomment{techno}



%Расположение

\setbeamersize{text margin left=15 mm,text margin right=5mm} 
\setlength{\leftmargini}{38 pt}

%\usepackage{showframe}
%\usepackage{enumitem}
% \setlist{leftmargin=5.5mm}


%Цвета от дирекции

\definecolor{dirblack}{RGB}{58, 58, 58}
\definecolor{dirwhite}{RGB}{245, 245, 245}
\definecolor{dirred}{RGB}{149, 55, 53}
\definecolor{dirblue}{RGB}{0, 90, 171}
\definecolor{dirorange}{RGB}{235, 143, 76}
\definecolor{dirlightblue}{RGB}{75, 172, 198}
\definecolor{dirgreen}{RGB}{155, 187, 89}
\definecolor{dircomment}{RGB}{128, 100, 162}

\setbeamercolor{title separator}{bg=dirlightblue!50, fg=dirblue}

%Цвета блоков

% Голубой блок!
\setbeamercolor{block title}{bg=dirblue!30,fg=dirblack}
\setbeamercolor{block title example}{bg=dirlightblue!50,fg=dirblack}
\setbeamercolor{block body example}{bg=dirlightblue!20,fg=dirblack}

\AtBeginEnvironment{exampleblock}{\setbeamercolor{itemize item}{fg=dirblack}}
%\setbeamertemplate{blocks}[rounded][shadow]

% Набор команд для удобства верстки

% Набор команд для структуризации

%\newcommand{\quest}{\faQuestionCircleO}
%\faPencilSquareO \faPuzzlePiece \faQuestionCircleO  \faIcon*[regular]{file} {\textcolor{dirblue}
%\newcommand{\quest}{\textcolor{dirblue}{\boxed{\textbf{?}}}
%\newcommand{\task}{\faIcon{tasks}}
%\newcommand{\exmpl}{\faPuzzlePiece}
%\newcommand{\dfn}{\faIcon{pen-square}}
%\newcommand{\quest}{\textcolor{dirblue}{\faQuestionCircle[regular]}}
%\newcommand{\acc}[1]{\textcolor{dirred}{#1}}
%\newcommand{\accm}[1]{\textcolor{dirred}{#1}}
%\newcommand{\acct}[1]{\textcolor{dirblue}{#1}}
%\newcommand{\acctm}[1]{\textcolor{dirblue}{#1}}
%\newcommand{\accex}[1]{\textcolor{dirblack}{\bf #1}}
%\newcommand{\accexm}[1]{\textcolor{dirblack}{ \mathbf{#1}}}
%\newcommand{\acclp}[1]{\textcolor{dirorange}{\it #1}}
\newcommand{\todo}[1]{\textcolor{dircomment}{\bf #1}}
%\newcommand{\graylink}[1]{{\fontsize{11}{12}\selectfont \textcolor{gray}{#1}}}
%\newcommand{\figcaption}[1]{{\fontsize{18}{20}\selectfont #1}}


\newcommand{\videotitle}[1]{
    {\fontsize{33}{30}\selectfont \textcolor{dirblue}{\textbf{#1}} }

    %\todo{название видеофрагмента}
}

\newcommand{\lecturetitle}[1]{
  {\fontsize{33}{30}\selectfont \textcolor{dirblue}{\textbf{#1}} }

    %\todo{название лекции}
}





%\newcommand{\spcbig}{\vspace{-10 pt}}
%\newcommand{\spcsmall}{\vspace{-5 pt}}

%\usepackage{listings}
%\lstset{
%xleftmargin=0 pt,
%  basicstyle=\small, 
%  language=Python,
  %tabsize = 2,
%  backgroundcolor=\color{mc!20!white}
%}



%\newcommand{\mypart}[1]{\begin{frame}[standout]{\huge #1}\end{frame}}

\setbeamercolor{background canvas}{bg=}

% frame title setup
\setbeamercolor{frametitle}{bg=,fg=dirblue}
\setbeamertemplate{frametitle}[default][left]

\addtobeamertemplate{frametitle}{\hspace*{0.1 cm}}{\vspace*{0.25cm}}


%Шрифты
\setbeamerfont{frametitle}{family=\rmfamily,series=\bfseries,size={\fontsize{33}{30}}}
\setbeamerfont{framesubtitle}{family=\rmfamily,series=\bfseries,size={\fontsize{26}{20}}}


% удобнее знать номер слайда, чтобы вносить правки!  

\setbeamercolor{footline}{fg=dircomment}
\setbeamerfont{footline}{series=\bfseries, size={\fontsize{12}{14}}}
%\setbeamertemplate{footline}[page number]


\defbeamertemplate{footline}{custom footline}
{%
  \hspace*{\fill}%
  \usebeamercolor[fg]{page number in head/foot}%
  \usebeamerfont{page number in head/foot}%
  page: \insertpagenumber\,/\,\insertpresentationendpage%
  \hspace{20pt}%
  slide: \insertframenumber\,/\,\inserttotalframenumber%
  %\hspace*{\fill}
  \vskip2pt%
}
%\setbeamertemplate{footline}[custom footline]

\usepackage{physics}
\usepackage[makeroom]{cancel}



% tikz block

\usepackage{pgfplots}
\pgfplotsset{compat=newest}

\usepackage{tikz}
\usetikzlibrary{calc}
\usetikzlibrary{quotes,angles}
\usetikzlibrary{arrows}
\usetikzlibrary{arrows.meta}
\usetikzlibrary{positioning,intersections,decorations.markings}
\usetikzlibrary{patterns}

\usepackage{tkz-euclide} 
%\tikzset{>=latex}

\tikzset{cross/.style={cross out, draw=black, minimum size=2*(#1-\pgflinewidth), inner sep=0pt, outer sep=0pt},
%default radius will be 1pt. 
cross/.default={5pt}}

\colorlet{veca}{red}
\colorlet{vecb}{blue}
\colorlet{vecc}{olive}


\newcommand{\grid}{\draw[color=gray,step=1.0,dotted] (-2.1,-2.1) grid (9.6,6.1)}

% end tikz block

\newcommand{\R}{\mathbb{R}}
\newcommand{\Rot}{\mathrm{R}}
\newcommand{\HH}{\mathrm{H}}
\newcommand{\Id}{\mathrm{I}}
\newcommand{\RR}{\mathbb{R}}
\newcommand{\ZZ}{\mathbb{Z}}
\newcommand{\la}{\lambda}
\let\P\relax
\newcommand{\P}{\mathbb{P}}
\newcommand{\E}{\mathbb{E}}

\newcommand{\cN}{\mathcal{N}}
\newcommand{\dN}{\mathcal{N}}

\newcommand{\qL}{q_{\text{left}}}
\newcommand{\qR}{q_{\text{right}}}



\newcommand{\ba}{\mathbf{a}}
\newcommand{\be}{\mathbf{e}}
\newcommand{\bb}{\mathbf{b}}
\newcommand{\bc}{\mathbf{c}}
\newcommand{\bd}{\mathbf{d}}
\newcommand{\bx}{\mathbf{x}}
\newcommand{\bff}{\mathbf{f}} % \bf is already def
\newcommand{\bv}{\mathbf{v}}
\newcommand{\bzero}{\mathbf{0}}



\DeclareMathOperator{\Var}{Var}
\DeclareMathOperator{\sVar}{sVar}
\DeclareMathOperator{\Cov}{Cov}
\DeclareMathOperator{\sCov}{sCov}
\DeclareMathOperator{\sCorr}{sCorr}
\DeclareMathOperator{\pCorr}{pCorr}
\DeclareMathOperator{\Corr}{Corr}
\DeclareMathOperator{\Med}{Med}
\let\L\relax
\DeclareMathOperator{\L}{L}


\DeclareMathOperator{\plim}{plim}
\DeclareMathOperator{\sign}{sign}


\newcommand{\graylink}[1]{{\fontsize{11}{12}\selectfont \textcolor{gray}{#1}}}
\newcommand{\figcaption}[1]{{\fontsize{18}{20}\selectfont #1}}





\begin{document}


\begin{frame} % название лекции


\lecturetitle{Добавляем предикторы}

\end{frame}


% !TEX root = ../om_ts_07.tex

\begin{frame} % название фрагмента

\videotitle{Как обойтись без моделей?}

\end{frame}



\begin{frame}{Обойтись без моделей: план}
  \begin{itemize}[<+->]
    \item Как переделать временные ряды в перекрестные данные? 
    \item Добавить \alert{лаги} переменной $y_t$.
    \item Использовать \alert{агрегирующие функции} и \alert{cкользящее} или \alert{растущее} окно. 
  \end{itemize}

\end{frame}


\begin{frame}
  \frametitle{Как обойтись без моделей?}

  \begin{block}{Старые друзья}
    Есть алгоритмы, которые по обучающей выборке зависимой переменной $y$, 
    обучающей матрице предикторов $X$, и новым предикторам $X_F$ строят прогноз $\hat y_F$.      
  \end{block}

  \pause

  \alert{Случайный лес}, \alert{градиентный бустинг}\ldots{ }\pause и даже \alert{обычная регрессия}!

  \pause 

  Можно \alert{усреднять} прогнозы ARIMA/ETS и прогнозы других алгоритмов.

\end{frame}


\begin{frame}
  \frametitle{Как создать предикторы?}

  Из одного столбца $y$ можно создать целую матрицу $X$ предикторов!

  \begin{itemize}[<+->]
    \item Использовать \alert{лаги} $y_{t-k}$. 
    \item Использовать \alert{функции от лагов} в качестве предикторов. 
  \end{itemize}


\end{frame}

\begin{frame}
  \frametitle{Используем лаги $y$}

  Для примера возьмём два лага, $Ly_t$ и $L^2 y_t$.
  \pause

  \alert{Обучающая} выборка:
  \[
  \begin{pmatrix}
    y_3 \\
    y_4 \\
    y_5 \\
    \vdots \\
    y_T 
  \end{pmatrix}  \quad 
  \begin{pmatrix}
    y_1 & y_2 \\
    y_2 & y_3 \\
    y_3 & y_4 \\
    \vdots & \vdots \\
    y_{T-2} & y_{T-1} \\ 
  \end{pmatrix}
  \]
  \pause
  Выборка для \alert{прогнозирования}:
  \[
  \begin{pmatrix}
    ? 
  \end{pmatrix}  \quad 
  \begin{pmatrix}
    y_{T-1} & y_{T} \\ 
  \end{pmatrix}
  \]
  
\end{frame}

\begin{frame}
  \frametitle{Сколько лагов добавить?}

  \begin{itemize}[<+->]
    \item Каждый добавленный лаг \alert{сокращает} обучающую выборку!
    \item Разумно добавить \alert{ближайшие лаги} $Ly_t$, $L^2y_t$.
    \item Для сезонных данных разумно добавить \alert{сезонный лаг} $L^{12} y_t$.
    \item Есть алгоритмы \alert{чувствительные к лишним предикторам}: например, регрессия. 
    \item Есть алгоритмы \alert{нечувствительные к лишним предикторам}: например, случайный лес.
  \end{itemize}
\end{frame}

\begin{frame}
  \frametitle{Функции от лагов}
  При прогнозировании $y_{t}$ \alert{честно} использовать любую функцию от \alert{предыдущих} $y_{t-1}$, $y_{t-2}$, \ldots

  \pause

  Например:
  \begin{itemize}[<+->]
    \item $\Delta y_{t-1} = y_{t-1} - y_{t-2}$;
    \item $\max\{ y_{t-1}, y_{t-2}, y_{t-3} \}$;
    \item $\min\{ y_{t-1}, y_{t-2}, \ldots, y_1\}$.
  \end{itemize}
\end{frame}

\begin{frame}
  \frametitle{Типичный предиктор}

  \begin{itemize}[<+->]
    \item \alert{Агрегирующая функция}:
    
    Минимум, максимум, среднее, медиана, размах, выборочная дисперсия, выборочное стандартное отклонение, \ldots

    \item \alert{Аргумент} агрегирующей функции:
    
    \alert{Скользящее окно}: агрегирующая функция применяется, скажем, к трём предыдущим значениям $y_{t-1}$, $y_{t-2}$, $y_{t-3}$.

    \alert{Растущее окно}: агрегирующая функция применяется ко всем предыдущим значениям $y_{t-1}$, $y_{t-2}$, \ldots,  $y_{1}$.
  \end{itemize}

\end{frame}


\begin{frame}
  \frametitle{Используем функции лагов $y$}

  Для примера возьмём максимум скользящим окном и минимум растущим окном. 
  \pause

  \alert{Обучающая} выборка:
  \[
  \begin{pmatrix}
    y_3 \\
    y_4 \\
    y_5 \\
    \vdots \\
    y_T 
  \end{pmatrix}  \quad 
  \begin{pmatrix}
    \max\{y_1, y_2\} & \min\{y_1, y_2\} \\
    \max\{y_2, y_3\} & \min\{y_1, y_2, y_3\} \\
    \max\{y_3, y_4\} & \min\{y_1, \ldots, y_4\} \\
    \vdots & \vdots \\
    \max\{y_{T-2}, y_{T-1}\} & \min\{y_1, \ldots, y_{T-1}\} \\
  \end{pmatrix}
  \]
  \pause
  Выборка для \alert{прогнозирования}:
  \[
  \begin{pmatrix}
    ? 
  \end{pmatrix}  \quad 
  \begin{pmatrix}
    \max\{y_{T-1}, y_{T}\} & \min\{y_1, \ldots, y_{T}\} \\
  \end{pmatrix}
  \]
  
\end{frame}



\begin{frame}{Обойтись без моделей: итоги}

  \begin{itemize}[<+->]
    \item Помните о \alert{случайном лесе}, \alert{градиентном бустинге} и даже об \alert{обычной регрессии}.
    \item Добавьте \alert{лаги зависимой переменной}.
    \item Добавьте \alert{агрегирующие функции} скользящим и растущим окном. 
  \end{itemize}
\end{frame}


% !TEX root = ../om_ts_07.tex

\begin{frame} % название фрагмента

\videotitle{У нас есть ещё время!}

% https://www.youtube.com/watch?v=7CCBsshm-cQ

\end{frame}



\begin{frame}{У нас ещё есть время: план}
  \begin{itemize}[<+->]
    \item Предикторы \alert{тренда}.
    \item \alert{Сезонные} и \alert{праздничные} дамми.
    \item \alert{Косинусы} и \alert{синусы}.
  \end{itemize}

\end{frame}

\begin{frame}
  \frametitle{Используем время!}

  Для примера возьмём $t$ и $\sqrt{t}$. 
  \pause

  \alert{Обучающая} выборка:
  \[
  \begin{pmatrix}
    y_1 \\
    y_2 \\
    y_3 \\
    \vdots \\
    y_T 
  \end{pmatrix}  \quad 
  \begin{pmatrix}
    1 & \sqrt{1} \\
    2 & \sqrt{2} \\
    3 & \sqrt{3} \\
    \vdots & \vdots \\
    T & \sqrt{T} \\ 
  \end{pmatrix}
  \]
  \pause
  Выборка для \alert{прогнозирования}:
  \[
  \begin{pmatrix}
    ? 
  \end{pmatrix}  \quad 
  \begin{pmatrix}
    T+1 & \sqrt{T+1} \\ 
  \end{pmatrix}
  \]
  
\end{frame}


\begin{frame}
  \frametitle{Включать ли монотонные преобразования времени?}

  \begin{itemize}[<+->]
    \item Всегда \alert{можно попробовать} включить!
    \item Алгоритмам основанным на построении \alert{деревьев} (случайные лес, градиентный бустинг)
    дополнительные монотонные преобразования времени \alert{бесполезны}. 
    \item Помните о возможном преобразовании \alert{исходной переменной} (логарифм, преобразование Бокса-Кокса).
  \end{itemize}
\end{frame}

\begin{frame}
  \frametitle{Сезонные и праздничные дамми}

  Если сезонов \alert{немного}, то разумно включить дамми на каждый сезон.

  \pause
  \alert{Обучающая} выборка для квартальных данных:
  \[
  \begin{pmatrix}
    y_1 \\
    y_2 \\
    y_3 \\
    y_4 \\
    y_5 \\
    y_6 \\
    \vdots \\
    y_T 
  \end{pmatrix}  \quad 
  \begin{pmatrix}
    1 & 0 & 0 & 0 \\
    0 & 1 & 0 & 0 \\
    0 & 0 & 1 & 0 \\
    0 & 0 & 0 & 1 \\
    1 & 0 & 0 & 0 \\
    0 & 1 & 0 & 0 \\
    \vdots & \vdots &  \vdots & \vdots \\
    0 & 0 & 1 & 0 \\
  \end{pmatrix}
  \]
\end{frame}


\begin{frame}
  \frametitle{Ловушка дамми-переменных}

  
  В \alert{регрессии} помните о \alert{ловушке} дамми-переменных! \pause 
  
  \begin{itemize}[<+->]
    \item Либо дамми на каждый сезон и модель без константы.
    \item Либо дамми на все сезоны кроме одного и модель с константой. 
  \end{itemize}
  
  \pause 
  Алгоритмы основанные на построении \alert{деревьев} (случайные лес, градиентный бустинг)
 \alert{устойчивы} к ловушке дамми. 
\end{frame}


\begin{frame}
  \frametitle{Зачем нужны синусы и косинусы?}

  Стратегия добавления всех дамми переменных \alert{плохо} работает, если их нужно \alert{много}. \pause 
  
  Вряд ли стоит добавлять 365 дамми-переменных для \alert{дневных} данных. \pause

  Обойтись \alert{малым числом} предикторов помогут синус и косинус!

  \pause 
  Два факта:
  \begin{itemize}[<+->]
    \item Период у $\sin t$ и $\cos t$ равен $2\pi$;
    \item При умножении аргумента на $a$ период \alert{сокращается} в $a$ раз.
  \end{itemize}

\end{frame}


\begin{frame}
  \frametitle{Разложение Фурье}

  \begin{block}{Теорема}
    Любая непрерывная и дифференциируемая функция $f$ с периодом $2\pi$ может быть представлена в виде 
        \[
        f(t) = c + \sum_{k=1}^{\infty} a_k \cos(kt) + b_k \sin (kt).   
        \]
  \end{block}

  \pause 
  Практический рецепт для дневных данных:
  \begin{itemize}[<+->]
    \item Добавьте предикторы $\cos\left(\frac{2\pi}{365} \cdot t\right)$ и $\sin\left(\frac{2\pi}{365}  \cdot t\right)$;
    \item Добавьте предикторы $\cos\left(\frac{2\pi}{365}  \cdot 2t\right)$ и $\sin\left(\frac{2\pi}{365} \cdot  2t\right)$;
    \item Добавьте предикторы $\cos\left(\frac{2\pi}{365}  \cdot 3t\right)$ и $\sin\left(\frac{2\pi}{365}  \cdot 3t\right)$;
    \item \ldots 
  \end{itemize}


  

\end{frame}

\begin{frame}{У нас есть ещё время: итоги}

  \begin{itemize}[<+->]
    \item Используйте \alert{время} в качестве предиктора.
    \item Сезонность в предикторах можно отразить с помощью \alert{дамми-переменных} или 
    с помощью \alert{косинуса} и \alert{синуса}.
  \end{itemize}
\end{frame}







\end{document}
