% !TEX root = ../om_ts_04.tex

\begin{frame} % название фрагмента

\videotitle{$MA(\infty)$}

\end{frame}



\begin{frame}{$MA(\infty)$: план}
  \begin{itemize}[<+->]
    \item Определение.
    \item Существование бесконечных сумм. 
    \item Стационарность.
  \end{itemize}
\end{frame}

\begin{frame}
  \frametitle{$MA(\infty)$}

  \begin{block}{Определение}
    Процесс $(y_t)$, который \alert{можно} представить в виде
    \[
    y_t = \mu + u_t + \alpha_1 u_{t-1} + \alpha_2 u_{t-2} + \ldots
    \]
    где $(u_t)$ — белый шум, бесконечное количество $\alpha_i \neq 0$ и 
    $\sum_{i=1}^{\infty} \alpha_i^2 < \infty$, 
    называется $MA(\infty)$ процессом. 
  \end{block}

  \pause 
  $MA(\infty)$:
  \[
  y_t = 5 + u_t + 0.5 u_{t-1} + 0.5^2 u_{t-2} + 0.5^3 u_{t-3} + \ldots
  \]

  \pause
  А \alert{так нельзя}:
  \[
   y_t = 5 + u_t + \frac{1}{\sqrt{2}}u_{t-1} + \frac{1}{\sqrt{3}} u_{t-2} + \frac{1}{\sqrt{4}} u_{t-3} + \ldots
  \]

\end{frame}

\begin{frame}
  \frametitle{Сходимости}
  \begin{block}{Теорема}
    Если 
    $\sum_{i=0}^{\infty} \alpha_i^2 < \infty$ и $(u_t)$ — стационарный процесс с нулевым ожиданием, 
    то последовательность частичных сумм $y^q_t$
    \[
      y^q_t = \sum_{i=0}^q \alpha_i u_{t-i}
    \]
  сходится при $q \to \infty$  \alert{в среднеквадртичном}, \alert{по вероятности} и \alert{по распределению}.
  \end{block}

  Нюанс: сходимость взвешенной суммы гарантирована для стационарного $(u_t)$.  
  \pause
  \begin{block}{Бонус}
    \ldots и получающийся процесс $(y_t)$ стационарен.
  \end{block}
\end{frame}


\begin{frame}
  \frametitle{Виды сходимости: $q \to \infty$}

  \begin{block}{$y^q_t \to y_t$ в среднеквадратичном}
    \[
      \E((y_t - y_t^q)^2) \to 0.
    \]
  \end{block}
\pause
\begin{block}{$y^q_t \to y_t$ по вероятности}
  \[
    \P(\abs{y_t - y_t^q} > \varepsilon) \to 0 \text{ для любого числа } \varepsilon > 0.
  \]
\end{block}

\pause
\begin{block}{$y^q_t \to y_t$ по распределению}
  \[
    \P(y_t^q \leq c)  \to \P(y_t \leq c)
  \]
  в точках непрерывности $F(c) = \P(y_t \leq c)$.
\end{block}

\end{frame}



\begin{frame}
  \frametitle{Теорема Вольда}

  \begin{block}{Теорема}
    Если $(y_t)$ — стационарный процесс, то он представим в виде:
    \[
    y_t = \sum_{i=0}^{\infty} \alpha_i u_{t-i} + r_t,   
    \]
    где 
    \begin{itemize}
      \item $(u_t)$ — белый шум,
      \item $\sum \alpha_i^2 < \infty$,
      \item $r_t$ — линейно \alert{предсказуемый} случайный процесс,
      \item $\Cov(u_t, r_t) = 0$.
    \end{itemize}
  \end{block}

  \pause
  Ахтунг: \alert{deterministic} часто ошибочно переводят как 
  последовательность констант. 

\end{frame}

\begin{frame}
  \frametitle{Предсказуемый процесс}


\begin{block}{Правильное определение}
  Процесс $(r_t)$ называется \alert{линейно предсказуемым}, если
  \begin{itemize}
    \item $(r_t)$ стационарен,
    \item $r_t = \beta_0 + \beta_1 r_{t-1} + \beta_2 r_{t-2} + \ldots + \beta_p r_{t-p}$.
  \end{itemize}

  \end{block}

\end{frame}


\begin{frame}
  \frametitle{$MA(\infty)$: плюсы}

  \begin{itemize}[<+->]
    \item \alert{Стационарный} процесс.
    \item Богатая структура корреляций $\rho_k$.
    \item \alert{Практически} любой стационарный процесс. 
  \end{itemize}

\end{frame}


\begin{frame}
    \frametitle{$MA(\infty)$: проблемы}

    \begin{itemize}[<+->]
        \item Оценить невозможно: \alert{бесконечное} число параметров $\alpha_i$. 
        
        \pause
        Введём \alert{ограничения} на $\alpha_i$!

        \item Да ещё и запись \alert{не единственна}.
        
        \pause
        Договоримся о \alert{канонической} записи!
    \end{itemize}

\end{frame}


\begin{frame}{$MA(\infty)$: итоги}

  \begin{itemize}[<+->]
    \item \alert{Быстро} стремящиеся к нулю коэффициенты. 
    \item \alert{Стационарный процесс}. 
    \item \alert{Пока} не ясно как оценивать.
  \end{itemize}
\end{frame}

