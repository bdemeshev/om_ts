% !TEX root = ../om_ts_02.tex

\begin{frame} % название фрагмента

\videotitle{Новые модели из старых}

\end{frame}



\begin{frame}{Новые модели из старых: план}
  \begin{itemize}[<+->]
    \item Преобразование переменной. 
    \item Разница между ожиданием и медианой.
    \item Усреднение моделей. 
  \end{itemize}
\end{frame}

\begin{frame}{Преобразование переменной}
Помимо модели $y_t \sim ETS(AAA)$ можно оценить:

\pause
\alert{вторую модель} $\ln y_t \sim ETS(AAA)$ или \pause \alert{третью модель} $\sqrt{y_t} \sim ETS(AAA)$.

\pause 
Чтобы сравнивать прогнозы моделей, нужно работать в общем масштабе!

\pause
В \alert{зависимости от софта}: либо сами приводим к исходным единицам, либо это происходит автоматически. 
\end{frame}

\begin{frame}{Преобразование Бокса-Кокса}
Для $y_t$, чей размах колебаний растёт с ростом $y_t$, \alert{разумно попробовать} логарифм 
или преобразование Бокса-Кокса. 
\pause
Логарифм: $y_t \to \ln y_t$.

Преобразование Бокса-Кокса: $y_t \to bc_{\lambda}(y_t)$.

\pause

(\alert{Обобщённое}) преобразование Бокса-Кокса: 
\[
bc_{\lambda}(y_t) =
\begin{cases}
\ln y_t, \text{ если } \lambda = 0, \\
\sign(y_t) (\abs{y_t}^{\lambda} - 1)/\lambda, \text { если } \lambda \neq 0. \\
\end{cases}
\]

\end{frame}


\begin{frame}{Параметр лямбда}

    Как \alert{выбрать} параметр $\lambda$ для перехода $y_t \to bc_{\lambda}(y_t)$?

\[
bc_{\lambda}(y_t) =
\begin{cases}
\ln y_t, \text{ если } \lambda = 0, \\
\sign(y_t) (\abs{y_t}^{\lambda} - 1)/\lambda, \text { если } \lambda \neq 0. \\
\end{cases}
\]
\pause

\begin{itemize}[<+->]
    \item Некоторые модели содержат его внутри себя и \alert{сами подбирают} $\lambda$.
    \item Можно подобрать $\lambda$ самостоятельно, чтобы \alert{стабилизировать амплитуду} колебаний ряда. 
\end{itemize}
\end{frame}


\begin{frame}
    \frametitle{Разница между медианой и ожиданием}

    Для модели $y_t \sim ETS(AAA)$ прогноз $\hat y_{t+h|t}$ означает \alert{две величины}:
    
    \begin{itemize}
        \item Ожидание $\E(y_{t+h} \mid \mathcal F_t)$;
        \item Медиана $\Med(y_{t+h} \mid \mathcal F_t)$.
    \end{itemize}
    
    \pause 

    Для модели $\ln y_t \sim ETS(AAA)$ ожидание и медиана \alert{не совпадают}! 
    \[
        \E(y_{t+h} \mid \mathcal F_t) \neq \Med(y_{t+h} \mid \mathcal F_t).
    \]

    \pause 
    Если $Z \sim \dN(\mu, \sigma^2)$, то:
    \[
        \Med(e^Z) = \exp(\mu), \quad \E(e^Z) = \exp(\mu) \cdot \exp(\sigma^2/2).
    \]


\end{frame}

\begin{frame}{Преобразование предиктивного интервала}

    Построили предиктивный интервал для $z_t = \ln y_t$:
    \[
    z_{T+1} \in [z_{left}; z_{right}].    
    \]

    \pause
    Преобразование \alert{естественное}:
    \[
    y_{T+1} \in [\exp(z_{left}); \exp(z_{right})].    
    \]


    \pause
    Предиктивный интервал для $y_{T+1}$ \alert{не симметричен} ни относительно ожидания, 
    ни относительно медианы. 


\end{frame}

\begin{frame}
    \frametitle{Усредняем модели!}
    Eсть две модели $y_t \sim \text{Model A}$, $y_t \sim \text{Model B}$.

    \pause 
    Создаём \alert{усреднённый алгоритм}:
    \[
    y_t \sim \frac{\text{Model A} + \text{Model B}}{2} = \text{Algorithm C}.    
    \]

    \pause 
    Точечные прогнозы считаем как \alert{среднее} прогнозов:
    \[
        \hat y^c_{t+h\mid t} = \frac{\hat y^a_{t+h\mid t} + \hat y^b_{t+h\mid t}}{2}.
    \]

    \pause
    Строго говоря, это — \alert{алгоритм}.

\end{frame}

\begin{frame}
    \frametitle{Маленькое чудо!}

    Усреднённая модель \alert{может быть лучше} каждой из усредняемых.

    \pause
    Разложение на дисперсию и смещение:
    \[
        \E((y_{t+h} - \hat y_{t+h \mid t})^2) = \Var(y_{t+h} - \hat y_{t+h \mid t}) + (\E(y_{t+h} - \hat y_{t+h \mid t}))^2.
    \]
    \pause
    Для \alert{несмещённых} моделей усреднение может снизить дисперсию ошибки прогноза.     
    

\end{frame}


\begin{frame}{Новые модели из старых: итоги}

  \begin{itemize}[<+->]
    \item Преобразование переменных: логарифм, преобразование Бокса-Кокса.
    \item Разница между ожиданием и медианой. 
    \item Усреднение моделей.
  \end{itemize}
\end{frame}

