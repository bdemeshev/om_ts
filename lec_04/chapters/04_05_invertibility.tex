% !TEX root = ../om_ts_04.tex

\begin{frame} % название фрагмента

\videotitle{Условие обратимости}

\end{frame}



\begin{frame}{Условие обратимости: план}
  \begin{itemize}[<+->]
    \item Два варианта условия. 
    \item Единственность записи.
    \item Возможность \alert{поймать единорога}.
  \end{itemize}

\end{frame}


\begin{frame}
  \frametitle{Помним о проблеме!}

  Машенька: этот $y_t$ — $MA(1)$ процесс с уравнением

  \[
    y_t = 5 + u_t + 0.5 u_{t-1}, \quad \sigma^2_u = 4.
  \]

  Вовочка: этот $y_t$ — $MA(1)$ процесс с уравнением

  \[
    y_t = 5 + u_t + 2 u_{t-1}, \quad \sigma^2_u = 1.
  \]

  \pause
  Хотим \alert{единственной формы} записи для одного $MA(q)$ процесса. 

\end{frame}


\begin{frame}
    \frametitle{Лаговый полином $MA(q)$ части}

    Рассмотрим запись 
    \[
        y_t = 5 + u_t + 0.6u_{t-1} + 0.2u_{t-2},
    \]

    \pause
    представим в виде
    \[
      y_t = 5 + (1 + 0.6 L + 0.2 L^2) y_t.
    \]

    \pause
    \alert{Лаговый многочлен}:
    \[
    P(L) = 1 + 0.6 L + 0.2 L^2.
    \]

\end{frame}

\begin{frame}
    \frametitle{Характеристический полином $MA(q)$ части}

    Рассмотрим запись 
    \[
        y_t = 5 + u_t + 0.6u_{t-1} + 0.2u_{t-2},
    \]

    \pause
    оставим только белый шум
    \[
        u_t + 0.6u_{t-1} + 0.2u_{t-2},
    \]
    \pause 
    подставим геометрическую прогрессию $u_t = \lambda^t$ и сократим
    \[
     \lambda^2 + 0.6 \lambda + 0.2.
    \]

    \pause
    \alert{Характеристический многочлен}:
    \[
    \phi(\lambda) = \lambda^2 + 0.6 \lambda + 0.2.
    \]

\end{frame}

\begin{frame}
    \frametitle{Связь многочленов}

    \[
        y_t = 5 + u_t + 0.6u_{t-1} + 0.2u_{t-2}
    \]

    \begin{block}{Теоремка}
        \[
        P(x) = \phi(1/x)    
        \]
    \end{block}

    \pause
    Ахтунг: \alert{путаница} в названиях!

    $P(L)= 1 + 0.6 L + 0.2 L^2$ — \alert{лаговый} многочлен,

    $\phi(\lambda) = \lambda^2 + 0.6 \lambda + 0.2$ — \alert{характеристический}.


\end{frame}



\begin{frame}
    \frametitle{Условие обратимости}

    \begin{block}{Характеристический вариант}
        Уравнение $MA(q)$ процесса удовлетворяет условию обратимости, если 
        у характеристического многочлена $\phi(\lambda)$ все корни $\abs{\lambda_i} <1$.
    \end{block}

    \pause

    \begin{block}{Лаговый вариант}
        Уравнение $MA(q)$ процесса удовлетворяет условию обратимости, если 
        у лагового многочлена $P(L)$ все корни $\abs{\ell_i} >1$.
    \end{block}
    

\end{frame}

\begin{frame}
    \frametitle{Пример обратимой записи $MA(1)$}

    \[
        y_t = 5 + u_t + 0.5 u_{t-1}, \quad \sigma^2_u = 4.
    \]
    \pause

    \[
    \phi(\lambda) = \lambda + 0.5    
    \]
    \pause
    \[
    \lambda_1 = -0.5.    
    \]

\end{frame}


\begin{frame}
    \frametitle{Пример необратимой записи $MA(1)$}

    \[
        y_t = 5 + u_t + 2 u_{t-1}, \quad \sigma^2_u = 1.
    \]
    \pause

    \[
    \phi(\lambda) = \lambda + 2    
    \]
    \pause
    \[
    \lambda_1 = -2.    
    \]
\end{frame}


\begin{frame}
    \frametitle{Нюанс}

    \begin{block}{Разница}
        Стационарность — это свойство самого процесса $(y_t)$.

        Обратимость — это свойство записи процесса (уравнения) для $(y_t)$.
    \end{block}

    \pause
    \alert{Один и тот же} процесс $(y_t)$ можно записать с помощью обратимого $MA(q)$ уравнения
    и с помощью необратимого $MA(q)$ уравнения. 
    \pause
    В этих уравнениях будут фигурировать разные единороги $(u_t)$.


\end{frame}

\begin{frame}
    \frametitle{Единственность записи}

    \begin{block}{Теорема}
        $MA(q)$ процесс допускает единственную обратимую запись. 
    \end{block}

    \pause
    \begin{block}{Теорема}
        $MA(\infty)$ процесс допускает единственную запись. 
    \end{block}    

\end{frame}

\begin{frame}
    \frametitle{Попутный бонус}

    \begin{block}{Теорема}
        Если запись $MA(q)$ процесса обратима, то $u_t$ можно представить в
        \[
            u_t = c + \sum_{i=0}^{\infty} \pi_i y_{t-i},
        \]
        где $\sum \abs{\pi_i} <\infty$.
    \end{block}
    
    \pause
    Возможность примерно найти $u_t$ иногда важна для интерпретации.

\end{frame}




\begin{frame}{Условие обратимости: итоги}

  \begin{itemize}[<+->]
    \item Корни характеристического многочлена $\abs{\lambda_i} < 1$. 
    \item Корни лагового многочлена $\abs{\ell_i} > 1$. 
    \item $MA(\infty)$ обратим.
    \item Единственность записи.
    \item Возможность восстановить $u_t$.
  \end{itemize}
\end{frame}



