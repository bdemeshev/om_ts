% arara: xelatex
%% arara: xelatex


% https://koalatea.io/r-knn-regression/
% http://freerangestats.info/blog/2017/04/09/propensity-v-regression
% https://economics.stackexchange.com/questions/45335/what-is-the-difference-between-ate-and-att
% https://kosukeimai.github.io/MatchIt/articles/matching-methods.html


\documentclass[14pt,xcolor=dvipsnames]{beamer}


% !TEX root = om_metrics_14.tex

%\usepackage{epsdice} % dice 1-6 for probability :)

% \usepackage[absolute,overlay]{textpos}

% \usefonttheme[onlymath]{serif}

\usefonttheme{professionalfonts}
% by default beamer changes math fonts for better visibility for projection
% this professionalfonst theme removes this behavior


\usepackage[orientation=portrait,size=custom,width=25.4,height=19.05]{beamerposter}




%25,4 см 19,05 см размеры слайда в powerpoint

\usetheme{metropolis}
\metroset{
  %progressbar=none,
  numbering=none,
  subsectionpage=progressbar,
  block=fill
}

%\usecolortheme{seahorse}

\usepackage{xunicode} % хак для акцентов!
% https://tex.stackexchange.com/questions/28003/

\usepackage{fontspec}
\usepackage{polyglossia}
\setmainlanguage{russian}


% \usepackage{fontawesome5} % removed [fixed]
\setmainfont[Ligatures=TeX]{Myriad Pro}
% \setsansfont{Myriad Pro}




% why do we need \newfontfamily:
% http://tex.stackexchange.com/questions/91507/
\newfontfamily{\cyrillicfonttt}{Myriad Pro}
\newfontfamily{\cyrillicfont}{Myriad Pro}
%\newfontfamily{\cyrillicfontbs}{Myriad Pro}
\newfontfamily{\cyrillicfontsf}{Myriad Pro}


% https://tex.stackexchange.com/questions/175860/why-does-unicode-math-break-the-kerning-of-accents-in-combination-with-amssymb
% "You shouldn't be using amssymb together with unicode-math"
\usepackage{amsmath}
\usepackage{amsthm} % amssymb 


% https://tex.stackexchange.com/questions/483722/
% \usepackage[MnSymbol]{mathspec}  % Includes amsmath.
% \usepackage{mathspec}  % Includes amsmath.
% \setmathsfont(Digits,Latin,Greek,Symbols)[Numbers={Lining,Proportional}]{Latin Modern Math}
% mathspec must be loaded earlier than amsmath



%\usepackage{bm}

% \usepackage{fdsymbol} % \nperp

% \usepackage{unicode-math} % \symbf
% \setmathfont{Latin Modern Math}



\usepackage{centernot}

\usepackage{graphicx}

\usepackage{wrapfig}
% \usepackage{animate} % animations :)
% \usepackage{tikz}
%\usetikzlibrary{shapes.geometric,patterns,positioning,matrix,calc,arrows,shapes,fit,decorations,decorations.pathmorphing}
% \usepackage{pifont}
\usepackage{comment}
\usepackage[font=small,labelfont=bf]{caption}
\captionsetup[figure]{labelformat=empty}
% \includecomment{techno}



%Расположение

\setbeamersize{text margin left=15 mm,text margin right=5mm} 
\setlength{\leftmargini}{38 pt}

%\usepackage{showframe}
%\usepackage{enumitem}
% \setlist{leftmargin=5.5mm}


%Цвета от дирекции

\definecolor{dirblack}{RGB}{58, 58, 58}
\definecolor{dirwhite}{RGB}{245, 245, 245}
\definecolor{dirred}{RGB}{149, 55, 53}
\definecolor{dirblue}{RGB}{0, 90, 171}
\definecolor{dirorange}{RGB}{235, 143, 76}
\definecolor{dirlightblue}{RGB}{75, 172, 198}
\definecolor{dirgreen}{RGB}{155, 187, 89}
\definecolor{dircomment}{RGB}{128, 100, 162}

\setbeamercolor{title separator}{bg=dirlightblue!50, fg=dirblue}

%Цвета блоков

% Голубой блок!
\setbeamercolor{block title}{bg=dirblue!30,fg=dirblack}
\setbeamercolor{block title example}{bg=dirlightblue!50,fg=dirblack}
\setbeamercolor{block body example}{bg=dirlightblue!20,fg=dirblack}

\AtBeginEnvironment{exampleblock}{\setbeamercolor{itemize item}{fg=dirblack}}
%\setbeamertemplate{blocks}[rounded][shadow]

% Набор команд для удобства верстки

% Набор команд для структуризации

%\newcommand{\quest}{\faQuestionCircleO}
%\faPencilSquareO \faPuzzlePiece \faQuestionCircleO  \faIcon*[regular]{file} {\textcolor{dirblue}
%\newcommand{\quest}{\textcolor{dirblue}{\boxed{\textbf{?}}}
%\newcommand{\task}{\faIcon{tasks}}
%\newcommand{\exmpl}{\faPuzzlePiece}
%\newcommand{\dfn}{\faIcon{pen-square}}
%\newcommand{\quest}{\textcolor{dirblue}{\faQuestionCircle[regular]}}
%\newcommand{\acc}[1]{\textcolor{dirred}{#1}}
%\newcommand{\accm}[1]{\textcolor{dirred}{#1}}
%\newcommand{\acct}[1]{\textcolor{dirblue}{#1}}
%\newcommand{\acctm}[1]{\textcolor{dirblue}{#1}}
%\newcommand{\accex}[1]{\textcolor{dirblack}{\bf #1}}
%\newcommand{\accexm}[1]{\textcolor{dirblack}{ \mathbf{#1}}}
%\newcommand{\acclp}[1]{\textcolor{dirorange}{\it #1}}
\newcommand{\todo}[1]{\textcolor{dircomment}{\bf #1}}
%\newcommand{\graylink}[1]{{\fontsize{11}{12}\selectfont \textcolor{gray}{#1}}}
%\newcommand{\figcaption}[1]{{\fontsize{18}{20}\selectfont #1}}


\newcommand{\videotitle}[1]{
    {\fontsize{33}{30}\selectfont \textcolor{dirblue}{\textbf{#1}} }

    %\todo{название видеофрагмента}
}

\newcommand{\lecturetitle}[1]{
  {\fontsize{33}{30}\selectfont \textcolor{dirblue}{\textbf{#1}} }

    %\todo{название лекции}
}





%\newcommand{\spcbig}{\vspace{-10 pt}}
%\newcommand{\spcsmall}{\vspace{-5 pt}}

%\usepackage{listings}
%\lstset{
%xleftmargin=0 pt,
%  basicstyle=\small, 
%  language=Python,
  %tabsize = 2,
%  backgroundcolor=\color{mc!20!white}
%}



%\newcommand{\mypart}[1]{\begin{frame}[standout]{\huge #1}\end{frame}}

\setbeamercolor{background canvas}{bg=}

% frame title setup
\setbeamercolor{frametitle}{bg=,fg=dirblue}
\setbeamertemplate{frametitle}[default][left]

\addtobeamertemplate{frametitle}{\hspace*{0.1 cm}}{\vspace*{0.25cm}}


%Шрифты
\setbeamerfont{frametitle}{family=\rmfamily,series=\bfseries,size={\fontsize{33}{30}}}
\setbeamerfont{framesubtitle}{family=\rmfamily,series=\bfseries,size={\fontsize{26}{20}}}


% удобнее знать номер слайда, чтобы вносить правки!  

\setbeamercolor{footline}{fg=dircomment}
\setbeamerfont{footline}{series=\bfseries, size={\fontsize{12}{14}}}
%\setbeamertemplate{footline}[page number]


\defbeamertemplate{footline}{custom footline}
{%
  \hspace*{\fill}%
  \usebeamercolor[fg]{page number in head/foot}%
  \usebeamerfont{page number in head/foot}%
  page: \insertpagenumber\,/\,\insertpresentationendpage%
  \hspace{20pt}%
  slide: \insertframenumber\,/\,\inserttotalframenumber%
  %\hspace*{\fill}
  \vskip2pt%
}
%\setbeamertemplate{footline}[custom footline]

\usepackage{physics}
\usepackage[makeroom]{cancel}



% tikz block

\usepackage{pgfplots}
\pgfplotsset{compat=newest}

\usepackage{tikz}
\usetikzlibrary{calc}
\usetikzlibrary{quotes,angles}
\usetikzlibrary{arrows}
\usetikzlibrary{arrows.meta}
\usetikzlibrary{positioning,intersections,decorations.markings}
\usetikzlibrary{patterns}

\usepackage{tkz-euclide} 
%\tikzset{>=latex}

\tikzset{cross/.style={cross out, draw=black, minimum size=2*(#1-\pgflinewidth), inner sep=0pt, outer sep=0pt},
%default radius will be 1pt. 
cross/.default={5pt}}

\colorlet{veca}{red}
\colorlet{vecb}{blue}
\colorlet{vecc}{olive}


\newcommand{\grid}{\draw[color=gray,step=1.0,dotted] (-2.1,-2.1) grid (9.6,6.1)}

% end tikz block

\newcommand{\R}{\mathbb{R}}
\newcommand{\Rot}{\mathrm{R}}
\newcommand{\HH}{\mathrm{H}}
\newcommand{\Id}{\mathrm{I}}
\newcommand{\RR}{\mathbb{R}}
\newcommand{\ZZ}{\mathbb{Z}}
\newcommand{\la}{\lambda}
\let\P\relax
\newcommand{\P}{\mathbb{P}}
\newcommand{\E}{\mathbb{E}}

\newcommand{\cN}{\mathcal{N}}
\newcommand{\dN}{\mathcal{N}}

\newcommand{\qL}{q_{\text{left}}}
\newcommand{\qR}{q_{\text{right}}}



\newcommand{\ba}{\mathbf{a}}
\newcommand{\be}{\mathbf{e}}
\newcommand{\bb}{\mathbf{b}}
\newcommand{\bc}{\mathbf{c}}
\newcommand{\bd}{\mathbf{d}}
\newcommand{\bx}{\mathbf{x}}
\newcommand{\bff}{\mathbf{f}} % \bf is already def
\newcommand{\bv}{\mathbf{v}}
\newcommand{\bzero}{\mathbf{0}}



\DeclareMathOperator{\Var}{Var}
\DeclareMathOperator{\sVar}{sVar}
\DeclareMathOperator{\Cov}{Cov}
\DeclareMathOperator{\sCov}{sCov}
\DeclareMathOperator{\sCorr}{sCorr}
\DeclareMathOperator{\pCorr}{pCorr}
\DeclareMathOperator{\Corr}{Corr}
\DeclareMathOperator{\Med}{Med}
\let\L\relax
\DeclareMathOperator{\L}{L}


\DeclareMathOperator{\plim}{plim}
\DeclareMathOperator{\sign}{sign}


\newcommand{\graylink}[1]{{\fontsize{11}{12}\selectfont \textcolor{gray}{#1}}}
\newcommand{\figcaption}[1]{{\fontsize{18}{20}\selectfont #1}}





\begin{document}


\begin{frame} % название лекции


\lecturetitle{MA процессы}

\end{frame}


% !TEX root = ../om_ts_04.tex

\begin{frame} % название фрагмента

\videotitle{Стационарные процессы}

\end{frame}



\begin{frame}{Стационарные процессы: план}
  \begin{itemize}[<+->]
    \item Определение стационарного процесса.
    \item Автоковариационная функция.
    \item Случайное блуждание и независимые величины. 
  \end{itemize}

\end{frame}

\begin{frame}
  \frametitle{Стационарный процесс}

  Случайный процесс с \alert{постоянными характеристиками}.
  \pause

  \begin{block}{Стационарность в широком смысле}
    Процесс $(y_t)$ стационарен в \alert{широком смысле}, если для любых $t$ и $k$:
    \[
    \begin{cases}
      \E(y_t ) = \mu \\
      \Cov(y_t, y_{t+k}) = \gamma_k \\          
    \end{cases}
    \]
  \end{block}

  \pause
  \begin{block}{Стационарность в узком смысле}
    Процесс $(y_t)$ стационарен в \alert{узком смысле}, если для любого $k$ 
    закон распределения вектора $(y_t, y_{t+1}, y_{t+2}, \ldots, y_{t+k})$ не зависит от $t$. 
  \end{block}
\end{frame}

\begin{frame}
  \frametitle{Стационарность: много уравнений}

  У стационарного процесса $(y_t)$:

  \[
  \E(y_5) = \E(y_7) = \E(y_{100}) = \E(y_{135}) = \ldots = \mu 
  \]
  \pause
  \[
  \Var(y_5) = \Var(y_7) = \Var(y_{100}) = \Var(y_{135}) = \ldots = \gamma_0 
  \]
  \pause
  \[
  \Cov(y_5, y_7) = \pause \Cov(y_8, y_{10}) = \pause \Cov(y_8, y_6) = \pause \ldots = \gamma_2 
  \]
  \pause
  \[
  \Cov(y_1, y_5) = \pause \Cov(y_8, y_{12}) = \pause \Cov(y_8, y_4) = \pause \ldots = \gamma_4 
  \]
  

\end{frame}


\begin{frame}
  \frametitle{Стационарный процесс: пример}

  \begin{block}{Независимые наблюдения}
    Величины $(y_t)$ независимы и одинаково распределены
    с конечным ожиданием $\mu_y$ и конечной дисперсией $\sigma^2_y$.
  \end{block}

  \pause
  \[
  \mu_y = \E(y_t)  
  \]
  \pause
  \[
  \gamma_0 = \Cov(y_t, y_t) = \Var(y_t) = \sigma^2_y.  
  \]
  \pause
  \[
  \gamma_k = \Cov(y_t, y_{t+k}) = 0, \text{ при } k \geq 1.  
  \]
\end{frame}


\begin{frame}
  \frametitle{Нестационарный процесс: пример}

  \begin{block}{Белый шум}
    \[
    \begin{cases}
    y_0 = \mu \\
    y_t = y_{t-1} + u_t, \text{ при } t \geq 1 \\
    \end{cases},
    \]
    где $u_t$ — белый шум.
    \end{block}

  \pause 
  В явном виде: $y_t = \mu + u_1 + u_2 + \ldots + u_t$.    
  \pause
  \[
  \mu_y = \E(y_t)   
  \]
  \pause
  \[
  \gamma_0 = \Cov(y_t, y_t) = \Var(y_t) = \Var(\mu + u_1 + \ldots + u_t) = t\sigma^2_u.
  \]
  \pause
  \[
  \gamma_k = \Cov(y_t, y_{t+k}) = \Cov(y_t, y_t + u_{t+1} + \ldots + u_{t+k}) = \Var(y_t).
  \]
\end{frame}


\begin{frame}
  \frametitle{Белый шум и случайная выборка}

  тут график!

\end{frame}





\begin{frame}
  \frametitle{Автоковариационная функция}

  \begin{block}{Определение}
    Для стационарного процесса $(y_t)$ функцию $\gamma_k = \Cov(y_t, y_{t+k})$ 
    называют \alert{автоковариационной}. 
  \end{block}
  
  \pause
  \begin{block}{Определение}
    Для стационарного процесса $(y_t)$ функцию $\rho_k = \Corr(y_t, y_{t+k})$ 
    называют \alert{автокорреляционной}. 
  \end{block}
\end{frame}


\begin{frame}
  \frametitle{Связь функций}

\[
\rho_k = \Corr(y_t, y_{y+j}) = \frac{\Cov(y_t, y_{y+j})}{\sqrt{\Var(y_t)\Var(y_{t+k})}} =  \pause
\frac{\gamma_k}{\sqrt{\gamma_0 \gamma_0 }} = \frac{\gamma_k}{\gamma_0}
\]  

\end{frame}



\begin{frame}
  \frametitle{Автоковариационная функция — наше всё!}

  \begin{block}{Теоремка}
    Если вектор $(y_t, y_{t+1}, \ldots, y_{t+k})$ имеет многомерное нормальное распределение 
    при любом количестве компонент, то константа $\mu = \E(y_t)$ и функция $\gamma_k = \Cov(y_t, y_{t+k})$
    полностью определяют конечномерные распределения случайного процесса $(y_t)$.
  \end{block}

\end{frame}



\begin{frame}{Стационарность: итоги}

  \begin{itemize}[<+->]
    \item Постоянные $\E(y_t)$, $\gamma_k = \Cov(y_t, y_{t+k})$.
    \item \alert{Автоковариационная} функция. 
    \item Случайное блуждание нестационарно. 
    \item Случайная выборка стационарна.
  \end{itemize}
\end{frame}



% !TEX root = ../om_ts_04.tex

\begin{frame} % название фрагмента

\videotitle{Частные корреляции}

\end{frame}



\begin{frame}{Частные корреляции: план}
  \begin{itemize}[<+->]
    \item Проекция для случайных величин. 
    \item Общее определение. 
    \item Частная автокорреляционная функция. 
  \end{itemize}

\end{frame}


\begin{frame}
  \frametitle{Геометрия случайных величин}

  \begin{block}{Длина и угол}
    Дисперсия $\Var(R)$ — \alert{квадрат длины} случайной величины.

    Корреляция $\Corr(L, R)$ — \alert{косинус угла} между случайными величинами.
  \end{block}
  \pause
  \begin{block}{Ортогональность}
  Величины $L$ и $R$ \alert{ортогональны}, если $\Cov(L, R) = 0$.     
  \end{block}

\end{frame}

\begin{frame}
  \frametitle{Проекция}

  \begin{block}{Обозначение}
    $Best(L; R_1, R_2, \ldots R_n)$ — линейная комбинация 1 и $R_1$, \ldots, $R_n$, 
    \alert{наиболее} похожая на $L$.
  \end{block}
  \pause
  $\hat L = Best(L; R_1, R_2, \ldots R_n)$ если:
  \begin{itemize}[<+->]
    \item $\hat L = \alpha_0 \cdot 1 + \alpha_1 R_1 + \ldots + \alpha_n R_n$;
    \item Ожидание $\E((L - \hat L)^2)$ минимально. 
  \end{itemize}  
\end{frame}


\begin{frame}
  \frametitle{Как найти проекцию?}

  Хотим найти $\hat L = Best(L; R_1, R_2, \ldots R_n)$:
  \[
    \hat L = \alpha_0 \cdot 1 + \alpha_1 R_1 + \ldots + \alpha_n R_n.
  \]

  Как найти коэффициенты?
  \pause
  
  \begin{itemize}[<+->]
    \item Минимизация:
  
    \[
        \E((L - \hat L)^2) \to \min
    \]
    \item Решение системы:
    \[
      \begin{cases}
        \E(L) = \E(\hat L);  \\
        \Cov(L, R_i) = \Cov(\hat L, R_i) \text{ при всех }i; \\
      \end{cases}    
    \]
  \end{itemize}
\end{frame}

\begin{frame}
  \frametitle{Частная корреляция}

  \begin{block}{Определение}
    \[
    \pCorr(U, D ; R_1, R_2, \ldots, R_n) = \Corr(U^*, D^*), \text{ где} 
    \]
    \[
    U^* = U - Best(U; R_1, R_2, \ldots, R_n), 
    \]
    \[
      D^* = D - Best(D; R_1, R_2, \ldots, R_n). 
    \]    
  \end{block}

\pause
Величины $U^*$ и $D^*$ — это \alert{очищенные} версии $U$ и $D$. 

\[
\Cov(U^*, R_i) = 0, \quad \Cov(D^*, R_i) = 0.
\]

\end{frame}

\begin{frame}
  \frametitle{Два угла на графике}
  \begin{center}
  \begin{tikzpicture}
    \coordinate [label=right:$U$] (U) at (2,9);
    \coordinate [label=right:$U^*$] (Ustar) at (2,2);
    \coordinate [label=left:$D$] (D) at (-3,9);
    \coordinate [label=left:$D^*$] (Dstar) at (-3,2);
    \coordinate [label=left:$R$] (R) at (0,8);
    \coordinate (0) at (0,0);
    \coordinate [label=left:$\color{red} R^{\perp}$] (Rort) at (3, 0);


    \draw[line width=2pt,-stealth] (0)--(U);
    \draw[line width=2pt,-stealth] (0)--(D);
    \draw[line width=2pt,-stealth] (0)--(Ustar);
    \draw[line width=2pt,-stealth] (0)--(Dstar);
    \draw[line width=1pt,-stealth] (0)--(R);
    \draw[dashed] (U)--(Ustar);
    \draw[dashed] (D)--(Dstar);
    \draw[dashed, red] (0,1) ellipse (7 and 2);
   
   
  \end{tikzpicture}
\end{center}
\[
\cos \measuredangle{UD} = \Corr(U, D), \quad \cos \measuredangle{U^*D^*} = \pCorr(U, D; R)
\]

\end{frame}


\begin{frame}
  \frametitle{PACF}

  \begin{block}{Определение}
    Для стационарного процесса $(y_t)$ функцию 
    \[
      \varphi_{kk} = \pCorr(y_t, y_{t+k}; y_{t+1}, \ldots, y_{t+k-1}).
    \] 
    называют \alert{частной автокорреляционной}. 
  \end{block}
\end{frame}

\begin{frame}
  \frametitle{ACF и PACF: интуиция}

  Для \alert{стационарного процесса}!

  \begin{itemize}
    \item ACF:
    \[
      \rho_k = \Corr(y_t, y_{t+k}).
    \]
    \alert{Общая сила} связи $y_t$ и $y_{t+k}$.
    \item PACF:
    \[
      \varphi_{kk} = \pCorr(y_t, y_{t+k}; y_{t+1}, \ldots, y_{t+k-1}).
    \]
    \alert{Сила} связи $y_t$ и $y_{t+k}$ при \alert{разорванных} связях через промежуточные наблюдения.
  \end{itemize}
\end{frame}

\begin{frame}
  \frametitle{Почему двойной индекс?}

  \[
  \varphi_{33} = \pCorr(y_t, y_{t+3}; y_{t+1}, y_{t+2}).  
  \]
  \pause
  \[
  \varphi_{23} = \pCorr(y_t, y_{t+2}; y_{t+1}, y_{t+3}).  
  \]
  \pause
  \[
  \varphi_{13} = \pCorr(y_t, y_{t+1}; y_{t+2}, y_{t+3}).  
  \]
\end{frame}

\begin{frame}
  \frametitle{Выборочная PACF через остатки}

  \begin{block}{Корреляция остатков}
    $PACF_4$ — выборочная корреляция \alert{между остатками} $a_t$ и остатками $b_t$.

    $a_t$ — остатки из регрессии
    \[
      y_t \text{ на } 1, y_{t-1}, y_{t-2}, y_{t-3}.
    \]

    $b_t$ — остатки из регрессии
    \[
      y_{t-4} \text{ на } 1, y_{t-1}, y_{t-2}, y_{t-3}.
    \]
  \end{block}
\end{frame}


\begin{frame}
  \frametitle{Выборочная PACF через коэффициент}

  \begin{block}{Оценка коэффициента}
    $PACF_4$ — оценка \alert{последнего коэффициента} в множественной регрессии:
      \[
        \hat y_t = \hat\beta + \hat\beta_1 y_{t-1} + \ldots + \hat\beta_4 y_{t-4}, \quad PACF_4 = \hat\beta_4.
      \]
  \end{block}

\end{frame}

\begin{frame}
  \frametitle{Выборочная и истинная PACF}
  \begin{itemize}[<+->]
    \item Истинная PACF есть \alert{только у стационарного} процесса. 
    \item Выборочную PACF можно посчитать \alert{у любого} процесса. 
    \item По выборочной PACF иногда \alert{можно судить} о стационарности. 
    \item Оба способа дают состоятельные оценки для стационарного процесса. 
    \item Способ с выборочной корреляцией остатков гарантирует числа из отрезка $[-1;1]$.
  \end{itemize}

\end{frame}

\begin{frame}{Частная корреляция: итоги}

  \begin{itemize}[<+->]
    \item Ковариация задаёт геометрию. 
    \item Частная корреляция — корреляция \alert{очищенных} величин.
    \item Во временных рядах очищаем два наблюдения от \alert{промежуточных}.
    \item Оцениваем частную корреляцию.
  \end{itemize}
\end{frame}




% !TEX root = ../om_ts_04.tex

\begin{frame} % название фрагмента

\videotitle{MA процессы}

\end{frame}



\begin{frame}{MA процессы: план}
  \begin{itemize}[<+->]
    \item Определение и запись с лагами. 
    \item Стационарность. 
    \item ACF и PACF.
    \item Неединственность записи. 
  \end{itemize}

\end{frame}

\begin{frame}
  \frametitle{Лаговый оператор}

  \begin{block}{Определение}
    Для процесса $(y_t)$, определённого при $t \in \mathbb{Z}$, \alert{лагированным} процессом 
    $L y_t$ называют ту же последовательность величин со сдвинутым индексом,
    \[
      L y_t = y_{t-1}.
    \]
  \end{block}

  \pause
  \[
  L^2 y_t = L\cdot L\cdot y_t = L\cdot y_{t-1} = y_{t-2}.  
  \]
  \pause
  \[
  \Delta y_t = y_t - y_{t-1} = (1 - L) y_t.  
  \]
  \pause
  \[
  \Delta_{12} y_t = y_t - y_{t-12} = (1 - L^{12}) y_t.  
  \]
\end{frame}

\begin{frame}
  \frametitle{MA процесс}

  \begin{block}{Определение}
    Процесс $(y_t)$, который \alert{можно} представить в виде
    \[
    y_t = \mu + u_t + \alpha_1 u_{t-1} + \ldots + \alpha_q u_{t-q},  
    \]
    где $\alpha_q \neq 0$ и $(u_t)$ — белый шум, называют $MA(q)$ процессом. 
    
    \alert{MA — Moving Average — скользящее среднее}. 
  \end{block}
  
  \pause
  Пример $MA(1)$ процесса:
  \[
    y_t = 5 + u_t + 0.3 u_{t-1},
  \]
  где $(u_t)$ — некоторый белый шум.
  \pause

  Нормировка коэффициента при $u_t$ \alert{к единице}.

\end{frame}

\begin{frame}
  \frametitle{Запись с лагами}

  \begin{block}{MA с лаговым полиномом}
    Процесс $(y_t)$, который \alert{можно} представить в виде 
    \[
    y_t = \mu + P(L) u_t,  
    \]
    где $P(L)$ — многочлен степени $q$ от лага $L$ с $P(0)=1$, а $(u_t)$ — белый шум,
    называют $MA(q)$ процессом. 
  \end{block}

  \pause
  Пример $MA(2)$ процесса:
  \[
  y_t = 5 + (1 - 0.2 L + 0.3 L^2) u_t,
\]
где $(u_t)$ — белый шум. 
\end{frame}


\begin{frame}
  \frametitle{Стационарность MA}

  \begin{block}{Теорема}
    Любой $MA(q)$ процесс стационарен. 
  \end{block}

  \pause
  \begin{block}{Доказательство на примере}
    \[
      \E(5 + u_t + 0.6u_{t-1} + 0.2u_{t-2}) = 5
    \]
    \pause
    \[
      \Cov(5 + u_t + 0.6u_{t-1} + 0.2u_{t-2}, 5 + u_{t+k} + 0.6u_{t+k-1} + 0.2u_{t+k-2}) = \gamma_k
    \]
    Ковариация для $(u_t)$ определяется \alert{совпадающими} индексами. 

    При изменении $t$ совпадающие индексами пары остаются те же.
  \end{block}
  

\end{frame}

\begin{frame}
  \frametitle{ACF}

  \begin{block}{Теорема}
    У $MA(q)$ процесса теоретическая автокорреляция $\rho_k$ равна нулю при $k>q$
  \end{block}
  \pause
  \begin{block}{Доказательство}
    Считаем $\gamma_3 = \Cov(y_t, y_{t+3})$ для $MA(2)$:
    \[
    \gamma_3 = \Cov(5 + u_t + 0.6u_{t-1} + 0.2u_{t-2}, 5 + u_{t+3} + 0.6u_{t+3-1} + 0.2u_{t+3-2}) = 0
    \]
    \alert{Нет совпадающих} индексов у белого шума!
  \end{block}
  \pause 
  Побочный результат: для $MA(q)$ процесса $\rho_q \neq 0$.
\end{frame}


\begin{frame}
  \frametitle{PACF}

  \begin{block}{Теорема}
    У $MA(q)$ процесса теоретическая частная автокорреляция $\varphi_{kk}$ \alert{экспоненциально} быстро 
    сходится к нулю.
  \end{block}
  \pause
  \[
  \abs{\varphi_{kk}} < b_0 \cdot r^k, \text{ с } r \in (0;1).
  \]

  

\end{frame}

\begin{frame}
  \frametitle{ACF и прогнозы}

  Традиционно $MA(q)$ процесс оценивают предполагая совместную нормальность $(y_t)$. 

  \pause
  Из нулевой $\rho_k=0$ при $k>q$ следует независимость $y_t$ и $y_{t+k}$.
  
  \pause
  Прогнозы на больше, чем $q$ шагов вперёд совершенно одинаковые. 

  \[
      (y_{T+q + 1} \mid \mathcal{F}_T) \sim (y_{T+q + 2} \mid \mathcal{F}_T) \sim (y_{T+q + 3} \mid \mathcal{F}_T) \sim \ldots 
  \]
\end{frame}

\begin{frame}
  \frametitle{Прогнозы для $MA(2)$}

  картинка
  

\end{frame}


\begin{frame}
  \frametitle{А корректно ли определение?}
  
  Нюанс: $(y_t)$ — наблюдаемый ряд, $(u_t)$ — \alert{единорог}.

  \pause

  Машенька: этот $y_t$ — $MA(1)$ процесс.

  Вовочка: этот $y_t$ — $MA(2)$ процесс. 

  \pause 

  Так \alert{не бывает!}

\end{frame}



\begin{frame}
  \frametitle{Однако!}

  Машенька: этот $y_t$ — $MA(1)$ процесс с уравнением

  \[
    y_t = 5 + u_t + 0.5 u_{t-1}, \quad \sigma^2_u = 4.
  \]

  Вовочка: этот $y_t$ — $MA(1)$ процесс с уравнением

  \[
    y_t = 5 + u_t + 2 u_{t-1}, \quad \sigma^2_u = 1.
  \]

  \pause
  Здесь \alert{нет противоречия}: $(u_t)$ — единорог!
\end{frame}

\begin{frame}
  \frametitle{Противоречия нет}

  Машенька: этот $y_t$ — $MA(1)$ процесс с уравнением

  \[
    y_t = 5 + \nu_t + 0.5 \nu_{t-1}, \quad \sigma^2_{\nu} = 4.
  \]

  Вовочка: этот $y_t$ — $MA(1)$ процесс с уравнением

  \[
    y_t = 5 + u_t + 2 u_{t-1}, \quad \sigma^2_u = 1.
  \]
  \pause
  Связь $(u_t)$ и $(\nu_t)$:
  \[
  (1+0.5L)\nu_t = (1+2L)u_t.  
  \]
\end{frame}



\begin{frame}{MA процессы: итоги}

  \begin{itemize}[<+->]
    \item $MA(q)$ — взвешивание нескольких белых шумов. 
    \item $MA(q)$ — стационарный процесс. 
    \item $ACF$ резко зануляется, $PACF$ стремится к нулю.
    \item Неединственность записи. 
  \end{itemize}
\end{frame}





% !TEX root = ../om_ts_04.tex

\begin{frame} % название фрагмента

\videotitle{$MA(\infty)$}

\end{frame}



\begin{frame}{$MA(\infty)$: план}
  \begin{itemize}[<+->]
    \item Определение.
    \item Существование бесконечных сумм. 
    \item Стационарность.
  \end{itemize}
\end{frame}

\begin{frame}
  \frametitle{$MA(\infty)$}

  \begin{block}{Определение}
    Процесс $(y_t)$, который \alert{можно} представить в виде
    \[
    y_t = \mu + u_t + \alpha_1 u_{t-1} + \alpha_2 u_{t-2} + \ldots
    \]
    где $(u_t)$ — белый шум, бесконечное количество $\alpha_i \neq 0$ и 
    $\sum_{i=1}^{\infty} \alpha_i^2 < \infty$, 
    называется $MA(\infty)$ процессом. 
  \end{block}

  \pause 
  $MA(\infty)$:
  \[
  y_t = 5 + u_t + 0.5 u_{t-1} + 0.5^2 u_{t-2} + 0.5^3 u_{t-3} + \ldots
  \]

  \pause
  А \alert{так нельзя}:
  \[
   y_t = 5 + u_t + \frac{1}{\sqrt{2}}u_{t-1} + \frac{1}{\sqrt{3}} u_{t-2} + \frac{1}{\sqrt{4}} u_{t-3} + \ldots
  \]

\end{frame}

\begin{frame}
  \frametitle{Сходимости}
  \begin{block}{Теорема}
    Если 
    $\sum_{i=0}^{\infty} \alpha_i^2 < \infty$ и $(u_t)$ — стационарный процесс с нулевым ожиданием, 
    то последовательность частичных сумм $y^q_t$
    \[
      y^q_t = \sum_{i=0}^q \alpha_i u_{t-i}
    \]
  сходится при $q \to \infty$  \alert{в среднеквадртичном}, \alert{по вероятности} и \alert{по распределению}.
  \end{block}

  Нюанс: сходимость взвешенной суммы гарантирована для стационарного $(u_t)$.  
  \pause
  \begin{block}{Бонус}
    \ldots и получающийся процесс $(y_t)$ стационарен.
  \end{block}
\end{frame}


\begin{frame}
  \frametitle{Виды сходимости: $q \to \infty$}

  \begin{block}{$y^q_t \to y_t$ в среднеквадратичном}
    \[
      \E((y_t - y_t^q)^2) \to 0.
    \]
  \end{block}
\pause
\begin{block}{$y^q_t \to y_t$ по вероятности}
  \[
    \P(\abs{y_t - y_t^q} > \varepsilon) \to 0 \text{ для любого числа } \varepsilon > 0.
  \]
\end{block}

\pause
\begin{block}{$y^q_t \to y_t$ по распределению}
  \[
    \P(y_t^q \leq c)  \to \P(y_t \leq c)
  \]
  в точках непрерывности $F(c) = \P(y_t \leq c)$.
\end{block}

\end{frame}



\begin{frame}
  \frametitle{Теорема Вольда}

  \begin{block}{Теорема}
    Если $(y_t)$ — стационарный процесс, то он представим в виде:
    \[
    y_t = \sum_{i=0}^{\infty} \alpha_i u_{t-i} + r_t,   
    \]
    где 
    \begin{itemize}
      \item $(u_t)$ — белый шум,
      \item $\sum \alpha_i^2 < \infty$,
      \item $r_t$ — линейно \alert{предсказуемый} случайный процесс,
      \item $\Cov(u_t, r_t) = 0$.
    \end{itemize}
  \end{block}

  \pause
  Ахтунг: \alert{deterministic} часто ошибочно переводят как 
  последовательность констант. 

\end{frame}

\begin{frame}
  \frametitle{Предсказуемый процесс}


\begin{block}{Правильное определение}
  Процесс $(r_t)$ называется \alert{линейно предсказуемым}, если
  \begin{itemize}
    \item $(r_t)$ стационарен,
    \item $r_t = \beta_0 + \beta_1 r_{t-1} + \beta_2 r_{t-2} + \ldots + \beta_p r_{t-p}$.
  \end{itemize}

  \end{block}

\end{frame}


\begin{frame}
  \frametitle{$MA(\infty)$: плюсы}

  \begin{itemize}[<+->]
    \item \alert{Стационарный} процесс.
    \item Богатая структура корреляций $\rho_k$.
    \item \alert{Практически} любой стационарный процесс. 
  \end{itemize}

\end{frame}


\begin{frame}
    \frametitle{$MA(\infty)$: проблемы}

    \begin{itemize}[<+->]
        \item Оценить невозможно: \alert{бесконечное} число параметров $\alpha_i$. 
        
        \pause
        Введём \alert{ограничения} на $\alpha_i$!

        \item Да ещё и запись \alert{не единственна}.
        
        \pause
        Договоримся о \alert{канонической} записи!
    \end{itemize}

\end{frame}


\begin{frame}{$MA(\infty)$: итоги}

  \begin{itemize}[<+->]
    \item \alert{Быстро} стремящиеся к нулю коэффициенты. 
    \item \alert{Стационарный процесс}. 
    \item \alert{Пока} не ясно как оценивать.
  \end{itemize}
\end{frame}



% !TEX root = ../om_ts_02.tex

\begin{frame} % название фрагмента

\videotitle{Сравнение моделей}

\end{frame}



\begin{frame}{Сравнение моделей: план}
  \begin{itemize}[<+->]
    \item MAE и ещё куча страшных слов. 
    \item Кросс-валидация.
    \item Критерий Акаике.
  \end{itemize}

\end{frame}


\begin{frame}
  \frametitle{Помните о цели!}

  Если цель построения модели — прогнозы на один шаг вперёд, 
  то разумно сравнивать модели по прогнозной силе на один шаг вперёд. 

  \pause
  Если цель — обнаружить момент разладки,
  то разумно искать модель дающую минимальную ошибку, когда нет разладки, 
  и максимальную ошибку, когда разладка есть. 

\end{frame}

\begin{frame}
    \frametitle{Обозначения для краткости}

    Для прогноза важно, \alert{когда} его строят, и на \alert{сколько шагов вперёд}:
    \[
    \hat y_{t+h \mid t}.    
    \]

    \pause 
    Иногда для \alert{краткости}:
    \[
    \hat y_{t+h}    
    \]
    \pause 
    Проблемка:
    \[
    \hat y_{(t+1) + 2} \neq \hat y_{(t+2) + 1}.    
    \]    
    
\end{frame}


\begin{frame}
    \frametitle{Показатели антикачества}

    \alert{Ошибка прогноза}: $e_{t+h} = y_{t+h} - \hat y_{t+h}$.

    \pause
    \alert{Средняя абсолютная ошибка} (Mean Absolute Error):
    \[
    MAE = \frac{\abs {e_{T+1}} + \abs{e_{T+1}}+ \ldots + \abs{e_{T+H}} }{H}.
    \]
    \pause
    \alert{Средняя квадратичная ошибка} (Root Mean Squared Error):
    \[
        RMSE = \sqrt{ \frac{e^2_{T+1} + e^2_{T+1}+ \ldots + e^2_{T+H} }{H} }.
    \]
    
\end{frame}


\begin{frame}
    \frametitle{Масштабируем}

    Переводим ошибку $e_{t+h} = y_{t+h} - \hat y_{t+h}$  \alert{в проценты} $p_t= e_t/y_t \cdot 100$ или
    $p^s_t = e_t/(0.5 y_t + 0.5\hat y_t) \cdot 100$.

    \pause
    \alert{Средняя абсолютная процентная ошибка} (Mean Absolute Persentage Error):
    \[
    MAPE = \frac{\abs {p_{T+1}} + \abs{p_{T+1}}+ \ldots + \abs{p_{T+H}} }{H}.
    \]
    \pause 
    \alert{Симметричная средняя абсолютная процентная ошибка} (Symmetric Mean Absolute Persentage Error):
    \[
    sMAPE = \frac{\abs {p^s_{T+1}} + \abs{p^s_{T+1}}+ \ldots + \abs{p^s_{T+H}} }{H}.
    \]
    
\end{frame}

\begin{frame}
    \frametitle{Автоматически сравниваем с наивной}

    \alert{Наивный прогноз}: $\hat y^{naive}_t = y_{t-1}$ или $\hat y^{naive}_t = y_{t-12}$.
    \pause
    Отмасштабируем ошибку нашего прогноза $e_t$ к $MAE^{naive}$:
    \[
    q_t = \frac{e_t}{MAE^{naive}}.
    \]

    \pause
    \alert{Средняя абсолютная отмасштабированная ошибка} (Mean Absolute Scaled Error):
    \[
    MASE  = \frac{\abs {q_{T+1}} + \abs{q_{T+1}}+ \ldots + \abs{q_{T+H}} }{H}.
    \]

    \pause 
    Сравнение $q$ с единицей сравнивает нашу модель с наивной. 


\end{frame}


\begin{frame}
    \frametitle{Обучающая и тестовая выборка}

    Стратегия: 
    \begin{enumerate}[<+->]
        \item Делим всю выборку на \alert{обучающую} (в начале) и \alert{тестовую} (в конце).
        \item \alert{Оцениваем} несколько моделей по обучающей выборке.
        \item \alert{Прогнозируем} каждое наблюдение тестовой выборки с помощью каждой модели.
        \item Рассчитываем \alert{ошибки} прогнозов моделей.  
        \item \alert{Сравниваем} модели по $MAE$ и выбираем лучшую.
    \end{enumerate}

    \pause
    Недостаток: \alert{у прогнозов разный горизонт}.

\end{frame}


\begin{frame}
    \frametitle{Деление на обучающую и тестовую}

    картинка с растущими стрелочками-параболками

\end{frame}


\begin{frame}
    \frametitle{Кросс-валидация скользящим окном}

    Стратегия:
    \begin{enumerate}[<+->]
        \item Выбираем стартовый размер \alert{обучающей} выборки (в начале).
        \item \alert{Оцениваем} несколько моделей по обучающей выборке.
        \item \alert{Прогнозируем} на один шаг вперёд с помощью каждой модели. 
        \item Рассчитываем \alert{ошибки} прогнозов моделей.  
        \item \alert{Сдвигаем} обучающую выборку на одно наблюдение вправо. 
        \item Повторяем шаги 2-5.
        \item \alert{Сравниваем} модели по $MAE$ и выбираем лучшую.
    \end{enumerate}

\end{frame}


\begin{frame}
    \frametitle{Кросс-валидация скользящим окном}

    картинка с растущими стрелочками-параболками

\end{frame}



\begin{frame}
    \frametitle{Кросс-валидация растущим окном}

    Стратегия:
    \begin{enumerate}
        \item Выбираем стартовый размер обучающей выборки (в начале).
        \item Оцениваем несколько моделей по обучающей выборке.
        \item Прогнозируем на один шаг вперёд с помощью каждой модели. 
        \item Рассчитываем ошибки прогнозов моделей.  
        \item \alert{Увеличиваем} обучающую выборку на одно наблюдение. 
        \item Повторяем шаги 2-5.
        \item Сравниваем модели по $MAE$ и выбираем лучшую.
    \end{enumerate}

\end{frame}


\begin{frame}
    \frametitle{Кросс-валидация растущим окном}

    картинка с растущими стрелочками-параболками

\end{frame}


\begin{frame}
    \frametitle{Кросс-валидация: обсуждение}

    Кросс-валидация \alert{скользящим} окном: наблюдений много и мы подозреваем, что 
    зависимость изменяется.
    \pause
    Кросс-валидация \alert{растущим} окном: наблюдений мало или мы уверены 
    в том, что зависимость сохраняется.
    \pause
    Кросс-валидация — может быть долгой!

\end{frame}

\begin{frame}
    \frametitle{Сделаем кросс-валидацию по-быстрому!}

    Примерная замена кросс-валидации на один шаг вперёд по $RMSE$.

    \alert{Критерий Акаике} (Akaike Information Criterion):
    \pause
    \[
      AIC = -2 \ln L + 2k,
    \]
    гда $\ln L$ — логарифм максимума правдоподобия на обучающей выборке, $k$ — общее число параметров модели. 
    
\end{frame}

\begin{frame}
    \frametitle{Нюансы $AIC$}

    \begin{itemize}[<+->]
        \item $AIC$ имеет \alert{теоретические основания}:
        \[
            \frac{AIC_A - AIC_B}{2} \approx KL(\text{Truth} || \text{Model A}) - KL(\text{Truth} || \text{Model B}).
        \]
        \item Может использоваться \alert{для невложенных моделей}. 
        \item Для гауссовских моделей $y_t$ критерий аппроксимирует \alert{сравнение по $RMSE$}.
        \item Сравниваемые модели должны моделировать \alert{те же} наблюдения. 
        \item Разный софт может исключать из правдоподобия \alert{разные константы}. 
    \end{itemize}

    

\end{frame}


\begin{frame}{Сравнение моделей: итоги}

  \begin{itemize}[<+->]
    \item MAE, RMSE, MAPE, MASE.
    \item Кросс-валидация: скользящее и растущее окно.
    \item AIC — быстрый примерный аналог кросс-валидации. 
  \end{itemize}
\end{frame}






\end{document}
