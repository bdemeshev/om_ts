% !TEX root = ../om_ts_03.tex

\begin{frame} % название фрагмента

\videotitle{Тета-метод}

\end{frame}


\begin{frame}{Тета-метод: план}
  \begin{itemize}[<+->]
    \item Неожиданный лидер. 
    \item Авторская версия.
    \item Частный случай ETS. 
  \end{itemize}
\end{frame}


\begin{frame}
    \frametitle{Тета-метод}

    Появился в 2000 году и стал сенсацией на \alert{соревнованиях M3} по прогнозированию рядов. 

    \pause

    Работает для \alert{несезонных} рядов. 

    \pause 

    Изначально \alert{без статистической модели}. 

\end{frame}


\begin{frame}
    \frametitle{Авторская версия}

    \begin{enumerate}[<+->]
        \item Раскладываем ряд на две тета-линии ($\theta=0$, $\theta = 2$).
        \item Прогнозируем нулевую линию с помощью линейной регрессии. 
        \item Прогнозируем вторую линию с помощью ETS(ANN).
        \item Усредняем прогнозы.
      \end{enumerate}

      \pause
      Можно предварительно удалить сезонность и в конце вернуть обратно. 

\end{frame}

\begin{frame}
    \frametitle{Что такое тета-линия?}

    Нулевая тета-линия — \alert{регрессия} ряда на время:

    \[
    \hat y_t = \hat \beta_1 + \hat \beta_2 t.    
    \]
    \pause
    
    Тета линия для произвольного тета:
    \[
        \Delta^2 y^{new}_t = \theta \Delta^2 y_t.
    \]

\end{frame}


\begin{frame}{Интуиция}
    
    \begin{itemize}[<+->]
    \item Нулевая тета-линия ловит долгосрочную тенденцию ряда.
    \item Тета-линия ($\theta=2$) ловит краткосрочную тенденцию.
    
    \alert{Ускорение} тета-линии в $\theta$ раза сильнее 
    ускорения исходного ряда. 
    

    \item Усреднение снижает дисперсию прогнозов. 
    \end{itemize}
\end{frame}



\begin{frame}
    \frametitle{Как подбирается тета-линия?}

    Берём $\theta = 2$:
    \[
        \Delta^2 y^{new}_t = 2 \Delta^2 y_t.
    \]
    \pause
    Или
    \[
        y^{new}_t - 2 y^{new}_{t-1} + y^{new}_{t-2}   = 2(y_t - 2 y_{t-1} + y_{t-2}).
    \]

    \pause
    Новый ряд $y^{new}_t$ полностью определяется $y_1^{new}$, $y_{2}^{new}$.

    \pause
    Решаем оптимизационную задачу:
    \[
        \sum_{t=1}^T (y_t - y_t^{new})^2 \to \min.
    \]
\end{frame}


\begin{frame}
    \frametitle{Статистическая модель}

    Уже в 2003 году появилась модель:

    \[
        \begin{cases}
        y_t = \ell_t + b + u_t; \\
        \ell_t = \ell_{t-1} + b + \alpha u_t; \\
        \ell_1 = y_1. \\    
        \end{cases}
    \]

    \pause
    Или:
    \[
        \Delta y_t = b + (\alpha - 1) u_{t-1} + u_t. 
    \]
\end{frame}

\begin{frame}
    \frametitle{Тета-метод — вариант ETS}

    \alert{Основа} — ETS(AAN):
    \[
        \begin{cases}
          y_t = \ell_{t-1} +  b_{t-1} + u_t; \\
         \ell_t = \ell_{t-1} + b_{t-1} + \alpha u_t, \text{ стартовое } \ell_1; \\
         b_t =  b_{t-1} + \beta u_t, \text{ стартовое } b_0; \\
         u_t \sim \dN(0;\sigma^2) \text{ и независимы.} \\
         \end{cases}
    \]
    \pause
    Убираем стохастичность тренда $\beta = 0$. 

    \pause
    Возможны \alert{нюансы} инициализации. 
\end{frame}




\begin{frame}{Тета-метод: итоги}

  \begin{itemize}[<+->]
    \item Хорошо работает для \alert{несезонных} данных.
    \item Особая \alert{вариация} ETS модели. 
  \end{itemize}
\end{frame}

