% arara: xelatex
%% arara: xelatex

\documentclass[12pt]{article}

\usepackage{tikz} % картинки в tikz
\usepackage{microtype} % свешивание пунктуации

\usepackage{array} % для столбцов фиксированной ширины

\usepackage{indentfirst} % отступ в первом параграфе

\usepackage{sectsty} % для центрирования названий частей
\allsectionsfont{\centering}

\usepackage{amsmath, amssymb, amsthm} % куча стандартных математических плюшек

\usepackage{amsfonts}

\usepackage{comment}

\usepackage[top=2cm, left=1.2cm, right=1.2cm, bottom=2cm]{geometry} % размер текста на странице

\usepackage{lastpage} % чтобы узнать номер последней страницы

\usepackage{enumitem} % дополнительные плюшки для списков
%  например \begin{enumerate}[resume] позволяет продолжить нумерацию в новом списке
\usepackage{caption}


\usepackage{hyperref} % гиперссылки

\usepackage{multicol} % текст в несколько столбцов


\usepackage{fancyhdr} % весёлые колонтитулы
\pagestyle{fancy}
\lhead{Анализ временных рядов, Экономический анализ, НИУ-ВШЭ}
\rhead{Домашнее задание}

\rfoot{}
% \rfoot{\thepage/3}
\renewcommand{\headrulewidth}{0.4pt}
\renewcommand{\footrulewidth}{0.4pt}



% \usepackage{todonotes} % для вставки в документ заметок о том, что осталось сделать
% \todo{Здесь надо коэффициенты исправить}
% \missingfigure{Здесь будет Последний день Помпеи}
% \listoftodos - печатает все поставленные \todo'шки


% более красивые таблицы
\usepackage{booktabs}
% заповеди из документации:
% 1. Не используйте вертикальные линии
% 2. Не используйте двойные линии
% 3. Единицы измерения - в шапку таблицы
% 4. Не сокращайте .1 вместо 0.1
% 5. Повторяющееся значение повторяйте, а не говорите "то же"



\usepackage{fontspec}
\usepackage{polyglossia}

\setmainlanguage{russian}
\setotherlanguages{english}

% download "Linux Libertine" fonts:
% http://www.linuxlibertine.org/index.php?id=91&L=1
\setmainfont{Linux Libertine O} % or Helvetica, Arial, Cambria
% why do we need \newfontfamily:
% http://tex.stackexchange.com/questions/91507/
\newfontfamily{\cyrillicfonttt}{Linux Libertine O}

\AddEnumerateCounter{\asbuk}{\russian@alph}{щ} % для списков с русскими буквами
\setlist[enumerate, 2]{label=\asbuk*),ref=\asbuk*}

%% эконометрические сокращения
\let\P\relax
\DeclareMathOperator{\P}{\mathbb{P}}
\DeclareMathOperator{\Cov}{\mathbb{C}ov}
\DeclareMathOperator{\Corr}{\mathbb{C}orr}
\DeclareMathOperator{\Var}{\mathbb{V}ar}
\DeclareMathOperator{\E}{\mathbb{E}}
\DeclareMathOperator{\tr}{trace}
\newcommand \hb{\hat{\beta}}
\newcommand \hs{\hat{\sigma}}
\newcommand \htheta{\hat{\theta}}
\newcommand \s{\sigma}
\newcommand \hy{\hat{y}}
\newcommand \hY{\hat{Y}}
\newcommand \vone{\vec{1}}
\newcommand \e{\varepsilon}
\newcommand \he{\hat{\e}}
\newcommand \z{z}
\newcommand \hVar{\widehat{\Var}}
\newcommand \hCorr{\widehat{\Corr}}
\newcommand \hCov{\widehat{\Cov}}
\newcommand \cN{\mathcal{N}}



\begin{document}


\begin{enumerate}

\item (10 баллов) Рассмотрим $MA(2)$ процесс $y_t = 10 + u_t + 3 u_{t-1}$, где величины $u_t$ независимы и нормально распределены 
$\cN(0;4)$.

\begin{enumerate}
    \item Рассчитайте теоретическую автокорреляционную функцию процесса $ACF$, $\rho_k$.
    \item Рассчитайте первые два значения частной автокорреляционной функции $PACF$, $\phi_{11}$, $\phi_{22}$.
    \item Сгенерируйте траекторию данного процесса длиной 30 наблюдений. 
    Постройте график ряда, график первых десяти значений выборочной $ACF$ и $PACF$. 
    \item Повторите предыдущий пункт для 300 наблюдений. 
    Верно ли, что с ростом числа наблюдений выборочная $ACF$ сходится к истинной $ACF$, а выборочная $PACF$ к истинной $PACF$?
\end{enumerate}


\item  (10 баллов) Рассмотрим случайное блуждание $y_t = 1 + y_{t-1} + u_t$, где величины $u_t$ независимы и нормально распределены 
$\cN(0;4)$, а $y_0 = 10$.

\begin{enumerate}
    \item Рассчитайте $\E(y_t)$, $\Var(y_t)$, $\Cov(y_{10}, y_{20})$.
    \item Сравните $\Corr(y_{10}, y_{20})$ и $\Corr(y_{110}, y_{120})$.
    \item Сгенерируйте траекторию данного процесса длиной 30 наблюдений. 
    Постройте график ряда, график первых десяти значений выборочной $ACF$ и $PACF$. 
    \item Повторите предыдущий пункт для 300 наблюдений. 
    Верно ли, что с ростом числа наблюдений выборочная $ACF$ сходится к истинной $ACF$, а выборочная $PACF$ к истинной $PACF$?
\end{enumerate}


\item (20 баллов)  Возьмите любой несезонный ряд годовой периодичности. 
Можно взять ряд с \url{https://fedstat.ru/}, \url{http://sophist.hse.ru/} или других источников. 

\begin{enumerate}
    \item Постройте график ряда, графики выборочных $ACF$ и $PACF$.
    \item Визуально оцените, есть ли тренд? Похож ли процесс на стационарный?
    \item Оцените для ряда $ETS(AAN)$ модель. 
    \item Выпишите полученные уравнение, использовав оценённые значения параметров вместо параметров.
    \item Получите 80\%-й доверительный интервал на один и два шага вперёд «руками», исходя из выписанных уравнений. 
    \item Получите 80\%-й доверительный интервал на один и два шага вперёд встроенными функциями.
    \item Постройте график прогноза и сам ряд. 
\end{enumerate}

\newpage

\item  (30 баллов) Возьмите любой сезонный ряд квартальной или месячной периодичности. 

\begin{enumerate}
    \item Постройте разложение ряда на составляющие, используя $STL$ алгоритм. 
    Визуализируйте результат для трех разных значений силы сглаживания сезонности. Кратко прокомментируйте.
    
    \item Постройте разложение ряда на составляющие, используя $ETS(AAA)$ модель. 
    
    \item Разделите данные на обучающую и тестовую выборку, выделив на тестовую выборку два года наблюдений. 
    
    \item Оцените $ETS(AAA)$, $ETS(MAM)$, сезонную наивную модель и примените тета-метод с $STL$-разложением по умолчанию и $ETS(AAA)$ для логарифма ряда. 
    
    \item Для каждого подхода найдите $MASE$ на тестовой выборке.
    
    \item Постройте прогноз, усредняющий прогнозы двух лидирующих по $MASE$ подхода. Удалось ли обыграть два усредняемых подхода?
    
\end{enumerate}



Работу следует представить в виде отчёта в pdf формате. 
В начале работы должен идти текст с графиками, в конце работы в качестве приложения должен идти код. 
Общий объем текста (без приложений) должен составлять \textbf{не более 10 страниц}.

Дедлайн сдачи - \textbf{28 февраля 2022, 20:59}. До указанного времени файл в формате pdf должен быть загружен по ссылке
\url{....}.

\end{enumerate}





\end{document}


