% !TEX root = ../om_ts_08.tex

\begin{frame} % название фрагмента

\videotitle{Байесовский подход}

\end{frame}



\begin{frame}{Байесовские подход: план}
  \begin{itemize}[<+->]
    \item Как добавить \alert{предикторы} в $ETS$? 
    \item Идея \alert{байесовского подхода}.
    \item \alert{Апостериорная выборка} параметров модели. 
  \end{itemize}

\end{frame}


\begin{frame}
  \frametitle{Как добавить предикторы в $ETS$?}

  \begin{itemize}
    \onslide<1->{\item Классическая $ETS$ модель \alert{не позволяет} включать предикторы. }
    \onslide<2->{\item А \alert{что мешает} их туда добавить и получить новую модель?}
    
    \onslide<3->{В новой модели может оказаться \alert{слишком много} параметров. }

    \onslide<4->{Качество прогнозов может быть плохим. }

    \onslide<5->{\item Спасительная идея — \alert{регуляризация}. }
    
    \onslide<6->{Рассматриваем модель с большим числом параметров и дополнительной 
    информацией, что параметры \alert{небольшие}.}

  \end{itemize}

\end{frame}

\begin{frame}
  \frametitle{Байесовский подход!}

  \begin{itemize}[<+->]
    \item Трактуем все параметры как ненаблюдаемые \alert{случайные величины}, $\theta = (a, b, c)$.
    \item \alert{Модель} задаёт распределение ряда при заданных параметрах,
    \[
    y_t = a + u_t + b u_{t-1}, \quad u_t \sim \cN(0; c).  
    \]
    \item Добавляем информацию в виде \alert{априорного распределения},
    \[
    a \sim \cN(0;100), \quad b \sim \cN(0, 1), \quad \ln c \sim \cN(0;4).  
    \]
    \item Алгоритмы MCMC (Markov Chain Monte Carlo) позволяют 
    сгенерировать большую выборку из \alert{апостериорного распределение}
    \[
    (a, b, c \mid y_1, y_2, \ldots, y_T).
    \]
\end{itemize}

\end{frame}


\begin{frame}
  \frametitle{Немного про MCMC}
  
\begin{itemize}
  \onslide<1->{\item Выборка из апостериорного распределения $(\theta \mid y)$ позволяет считать \alert{всё!}
  \[
    (a_1, b_1, c_1), (a_2, b_2, c_2), \ldots, (a_S, b_S, c_S).  
  \]}
  \onslide<2->{\item Имея значения параметров можно симулировать \alert{будущие траектории}.}
  \onslide<3->{\item MCMC позволяет работать с моделями \alert{фантастической} сложности.}
  \onslide<4->{\item MCMC работает \alert{медленно}.}
  \onslide<5->{\item Генерируемая выборка только \alert{в пределе} похожа на выборку из апостериорного закона.}
  
\end{itemize}

\end{frame}

\begin{frame}
  \frametitle{Конструктор структурных моделей}

  Компоненты: \alert{тренд}, \alert{сезонность}, \alert{предикторы}, \alert{ошибка}:
  \[
  y_t = \mu_t + s_t + \beta_t x_t +  u_t, \quad u_t \sim \cN(0; \sigma^2_{obs}).  
  \]

  \onslide<2->{Для каждой компоненты есть \alert{куча} вариантов.}

\end{frame}



\begin{frame}{Байесовские структурные модели: итоги}

  \begin{itemize}[<+->]
    \item Трактуем параметры как \alert{случайные величины}. 
    \item Можно оценивать \alert{сложные модели}.
    \item MCMC работает \alert{медленно}.
    \item С помощью MCMC можно сгенерировать большую выборку 
    из \alert{апостериорного} распределения параметров. 
  \end{itemize}
\end{frame}

