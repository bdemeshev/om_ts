% !TEX root = ../om_ts_08.tex

\begin{frame} % название фрагмента

\videotitle{Заполнение пропусков}

\end{frame}



\begin{frame}{Заполнение пропусков: план}
  \begin{itemize}[<+->]
    \item Линейная \alert{интерполяция}.
    \item \alert{Модели} для заполнения пропусков.
    \item Использование \alert{STL-разложения}. 
  \end{itemize}

\end{frame}


\begin{frame}
  \frametitle{Линейная интерполяция}

  \begin{block}{Идея}
    Заполним пропуски так, чтобы восстановленные значения идеально ложились на прямую (образовывали \alert{арифметическую прогрессию}),
    \[
    \Delta y_t^{imp} = const.  
    \]
  \end{block}
  \pause

  Пример: 
  
  10, \alert{NA}, \alert{NA}, 100.
  
  \pause 

  10, \alert{40}, \alert{70}, 100

\end{frame}


\begin{frame}
  \frametitle{Модели для заполнения пропусков}

  \begin{enumerate}[<+->]
    \item Оцениваем модель, \alert{допускающую} пропуски в данных. 
    
    ARIMA подходит! И автоматическая ARIMA тоже!
    \item Пропущенные значения $y_t$ заменяем на условное математическое ожидание, 
    полагая оценённые параметры модели равным истинными,
    \[
    y_t^{imp} = \E(y_t \mid \text{данные}).
    \]
    Используется \alert{фильтр Калмана}.
  \end{enumerate}

  \pause
  Возможность оценивать модель на данных с пропусками сильно зависит от \alert{реализации}. 

\end{frame}

\begin{frame}
  \frametitle{Использование STL-разложения}

  \begin{enumerate}[<+->]
    \item Раскладываем ряд с пропусками на составляющие: 
    
    \[
    y_t = \text{trend}_t + \text{seasonal}_t + \text{remainder}_t = \text{seasonal}_t + \text{deseason}_t.
    \]

    STL восстанавливает \alert{сезонную компоненту} без пропусков!
    \item Восставливаем пропущенные значения десезонированного ряда \alert{линейной} интерполяцией. 
    
    \item Пропущенные значения $y_t$ заменяем на сумму восстановленных десезонированных значений и сезонной составляющей,
    \[
    y_t^{imp} = \text{seasonal}_t + \text{deseason}_t^{imp}.
    \]

  \end{enumerate}

\end{frame}

\begin{frame}
  \frametitle{Зачем заполнять пропуски?}

  \begin{itemize}[<+->]
    \item Иногда заполнение пропусков — \alert{основная задача}. 
    \item Возможность использовать \alert{больше алгоритмов} прогнозирования для восстановленного ряда.
    \item Возможность использовать восстановленный ряд \alert{как предиктор}. 
  \end{itemize}
  

\end{frame}


\begin{frame}{Заполнение пропусков: итоги}

  \begin{itemize}[<+->]
    \item Линейная \alert{интерполяция}: просто и быстро! 
    \item Использование \alert{ARIMA} или более сложных моделей.
    \item \alert{STL-разложение} и восстановление компонент. 
    \item \alert{Вариации} у каждого алгоритма. 
  \end{itemize}
\end{frame}

