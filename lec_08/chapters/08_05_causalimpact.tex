% !TEX root = ../om_ts_08.tex

\begin{frame} % название фрагмента

\videotitle{Как оценить эффект воздействия?}

\end{frame}



\begin{frame}{Оценивание эффекта: план}
  \begin{itemize}[<+->]
    \item Основная идея оценивания.
    \item Чем хорош байесовский подход?
  \end{itemize}

\end{frame}


\begin{frame}
  \frametitle{Оценивание эффекта: идея}

  \begin{itemize}[<+->]
    \item Возьмём \alert{любую} модель. 
    
    \item Разделим выборку на обучающую и тестовую \alert{по точке 
    воздействия}. 
    \item Оценим модель по обучающей выборке и \alert{получим прогноз}.
    \item \alert{Разница} прогноза и фактических значений на тестовой выборке 
    оценивает эффект воздействия.  
  \end{itemize}

\end{frame}


\begin{frame}
  \frametitle{Нюансы идеи}

  \begin{itemize}
    \onslide<1->{\item Модель должна \alert{хорошо} прогнозировать.}

    \onslide<2->{Если сезонность сильна, то даже \alert{наивная сезонная} модель подойдёт.}

    \onslide<3->{\item Разумно \alert{протестировать идею} в точке, где воздействия ещё нет.}
    
    \onslide<4->{\item Можно оценивать эффект воздействия на конкретное значение $y_{T+h}$, 
    а можно оценивать \alert{кумулятивный эффект} на $y_{T+1} + y_{T+2} + \ldots + y_{T+h}$.}

    \onslide<5->{\item Помните о \alert{доверительных} интервалах.}
\end{itemize}

\end{frame}


\begin{frame}
  \frametitle{А зачем тут байесовский подход?}

  \begin{itemize}
    \onslide<1->{\item Позволяет оценивать \alert{более сложные} модели с предикторами.}

  
    \onslide<2->{Возможность более точно оценить эффект воздействия.}

    \onslide<3->{\item Позволяет построить \alert{доверительный интервал} для кумулятивного эффекта.}
    
    \onslide<4->{\alert{Проблема}: прогнозы $\hat y_{T+1}$, \ldots, $\hat y_{T+h}$ нетривиально 
   коррелированы между собой. В частотном подходе сложно получить \alert{явную формулу} для доверительного интервала.}

   \onslide<5->{\alert{Решение}: в байесовском подходе апостериорная выборка параметров модели позволяет 
   сгенерировать множество \alert{гипотетических будущих} траекторий без воздействия.}

   \onslide<6->{\item Графики \alert{будущих} траекторий!}
\end{itemize}

  

\end{frame}
  


\begin{frame}{Оценивание эффекта воздействия: итоги}

  \begin{itemize}[<+->]
    \item Делим выборку на обучающую и тестовую \alert{в точке воздействия}.
    \item Смотрим на \alert{разницу} прогноза модели и фактических значений.
    \item Байесовских подход дарит \alert{доверительные интервалы} для кумулятивного эффекта.
  \end{itemize}
\end{frame}

