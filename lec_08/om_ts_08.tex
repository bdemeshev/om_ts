% arara: xelatex
%% arara: xelatex


% https://koalatea.io/r-knn-regression/
% http://freerangestats.info/blog/2017/04/09/propensity-v-regression
% https://economics.stackexchange.com/questions/45335/what-is-the-difference-between-ate-and-att
% https://kosukeimai.github.io/MatchIt/articles/matching-methods.html


\documentclass[14pt,xcolor=dvipsnames]{beamer}


% !TEX root = om_metrics_14.tex

%\usepackage{epsdice} % dice 1-6 for probability :)

% \usepackage[absolute,overlay]{textpos}

% \usefonttheme[onlymath]{serif}

\usefonttheme{professionalfonts}
% by default beamer changes math fonts for better visibility for projection
% this professionalfonst theme removes this behavior


\usepackage[orientation=portrait,size=custom,width=25.4,height=19.05]{beamerposter}




%25,4 см 19,05 см размеры слайда в powerpoint

\usetheme{metropolis}
\metroset{
  %progressbar=none,
  numbering=none,
  subsectionpage=progressbar,
  block=fill
}

%\usecolortheme{seahorse}

\usepackage{xunicode} % хак для акцентов!
% https://tex.stackexchange.com/questions/28003/

\usepackage{fontspec}
\usepackage{polyglossia}
\setmainlanguage{russian}


% \usepackage{fontawesome5} % removed [fixed]
\setmainfont[Ligatures=TeX]{Myriad Pro}
% \setsansfont{Myriad Pro}




% why do we need \newfontfamily:
% http://tex.stackexchange.com/questions/91507/
\newfontfamily{\cyrillicfonttt}{Myriad Pro}
\newfontfamily{\cyrillicfont}{Myriad Pro}
%\newfontfamily{\cyrillicfontbs}{Myriad Pro}
\newfontfamily{\cyrillicfontsf}{Myriad Pro}


% https://tex.stackexchange.com/questions/175860/why-does-unicode-math-break-the-kerning-of-accents-in-combination-with-amssymb
% "You shouldn't be using amssymb together with unicode-math"
\usepackage{amsmath}
\usepackage{amsthm} % amssymb 


% https://tex.stackexchange.com/questions/483722/
% \usepackage[MnSymbol]{mathspec}  % Includes amsmath.
% \usepackage{mathspec}  % Includes amsmath.
% \setmathsfont(Digits,Latin,Greek,Symbols)[Numbers={Lining,Proportional}]{Latin Modern Math}
% mathspec must be loaded earlier than amsmath



%\usepackage{bm}

% \usepackage{fdsymbol} % \nperp

% \usepackage{unicode-math} % \symbf
% \setmathfont{Latin Modern Math}



\usepackage{centernot}

\usepackage{graphicx}

\usepackage{wrapfig}
% \usepackage{animate} % animations :)
% \usepackage{tikz}
%\usetikzlibrary{shapes.geometric,patterns,positioning,matrix,calc,arrows,shapes,fit,decorations,decorations.pathmorphing}
% \usepackage{pifont}
\usepackage{comment}
\usepackage[font=small,labelfont=bf]{caption}
\captionsetup[figure]{labelformat=empty}
% \includecomment{techno}



%Расположение

\setbeamersize{text margin left=15 mm,text margin right=5mm} 
\setlength{\leftmargini}{38 pt}

%\usepackage{showframe}
%\usepackage{enumitem}
% \setlist{leftmargin=5.5mm}


%Цвета от дирекции

\definecolor{dirblack}{RGB}{58, 58, 58}
\definecolor{dirwhite}{RGB}{245, 245, 245}
\definecolor{dirred}{RGB}{149, 55, 53}
\definecolor{dirblue}{RGB}{0, 90, 171}
\definecolor{dirorange}{RGB}{235, 143, 76}
\definecolor{dirlightblue}{RGB}{75, 172, 198}
\definecolor{dirgreen}{RGB}{155, 187, 89}
\definecolor{dircomment}{RGB}{128, 100, 162}

\setbeamercolor{title separator}{bg=dirlightblue!50, fg=dirblue}

%Цвета блоков

% Голубой блок!
\setbeamercolor{block title}{bg=dirblue!30,fg=dirblack}
\setbeamercolor{block title example}{bg=dirlightblue!50,fg=dirblack}
\setbeamercolor{block body example}{bg=dirlightblue!20,fg=dirblack}

\AtBeginEnvironment{exampleblock}{\setbeamercolor{itemize item}{fg=dirblack}}
%\setbeamertemplate{blocks}[rounded][shadow]

% Набор команд для удобства верстки

% Набор команд для структуризации

%\newcommand{\quest}{\faQuestionCircleO}
%\faPencilSquareO \faPuzzlePiece \faQuestionCircleO  \faIcon*[regular]{file} {\textcolor{dirblue}
%\newcommand{\quest}{\textcolor{dirblue}{\boxed{\textbf{?}}}
%\newcommand{\task}{\faIcon{tasks}}
%\newcommand{\exmpl}{\faPuzzlePiece}
%\newcommand{\dfn}{\faIcon{pen-square}}
%\newcommand{\quest}{\textcolor{dirblue}{\faQuestionCircle[regular]}}
%\newcommand{\acc}[1]{\textcolor{dirred}{#1}}
%\newcommand{\accm}[1]{\textcolor{dirred}{#1}}
%\newcommand{\acct}[1]{\textcolor{dirblue}{#1}}
%\newcommand{\acctm}[1]{\textcolor{dirblue}{#1}}
%\newcommand{\accex}[1]{\textcolor{dirblack}{\bf #1}}
%\newcommand{\accexm}[1]{\textcolor{dirblack}{ \mathbf{#1}}}
%\newcommand{\acclp}[1]{\textcolor{dirorange}{\it #1}}
\newcommand{\todo}[1]{\textcolor{dircomment}{\bf #1}}
%\newcommand{\graylink}[1]{{\fontsize{11}{12}\selectfont \textcolor{gray}{#1}}}
%\newcommand{\figcaption}[1]{{\fontsize{18}{20}\selectfont #1}}


\newcommand{\videotitle}[1]{
    {\fontsize{33}{30}\selectfont \textcolor{dirblue}{\textbf{#1}} }

    %\todo{название видеофрагмента}
}

\newcommand{\lecturetitle}[1]{
  {\fontsize{33}{30}\selectfont \textcolor{dirblue}{\textbf{#1}} }

    %\todo{название лекции}
}





%\newcommand{\spcbig}{\vspace{-10 pt}}
%\newcommand{\spcsmall}{\vspace{-5 pt}}

%\usepackage{listings}
%\lstset{
%xleftmargin=0 pt,
%  basicstyle=\small, 
%  language=Python,
  %tabsize = 2,
%  backgroundcolor=\color{mc!20!white}
%}



%\newcommand{\mypart}[1]{\begin{frame}[standout]{\huge #1}\end{frame}}

\setbeamercolor{background canvas}{bg=}

% frame title setup
\setbeamercolor{frametitle}{bg=,fg=dirblue}
\setbeamertemplate{frametitle}[default][left]

\addtobeamertemplate{frametitle}{\hspace*{0.1 cm}}{\vspace*{0.25cm}}


%Шрифты
\setbeamerfont{frametitle}{family=\rmfamily,series=\bfseries,size={\fontsize{33}{30}}}
\setbeamerfont{framesubtitle}{family=\rmfamily,series=\bfseries,size={\fontsize{26}{20}}}


% удобнее знать номер слайда, чтобы вносить правки!  

\setbeamercolor{footline}{fg=dircomment}
\setbeamerfont{footline}{series=\bfseries, size={\fontsize{12}{14}}}
%\setbeamertemplate{footline}[page number]


\defbeamertemplate{footline}{custom footline}
{%
  \hspace*{\fill}%
  \usebeamercolor[fg]{page number in head/foot}%
  \usebeamerfont{page number in head/foot}%
  page: \insertpagenumber\,/\,\insertpresentationendpage%
  \hspace{20pt}%
  slide: \insertframenumber\,/\,\inserttotalframenumber%
  %\hspace*{\fill}
  \vskip2pt%
}
%\setbeamertemplate{footline}[custom footline]

\usepackage{physics}
\usepackage[makeroom]{cancel}



% tikz block

\usepackage{pgfplots}
\pgfplotsset{compat=newest}

\usepackage{tikz}
\usetikzlibrary{calc}
\usetikzlibrary{quotes,angles}
\usetikzlibrary{arrows}
\usetikzlibrary{arrows.meta}
\usetikzlibrary{positioning,intersections,decorations.markings}
\usetikzlibrary{patterns}

\usepackage{tkz-euclide} 
%\tikzset{>=latex}

\tikzset{cross/.style={cross out, draw=black, minimum size=2*(#1-\pgflinewidth), inner sep=0pt, outer sep=0pt},
%default radius will be 1pt. 
cross/.default={5pt}}

\colorlet{veca}{red}
\colorlet{vecb}{blue}
\colorlet{vecc}{olive}


\newcommand{\grid}{\draw[color=gray,step=1.0,dotted] (-2.1,-2.1) grid (9.6,6.1)}

% end tikz block

\newcommand{\R}{\mathbb{R}}
\newcommand{\Rot}{\mathrm{R}}
\newcommand{\HH}{\mathrm{H}}
\newcommand{\Id}{\mathrm{I}}
\newcommand{\RR}{\mathbb{R}}
\newcommand{\ZZ}{\mathbb{Z}}
\newcommand{\la}{\lambda}
\let\P\relax
\newcommand{\P}{\mathbb{P}}
\newcommand{\E}{\mathbb{E}}

\newcommand{\cN}{\mathcal{N}}
\newcommand{\dN}{\mathcal{N}}

\newcommand{\qL}{q_{\text{left}}}
\newcommand{\qR}{q_{\text{right}}}



\newcommand{\ba}{\mathbf{a}}
\newcommand{\be}{\mathbf{e}}
\newcommand{\bb}{\mathbf{b}}
\newcommand{\bc}{\mathbf{c}}
\newcommand{\bd}{\mathbf{d}}
\newcommand{\bx}{\mathbf{x}}
\newcommand{\bff}{\mathbf{f}} % \bf is already def
\newcommand{\bv}{\mathbf{v}}
\newcommand{\bzero}{\mathbf{0}}



\DeclareMathOperator{\Var}{Var}
\DeclareMathOperator{\sVar}{sVar}
\DeclareMathOperator{\Cov}{Cov}
\DeclareMathOperator{\sCov}{sCov}
\DeclareMathOperator{\sCorr}{sCorr}
\DeclareMathOperator{\pCorr}{pCorr}
\DeclareMathOperator{\Corr}{Corr}
\DeclareMathOperator{\Med}{Med}
\let\L\relax
\DeclareMathOperator{\L}{L}


\DeclareMathOperator{\plim}{plim}
\DeclareMathOperator{\sign}{sign}


\newcommand{\graylink}[1]{{\fontsize{11}{12}\selectfont \textcolor{gray}{#1}}}
\newcommand{\figcaption}[1]{{\fontsize{18}{20}\selectfont #1}}





\begin{document}


\begin{frame} % название лекции


\lecturetitle{Пропуски, аномалии и структурные сдвиги}

\end{frame}


% !TEX root = ../om_ts_08.tex

\begin{frame} % название фрагмента

\videotitle{Заполнение пропусков}

\end{frame}



\begin{frame}{Заполнение пропусков: план}
  \begin{itemize}[<+->]
    \item Линейная \alert{интерполяция}.
    \item \alert{Модели} для заполнения пропусков.
    \item Использование \alert{STL-разложения}. 
  \end{itemize}

\end{frame}


\begin{frame}
  \frametitle{Линейная интерполяция}

  \begin{block}{Идея}
    Заполним пропуски так, чтобы восстановленные значения идеально ложились на прямую (образовывали \alert{арифметическую прогрессию}),
    \[
    \Delta y_t^{imp} = const.  
    \]
  \end{block}
  \pause

  Пример: 
  
  10, \alert{NA}, \alert{NA}, 100.
  
  \pause 

  10, \alert{40}, \alert{70}, 100

\end{frame}


\begin{frame}
  \frametitle{Модели для заполнения пропусков}

  \begin{enumerate}[<+->]
    \item Оцениваем модель, \alert{допускающую} пропуски в данных. 
    
    ARIMA подходит! И автоматическая ARIMA тоже!
    \item Пропущенные значения $y_t$ заменяем на условное математическое ожидание, 
    полагая оценённые параметры модели равным истинными,
    \[
    y_t^{imp} = \E(y_t \mid \text{данные}).
    \]
    Используется \alert{фильтр Калмана}.
  \end{enumerate}

  \pause
  Возможность оценивать модель на данных с пропусками сильно зависит от \alert{реализации}. 

\end{frame}

\begin{frame}
  \frametitle{Использование STL-разложения}

  \begin{enumerate}[<+->]
    \item Раскладываем ряд с пропусками на составляющие: 
    
    \[
    y_t = \text{trend}_t + \text{seasonal}_t + \text{remainder}_t = \text{seasonal}_t + \text{deseason}_t.
    \]

    STL восстанавливает \alert{сезонную компоненту} без пропусков!
    \item Восставливаем пропущенные значения десезонированного ряда \alert{линейной} интерполяцией. 
    
    \item Пропущенные значения $y_t$ заменяем на сумму восстановленных десезонированных значений и сезонной составляющей,
    \[
    y_t^{imp} = \text{seasonal}_t + \text{deseason}_t^{imp}.
    \]

  \end{enumerate}

\end{frame}

\begin{frame}
  \frametitle{Зачем заполнять пропуски?}

  \begin{itemize}[<+->]
    \item Иногда заполнение пропусков — \alert{основная задача}. 
    \item Возможность использовать \alert{больше алгоритмов} прогнозирования для восстановленного ряда.
    \item Возможность использовать восстановленный ряд \alert{как предиктор}. 
  \end{itemize}
  

\end{frame}


\begin{frame}{Заполнение пропусков: итоги}

  \begin{itemize}[<+->]
    \item Линейная \alert{интерполяция}: просто и быстро! 
    \item Использование \alert{ARIMA} или более сложных моделей.
    \item \alert{STL-разложение} и восстановление компонент. 
    \item \alert{Вариации} у каждого алгоритма. 
  \end{itemize}
\end{frame}



% !TEX root = ../om_ts_08.tex

\begin{frame} % название фрагмента

\videotitle{Обнаружение аномалий}

\end{frame}



\begin{frame}{Обнаружение аномалий: план}
  \begin{itemize}[<+->]
    \item Какое наблюдение считать аномальным? 
    \item Алгоритмы обнаружения и исправления аномалий.
    \item Зачем искать аномальные наблюдения?
  \end{itemize}

\end{frame}


\begin{frame}
  \frametitle{Какое наблюдение считать аномальным?}


  \pause 
  Деление наблюдений на аномальные и обычные \alert{субъективно}.

  \pause
  Неформально, аномальное наблюдение \alert{выбивается} из \alert{основной динамики} ряда. 

  \pause
  Что считать «основной динамикой»? Что значит «выбивается»?

\end{frame}

\begin{frame}
  \frametitle{Алгоритм обнаружения аномалий}

  \begin{itemize}
    \onslide<1->{\item Берём любой алгоритм, позволяющий выделять из ряда \alert{остаток} $\hat u_t$.}
    
    \onslide<2->{Подойдут как модели $ARIMA$, $ETS$, \ldots, так и алгоритм $STL$.}

    \onslide<3->{Остатком для моделей называют разницу фактическим значением и прогнозом внутри обучающей выборки.}

    \onslide<4->{\item Оцениваем \alert{стандартную ошибку} остатков.}
    \onslide<5->{\item Если остаток по модулю больше \alert{трех} стандартных ошибок, считаем наблюдение аномальным.}

  \end{itemize}
  
\end{frame}


\begin{frame}
  \frametitle{Исправление аномалий}

  \pause
  Вычитаем из аномального наблюдения остаток:
  \[
  y_t^{imp} = y_t  - \hat u_t.  
  \]

\end{frame}



\begin{frame}
  \frametitle{Зачем искать аномальные наблюдения?}

  \begin{itemize}[<+->]
    \item Иногда обнаружение аномалий — \alert{основная задача}. 
    \item Возможность получить \alert{более точные} прогнозы для исправленного ряда.
    \item Возможность получить \alert{более точные} прогнозы, если использовать исправленный ряд как предиктор.
  \end{itemize}
  

\end{frame}



\begin{frame}{Обнаружение аномалий: итоги}

  \begin{itemize}[<+->]
    \item Берём любой алгоритм (STL, ARIMA, ETS, \ldots), выделяющий из ряда \alert{остаток}.
    \item Есть \alert{куча} специальных алгоритмов.
    \item Если остаток \alert{велик}, то считаем наблюдение аномальным. 
    \item Чтобы исправить аномальное наблюдение, заменяем остаток \alert{на ноль}.
    \item \alert{Исправление} аномальных наблюдений перед прогнозирование может улучшить прогнозы!
  \end{itemize}
\end{frame}



% !TEX root = ../om_ts_08.tex

\begin{frame} % название фрагмента

\videotitle{Обнаружение структурного сдвига}

\end{frame}



\begin{frame}{Обнаружение структурного сдвига: план}
  \begin{itemize}[<+->]
    \item Что такое структурный сдвиг? 
    \item Обнаружение одного структурного сдвига. 
    \item Обнаружение нескольких структурных сдвигов. 
  \end{itemize}

\end{frame}

\begin{frame}
  \frametitle{Что считать структурным сдвигом?}

  \pause 
  Деление временного ряда на периоды между структурными сдвигами \alert{субъективно}.

  \pause
  Неформально, момент структурного сдвига \alert{меняет} поведение ряда. 

  \pause
  Что считать «меняет»?

\end{frame}


\begin{frame}
  \frametitle{Идея обнаружения отдельного сдвига}

  \begin{itemize}
    \onslide<1->{\item Стартуем со штрафной функции, измеряющей \alert{неоднородность}
    наблюдений $y_a$, $y_{a+1}$, \ldots, $y_b$,
    \[
    C(y_{a:b}).
    \]
    }
    \onslide<2->{\item Перебираем все моменты $\tau \in [1;T-1]$ и находим минимум величины
    \[
      C(y_{1:\tau}) + C(y_{\tau+1 : T}).
    \]}
    \onslide<3->{Подозреваем, что сдвиг мог быть в этот момент $\tau^*$.}
    
    \onslide<4->{\item Считаем, что сдвиг был в $\tau^*$, если 
    суммарная неоднородность фрагментов \alert{сильно} меньше неоднородности всего ряда,
    \[
      C(y_{1:\tau^*}) + C(y_{\tau^*+1 : T}) < C(y_{1:T}) - \beta.
    \]
    }
\end{itemize}

\end{frame}


\begin{frame}
  \frametitle{Выбор штрафной функции $C$}

  \begin{itemize}
    \onslide<1->{\item Есть \alert{огромное} вариантов.}
    \onslide<2->{\item Часто берут функцию лог-функцию правдоподобия \alert{некоторой} модели,
    домноженную на минус два:
    \[
      C(y_{a:b}) = -2 \max_{\theta} \ln L(y_a, \ldots, y_b \mid \theta ).
    \]
    }
    \onslide<3->{\alert{Простейший вариант}: считать, что $y_t \sim \cN(\mu, \sigma^2)$ и независимы.}
    \onslide<4->{\item Выбор функции $C$ связан с выбором $\beta$ при проверке наличия сдвига в подозрительной точке $\tau^*$,
    \[
      C(y_{1:\tau^*}) + C(y_{\tau^*+1 : T}) < C(y_{1:T}) - \beta.
    \]
    }
    \onslide<5->{Чем больше параметров в $\theta$, тем больше должно быть $\beta$.}
\end{itemize}

\end{frame}

\begin{frame}
  \frametitle{Как обнаружить много структурных сдвигов?}

  \begin{itemize}
    \onslide<1->{\item Запустить алгоритм по обнаружению \alert{одного} структурного сдвига.}
  
    \onslide<2->{Если алгоритм не обнаружил сдвиг, то считаем, что сдвигов на данном участке нет. }
    
    \onslide<3->{\item Разбиваем исходный ряд на \alert{два участка} согласно обнаруженному структурному сдвигу.}

    \onslide<4->{\item \alert{Рекурсивно} запускаем алгоритм обнаружения 
    одного структурного сдвига на \alert{каждом} обнаруженном участке. }
\end{itemize}


\end{frame}


\begin{frame}
  \frametitle{Преобразования для поиска сдвига}

  Сдвиг может \alert{легче} обнаруживаться на преобразованном ряду. 

  \begin{itemize}
    \onslide<1->{\item \alert{Простые действия} с исходным рядом: логарифм, преобразование Бокса-Кокса, переход к разностям.}

    \onslide<2->{\item Разложение на компоненты и \alert{поиск сдвига в компонентах} ряда: $STL$, $ETS$, \ldots }

\end{itemize}


\end{frame}

  

\begin{frame}
  \frametitle{Зачем искать структурные сдвиги?}

  \begin{itemize}[<+->]
    \item Иногда обнаружение сдвигов — \alert{основная задача}. 
    \item Возможность получить \alert{более точные} прогнозы, если добавить в предикторы 
    дамми-переменную равную единице после сдвига.
    \item Возможность получить \alert{более точные} прогнозы других рядов, если 
    скорректировать структурный сдвиг в предикторе.
  \end{itemize}
  

\end{frame}


\begin{frame}{Обнаружение структурного сдвига: итоги}

  \begin{itemize}[<+->]
    \item Есть \alert{куча} специальных алгоритмов.
    \item Чтобы найти \alert{много} сдвигов, достаточно поискать очередной сдвиг на 
    уже выявленных участках ряда.
    \item $STL$ разложение позволяет искать \alert{сдвиги в компонентах} ряда.
 \end{itemize}
\end{frame}



% !TEX root = ../om_ts_08.tex

\begin{frame} % название фрагмента

\videotitle{Байесовский подход}

\end{frame}



\begin{frame}{Байесовские подход: план}
  \begin{itemize}[<+->]
    \item Как добавить \alert{предикторы} в $ETS$? 
    \item Идея \alert{байесовского подхода}.
    \item \alert{Апостериорная выборка} параметров модели. 
  \end{itemize}

\end{frame}


\begin{frame}
  \frametitle{Как добавить предикторы в $ETS$?}

  \begin{itemize}
    \onslide<1->{\item Классическая $ETS$ модель \alert{не позволяет} включать предикторы. }
    \onslide<2->{\item А \alert{что мешает} их туда добавить и получить новую модель?}
    
    \onslide<3->{В новой модели может оказаться \alert{слишком много} параметров. }

    \onslide<4->{Качество прогнозов может быть плохим. }

    \onslide<5->{\item Спасительная идея — \alert{регуляризация}. }
    
    \onslide<6->{Рассматриваем модель с большим числом параметров и дополнительной 
    информацией, что параметры \alert{небольшие}.}

  \end{itemize}

\end{frame}

\begin{frame}
  \frametitle{Байесовский подход!}

  \begin{itemize}[<+->]
    \item Трактуем все параметры как ненаблюдаемые \alert{случайные величины}, $\theta = (a, b, c)$.
    \item \alert{Модель} задаёт распределение ряда при заданных параметрах,
    \[
    y_t = a + u_t + b u_{t-1}, \quad u_t \sim \cN(0; c).  
    \]
    \item Добавляем информацию в виде \alert{априорного распределения},
    \[
    a \sim \cN(0;100), \quad b \sim \cN(0, 1), \quad \ln c \sim \cN(0;4).  
    \]
    \item Алгоритмы MCMC (Markov Chain Monte Carlo) позволяют 
    сгенерировать большую выборку из \alert{апостериорного распределение}
    \[
    (a, b, c \mid y_1, y_2, \ldots, y_T).
    \]
\end{itemize}

\end{frame}


\begin{frame}
  \frametitle{Немного про MCMC}
  
\begin{itemize}
  \onslide<1->{\item Выборка из апостериорного распределения $(\theta \mid y)$ позволяет считать \alert{всё!}
  \[
    (a_1, b_1, c_1), (a_2, b_2, c_2), \ldots, (a_S, b_S, c_S).  
  \]}
  \onslide<2->{\item Имея значения параметров можно симулировать \alert{будущие траектории}.}
  \onslide<3->{\item MCMC позволяет работать с моделями \alert{фантастической} сложности.}
  \onslide<4->{\item MCMC работает \alert{медленно}.}
  \onslide<5->{\item Генерируемая выборка только \alert{в пределе} похожа на выборку из апостериорного закона.}
  
\end{itemize}

\end{frame}

\begin{frame}
  \frametitle{Конструктор структурных моделей}

  Компоненты: \alert{тренд}, \alert{сезонность}, \alert{предикторы}, \alert{ошибка}:
  \[
  y_t = \mu_t + s_t + \beta_t x_t +  u_t, \quad u_t \sim \cN(0; \sigma^2_{obs}).  
  \]

  \onslide<2->{Для каждой компоненты есть \alert{куча} вариантов.}

\end{frame}



\begin{frame}{Байесовские структурные модели: итоги}

  \begin{itemize}[<+->]
    \item Трактуем параметры как \alert{случайные величины}. 
    \item Можно оценивать \alert{сложные модели}.
    \item MCMC работает \alert{медленно}.
    \item С помощью MCMC можно сгенерировать большую выборку 
    из \alert{апостериорного} распределения параметров. 
  \end{itemize}
\end{frame}



% !TEX root = ../om_ts_08.tex

\begin{frame} % название фрагмента

    \videotitle{Структурная модель как конструктор}
    
    \end{frame}
    


    \begin{frame}{Структурная модель: план}
        \begin{itemize}[<+->]
          \item \alert{Локальный линейный} тренд. 
          \item Два варианта \alert{сезонной} составляющей.
          \item \alert{Динамическая регрессия}. 
        \end{itemize}
      
      \end{frame}
      
      

\begin{frame}
    \frametitle{Тренд}
    \[
    y_t = \mu_t + s_t + \beta_t x_t +  u_t, \quad u_t \sim \cN(0; \sigma^2_{obs}).  
    \]
  
    \pause
    \alert{Локальный линейный тренд}:
    \[
       \mu_t = \mu_{t-1} + \delta_{t-1} + w_{1t}, \quad w_{1t} \sim \cN(0; \sigma^2_{level}).
    \]
    \pause
    Уравнение для наклона $\delta_t$:
    \[
      \delta_{t} = \delta_{t-1} +  w_{2t}, \quad w_{2t} \sim \cN(0; \sigma^2_{slope}).
    \]
    
  \end{frame}
  
  \begin{frame}
    \frametitle{Сезонность с помощью дамми}
    \[
    y_t = \mu_t + s_t + \beta_t x_t +  u_t, \quad u_t \sim \cN(0; \sigma^2_{obs}).  
    \]
  
    \pause
    $\gamma_{it}$ — оценка сезонного эффекта для наблюдения $t-i$ в момент $t$. 
    \[
    s_t = \gamma_{0t},  
    \]
    \pause
    \begin{eqnarray*}
      \gamma_{it} = \gamma{i-1,t-1}, \quad i \in \{1, \ldots, 11\}.  \\
      \gamma_{0t} + \gamma{1,t-1} + \gamma{2,t-1} + \ldots + \gamma_{11,t-1} = w_{3t} \sim \cN(0;\sigma^2_{seas})
    \end{eqnarray*}
  \end{frame}
  
  
  \begin{frame}
    \frametitle{Сезонность с помощью Фурье}
    \[
    y_t = \mu_t + s_t + \beta_t x_t +  u_t, \quad u_t \sim \cN(0; \sigma^2_{obs}).  
    \]
  
    \pause
    \begin{eqnarray*}
       s_t = a_{1t} \cos(\frac{2\pi}{365} t) + b_{1t} \sin(\frac{2\pi}{365} t) + \\
           + a_{2t} \cos(2\cdot \frac{2\pi}{365} t) + b_{2t} \sin(2\cdot\frac{2\pi}{365} t)
    \end{eqnarray*}
    \pause
    \begin{eqnarray*}
      a_{it} = a_{i,t-1} + w_{4it}, \quad w_{4it} \sim \cN(0;\sigma^2_{ai}) \\
      b_{it} = b_{i,t-1} + w_{5it}, \quad w_{5it} \sim \cN(0;\sigma^2_{bi})
    \end{eqnarray*}
    
  \end{frame}
  
  \begin{frame}
    \frametitle{Эволюция зависимости от предиктора}
    \[
    y_t = \mu_t + s_t + \beta_t x_t +  u_t, \quad u_t \sim \cN(0; \sigma^2_{obs}).  
    \]
  
    \pause
    \[
       \beta_{t} = \beta_{t-1} + w_{6t}, \quad  w_{6t} \sim \cN(0;\sigma^2_{reg})
    \]
    
  \end{frame}
  
  

  \begin{frame}{Структурная модель как конструктор: итоги}

    \begin{itemize}[<+->]
      \item Огромное количество \alert{вариантов модели}.
    \item Сезонность: \alert{тригонометрические функции} и \alert{дамми-переменные}.
      \item Все параметры могут \alert{меняться} во времени.
      \item В отличие от $ETS$ модели \alert{много источников} случайности. 
    \end{itemize}
  \end{frame}
  


% !TEX root = ../om_ts_07.tex

\begin{frame} % название фрагмента

\videotitle{Как обойтись без моделей?}

\end{frame}



\begin{frame}{Обойтись без моделей: план}
  \begin{itemize}[<+->]
    \item Как переделать временные ряды в перекрестные данные? 
    \item Добавить \alert{лаги} переменной $y_t$.
    \item Использовать \alert{агрегирующие функции} и \alert{cкользящее} или \alert{растущее} окно. 
  \end{itemize}

\end{frame}


\begin{frame}
  \frametitle{Как обойтись без моделей?}

  \begin{block}{Старые друзья}
    Есть алгоритмы, которые по обучающей выборке зависимой переменной $y$, 
    обучающей матрице предикторов $X$, и новым предикторам $X_F$ строят прогноз $\hat y_F$.      
  \end{block}

  \pause

  \alert{Случайный лес}, \alert{градиентный бустинг}\ldots{ }\pause и даже \alert{обычная регрессия}!

  \pause 

  Можно \alert{усреднять} прогнозы ARIMA/ETS и прогнозы других алгоритмов.

\end{frame}


\begin{frame}
  \frametitle{Как создать предикторы?}

  Из одного столбца $y$ можно создать целую матрицу $X$ предикторов!

  \begin{itemize}[<+->]
    \item Использовать \alert{лаги} $y_{t-k}$. 
    \item Использовать \alert{функции от лагов} в качестве предикторов. 
  \end{itemize}


\end{frame}

\begin{frame}
  \frametitle{Используем лаги $y$}

  Для примера возьмём два лага, $Ly_t$ и $L^2 y_t$.
  \pause

  \alert{Обучающая} выборка:
  \[
  \begin{pmatrix}
    y_3 \\
    y_4 \\
    y_5 \\
    \vdots \\
    y_T 
  \end{pmatrix}  \quad 
  \begin{pmatrix}
    y_1 & y_2 \\
    y_2 & y_3 \\
    y_3 & y_4 \\
    \vdots & \vdots \\
    y_{T-2} & y_{T-1} \\ 
  \end{pmatrix}
  \]
  \pause
  Выборка для \alert{прогнозирования}:
  \[
  \begin{pmatrix}
    ? 
  \end{pmatrix}  \quad 
  \begin{pmatrix}
    y_{T-1} & y_{T} \\ 
  \end{pmatrix}
  \]
  
\end{frame}

\begin{frame}
  \frametitle{Сколько лагов добавить?}

  \begin{itemize}[<+->]
    \item Каждый добавленный лаг \alert{сокращает} обучающую выборку!
    \item Разумно добавить \alert{ближайшие лаги} $Ly_t$, $L^2y_t$.
    \item Для сезонных данных разумно добавить \alert{сезонный лаг} $L^{12} y_t$.
    \item Есть алгоритмы \alert{чувствительные к лишним предикторам}: например, регрессия. 
    \item Есть алгоритмы \alert{нечувствительные к лишним предикторам}: например, случайный лес.
  \end{itemize}
\end{frame}

\begin{frame}
  \frametitle{Функции от лагов}
  При прогнозировании $y_{t}$ \alert{честно} использовать любую функцию от \alert{предыдущих} $y_{t-1}$, $y_{t-2}$, \ldots

  \pause

  Например:
  \begin{itemize}[<+->]
    \item $\Delta y_{t-1} = y_{t-1} - y_{t-2}$;
    \item $\max\{ y_{t-1}, y_{t-2}, y_{t-3} \}$;
    \item $\min\{ y_{t-1}, y_{t-2}, \ldots, y_1\}$.
  \end{itemize}
\end{frame}

\begin{frame}
  \frametitle{Типичный предиктор}

  \begin{itemize}[<+->]
    \item \alert{Агрегирующая функция}:
    
    Минимум, максимум, среднее, медиана, размах, выборочная дисперсия, выборочное стандартное отклонение, \ldots

    \item \alert{Аргумент} агрегирующей функции:
    
    \alert{Скользящее окно}: агрегирующая функция применяется, скажем, к трём предыдущим значениям $y_{t-1}$, $y_{t-2}$, $y_{t-3}$.

    \alert{Растущее окно}: агрегирующая функция применяется ко всем предыдущим значениям $y_{t-1}$, $y_{t-2}$, \ldots,  $y_{1}$.
  \end{itemize}

\end{frame}


\begin{frame}
  \frametitle{Используем функции лагов $y$}

  Для примера возьмём максимум скользящим окном и минимум растущим окном. 
  \pause

  \alert{Обучающая} выборка:
  \[
  \begin{pmatrix}
    y_3 \\
    y_4 \\
    y_5 \\
    \vdots \\
    y_T 
  \end{pmatrix}  \quad 
  \begin{pmatrix}
    \max\{y_1, y_2\} & \min\{y_1, y_2\} \\
    \max\{y_2, y_3\} & \min\{y_1, y_2, y_3\} \\
    \max\{y_3, y_4\} & \min\{y_1, \ldots, y_4\} \\
    \vdots & \vdots \\
    \max\{y_{T-2}, y_{T-1}\} & \min\{y_1, \ldots, y_{T-1}\} \\
  \end{pmatrix}
  \]
  \pause
  Выборка для \alert{прогнозирования}:
  \[
  \begin{pmatrix}
    ? 
  \end{pmatrix}  \quad 
  \begin{pmatrix}
    \max\{y_{T-1}, y_{T}\} & \min\{y_1, \ldots, y_{T}\} \\
  \end{pmatrix}
  \]
  
\end{frame}



\begin{frame}{Обойтись без моделей: итоги}

  \begin{itemize}[<+->]
    \item Помните о \alert{случайном лесе}, \alert{градиентном бустинге} и даже об \alert{обычной регрессии}.
    \item Добавьте \alert{лаги зависимой переменной}.
    \item Добавьте \alert{агрегирующие функции} скользящим и растущим окном. 
  \end{itemize}
\end{frame}



\end{document}
